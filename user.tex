%
%   $Id: user.tex,v 1.32 2005/05/07 13:12:36 michael Exp $
%   This file is part of the FPC documentation.
%   Copyright (C) 1997, by Michael Van Canneyt
%
%   The FPC documentation is free text; you can redistribute it and/or
%   modify it under the terms of the GNU Library General Public License as
%   published by the Free Software Foundation; either version 2 of the
%   License, or (at your option) any later version.
%
%   The FPC Documentation is distributed in the hope that it will be useful,
%   but WITHOUT ANY WARRANTY; without even the implied warranty of
%   MERCHANTABILITY or FITNESS FOR A PARTICULAR PURPOSE.  See the GNU
%   Library General Public License for more details.
%
%   You should have received a copy of the GNU Library General Public
%   License along with the FPC documentation; see the file COPYING.LIB.  If not,
%   write to the Free Software Foundation, Inc., 59 Temple Place - Suite 330,
%   Boston, MA 02111-1307, USA.
%
%%%%%%%%%%%%%%%%%%%%%%%%%%%%%%%%%%%%%%%%%%%%%%%%%%%%%%%%%%%%%%%%%%%%%%%
% Preamble.
\input{preamble.inc}
\begin{latexonly}
  \ifpdf
  \pdfinfo{/Author(Michael Van Canneyt)
           /Title(User's Guide)
           /Subject(Free Pascal User's Guide)
           /Keywords(Free Pascal)
           }
  \fi
\end{latexonly}

%
% Settings
%
\makeindex
%
% Start of document.
%
\begin{document}
\title{Free Pascal :\\ User's Guide}
\docdescription{User's Guide for \fpc, Version \fpcversion}
\docversion{2.2}
\input{date.inc}
\author{Micha\"el Van Canneyt\\Florian Kl\"ampfl}
\maketitle
\tableofcontents

%%%%%%%%%%%%%%%%%%%%%%%%%%%%%%%%%%%%%%%%%%%%%%%%%%%%%%%%%%%%%%%%%%%%%
% Introduction
%%%%%%%%%%%%%%%%%%%%%%%%%%%%%%%%%%%%%%%%%%%%%%%%%%%%%%%%%%%%%%%%%%%%%
\chapter{Introduction}

%%%%%%%%%%%%%%%%%%%%%%%%%%%%%%%%%%%%%%%%%%%%%%%%%%%%%%%%%%%%%%%%%%%%%%%
% About this document
\section{About this document}
This is the user's guide for \fpc . It describes the installation and
use of the \fpc compiler on the different supported platforms.
It does not attempt to give an exhaustive list of all supported commands,
nor a definition of the Pascal language. Look at the
\refref for these things. For a description of the possibilities and the 
inner workings of the compiler, see the
\progref . In the appendices of this document you will find lists of
reserved words and compiler error messages (with descriptions).

This document describes the compiler as it is/functions at the time of
writing. First consult the \file{README} and \file{FAQ} files, distributed 
with the compiler. The \file{README} and \file{FAQ} files are, in case of 
conflict with this manual, authoritative.

%%%%%%%%%%%%%%%%%%%%%%%%%%%%%%%%%%%%%%%%%%%%%%%%%%%%%%%%%%%%%%%%%%%%%%%
% About the compiler
\section{About the compiler}
\fpc is a 32- and 64-bit Pascal compiler. The current version (2.2)
can compile code for the following processors:
\begin{itemize}
\item Intel i386 and higher (i486, Pentium family and higher)
\item AMD64/x86\_64
\item PowerPC
\item PowerPC64
\item SPARC
\item ARM
\item The m68K processor is supported by an older version.
\end{itemize}
The compiler and Run-Time Library are available for the following operating systems:
\begin{itemize}
\item \dos
\item \linux % (Intel, AMD64, Arm, SPARC, PPC and m68k)
\item \amiga (version 0.99.5 only)
\item \windows
\item Mac OS X 
\item \ostwo (optionally using the EMX package, so it also works on DOS/Windows)
\item \freebsd
\item \beos 
\item \solaris
\item \netbsd 
\item \netware
\item \openbsd 
\item MorphOS
\item Symbian
\end{itemize}
The complete list is at all times available on the Free Pascal website.

\fpc is designed to be, as much as possible, source compatible with
Turbo Pascal 7.0 and Delphi 7 (although this goal is not yet attained),
but it also enhances these languages with elements like operator overloading.
And, unlike these ancestors, it supports multiple platforms.

It also differs from them in the sense that you cannot use compiled units
from one system for the other, i.e. you cannot use TP compiled units.

Also, there is a text version of an Integrated Development Environment (IDE) 
available for \fpc. Users that prefer a graphical IDE can have a look at the
Lazarus or MSIDE projects.

\fpc consists of several parts :
\begin{enumerate}
\item The compiler program itself.
\item The Run-Time Library (RTL).
\item The packages. This is a collection of many utility units, ranging from
the whole Windows 32 API, through native ZIP/BZIP file handling to the whole GTK-2 interface.
\item The Free Class Library. This is a set of class-based utility units which give
a database framework, image support, web support, XML support and many many more.
\item Utility programs and units.
\end{enumerate}

Of these you only need the first two, in order to be able to use the compiler.
In this document, we describe the use of the compiler and utilities. 
The Pascal Language is described in the \refref, and the available routines
(units) are described in the RTL and FCL Unit reference guides.

%%%%%%%%%%%%%%%%%%%%%%%%%%%%%%%%%%%%%%%%%%%%%%%%%%%%%%%%%%%%%%%%%%%%%%%
% Getting more information.
\section{Getting more information.}
If the documentation doesn't give an answer to your questions,
you can obtain more information on the Internet, at the following addresses:
\begin{itemize}
\item
\seeurl{http://www.freepascal.org/}
{http://www.freepascal.org} is the main
site. It contains also useful mail addresses and
links to other places.
It also contains the instructions for subscribing to the
\textit{mailinglist}.

\item
\seeurl{http://community.freepascal.org:10000/}
{http://community.freepascal.org:10000/} is a forum site where
questions can be posted.
\end{itemize}
Other than that, some mirrors exist.

Finally, if you think something should be added to this manual
(entirely possible), please do not hesitate and contact me at
\seeurl{michael@freepascal.org}{mailto:michael@freepascal.org}.
.

Let's get on with something useful.

%%%%%%%%%%%%%%%%%%%%%%%%%%%%%%%%%%%%%%%%%%%%%%%%%%%%%%%%%%%%%%%%%%%%%
% Installation
%%%%%%%%%%%%%%%%%%%%%%%%%%%%%%%%%%%%%%%%%%%%%%%%%%%%%%%%%%%%%%%%%%%%%

\chapter{Installing the compiler}
\label{ch:Installation}

%%%%%%%%%%%%%%%%%%%%%%%%%%%%%%%%%%%%%%%%%%%%%%%%%%%%%%%%%%%%%%%%%%%%%%%
% Before Installation : Requirements
\section{Before Installation : Requirements}

%
% System requirements
%
\subsection{Hardware requirements}
The compiler needs at least one of the following processors:
\begin{enumerate}
\item An Intel 80386 or higher processor. A coprocessor 
is not required, although it will slow down your program's performance if you do 
floating point calculations without a coprocessor, since emulation will be used.
\item An AMD64 or EMT64 processor. 
\item A PowerPC processor. 
\item A SPARC processor
\item An ARM processor.
\item Older FPC versions exist for the motorola 68000 processor, 
but these are no longer maintained.
\end{enumerate}


Memory and disk requirements:
\begin{enumerate}
\item 8 Megabytes of free memory. This is sufficient to allow compilation of small programs. 
\item Large programs (such as the compiler itself) will require at least 64 MB. 
of memory, but 128MB is recommended. 
(Note that the compiled programs themselves do not need so much memory.)
\item At least 80 MB free disk space. 
When the sources are installed, another 270 MB are needed.
\end{enumerate}

% Software requirements
\subsection{Software requirements}

\subsubsection{Under DOS}
The \dos distribution contains all the files you need to run the compiler
and compile Pascal programs.

\subsubsection{Under UNIX}
Under \unix systems (such as \linux) you need to have the following programs 
installed :
\begin{enumerate}
\item \gnu \file{as}, the \gnu assembler.
\item \gnu \file{ld}, the \gnu linker.
\item Optionally (but highly recommended) : \gnu \file{make}. For easy
recompiling of the compiler and Run-Time Library, this is needed.
\end{enumerate}

\subsubsection{Under Windows}
The \windows distributions (both 32 and 64 bit) contain all the files you need to run the compiler
and compile Pascal programs. However, it may be a good idea to install
the \file{mingw32} tools or the \var{cygwin} development tools. Links
to both of these tools can be found on \var{http://www.freePascal.org}

\subsubsection{Under OS/2}
While the \fpc distribution comes with all necessary tools, it is a good
idea to install the EMX extender in order to compile and run
programs with the Free Pascal compiler. The EMX extender can be found on:\\
\var{http://www.leo.org/pub/comp/os/os2/leo/gnu/emx+gcc/index.html}

\subsubsection{Under Mac OS X}
Mac OS X 10.1 or higher is required, and the developer tools or XCode 
should be installed.

%%%%%%%%%%%%%%%%%%%%%%%%%%%%%%%%%%%%%%%%%%%%%%%%%%%%%%%%%%%%%%%%%%%%%%%
% Installing the compiler.
\section{Installing the compiler.}
The installation of \fpc is easy, but is platform-dependent.
We discuss the process for each platform separately.

% Installing under DOS
\subsection{Installing under Windows}
For \windows, there is a \windows installer, \file{setup.exe}. This is a
normal installation program, which offers the usual options
of selecting a directory, and which parts of the distribution you want
to install. It will, optionally, associate the \file{.pp} or \var{.pas}
extensions with the text mode IDE.

It is not recommended to install the compiler in a directory which
has spaces in it's path name. Some of the external tools do not support
filenames with spaces in them, and you will have problems creating
programs.

\subsection{Installing under DOS or OS/2}
\subsubsection{Mandatory installation steps.}
First, you must get the latest distribution files of \fpc. They come as zip
files, which you must unzip first, or you can download the compiler as a
series of separate files. This is especially useful if you have a slow
connection, but it is also nice if you want to install only some parts of the
compiler distribution.  The distribution zip files for DOS or OS/2 contain an
installation program \file{INSTALL.EXE}. You must run this program to install
the compiler.

The screen of the DOS or OS/2 installation program looks like figure 
\ref{fig:install1}.

\FPCpic{The \dos install program screen}{}{install1}
\FPCpic{}{}{install2}

The program allows you to select:
\begin{itemize}
\item What components you wish to install. e.g do you want the sources or
not, do you want docs or not. Items that you didn't download when
downloading as separate files, will not be enabled, i.e. you can't
select them.

\item Where you want to install (the default location is \verb|C:\PP|).
\end{itemize}

In order to run \fpc from any directory on your system, you must extend
your path variable to contain the \verb|C:\PP\BIN| directory.
Usually this is done in the \file{AUTOEXEC.BAT} file.
It should look something like this :
\begin{verbatim}
  SET PATH=%PATH%;C:\PP\2.2\BIN\i386-DOS
\end{verbatim}
for \dos or 
\begin{verbatim}
  SET PATH=%PATH%;C:\PP\2.2\BIN\i386-OS2
\end{verbatim}
for \ostwo.
(Again, assuming that you installed in the default location).

On \ostwo, \fpc installs some libraries from the EMX package if they
were not yet installed. (The installer will notify you if they should be
installed). They are located in the 
\begin{verbatim}
C:\PP\DLL
\end{verbatim}
directory. The name of this directory should be added to the \var{LIBPATH}
directive in the \file{config.sys} file:
\begin{verbatim}
LIBPATH=XXX;C:\PP\DLL
\end{verbatim}
Obviously, any existing directories in the \var{LIBPATH} directive
(indicated by \var{XXX} in the above example) should be preserved.

\subsubsection{Optional Installation: The coprocessor emulation}
For people who have an older CPU type, without math coprocessor (i387)
it is necessary to install a coprocessor emulation, since \fpc uses the
coprocessor to do all floating point operations.

The installation of the coprocessor emulation is handled by the
installation program (\file{INSTALL.EXE}) under \dos and \windows.

%
% Installing under Linux
%
\subsection{Installing under Linux}
\subsubsection{Mandatory installation steps.}
The \linux distribution of \fpc comes in three forms:
\begin{itemize}
\item a \file{tar.gz} version, also available as separate files.
\item a \file{.rpm} (Red Hat Package Manager) version, and
\item a \file{.deb} (Debian) version.
\end{itemize}

If you use the \file{.rpm} format, installation is limited to
\begin{verbatim}
rpm -i fpc-X.Y.Z-N.ARCH.rpm
\end{verbatim}
Where \var{X.Y.Z} is the version number of the \file{.rpm} file, 
and \var{ARCH} is one of the supported architectures (i386, x86\_64 etc.).

If you use Debian, installation is limited to
\begin{verbatim}
dpkg -i fpc-XXX.deb
\end{verbatim}
Here again, \var{XXX} is the version number of the \file{.deb} file.

You need root access to install these packages. The \file{.tar} file
allows you to do an installation below your home directory if you 
don't have root permissions.

When downloading the \var{.tar} file, or the separate files,
installation is more interactive.

In case you downloaded the \file{.tar} file, you should first untar
the file, in some directory where
you have write permission, using the following command:
\begin{verbatim}
tar -xvf fpc.tar
\end{verbatim}
We supposed here that you downloaded the file \file{fpc.tar} somewhere
from the Internet. (The real filename will have some version number in it,
which we omit here for clarity.)

When the file is untarred, you will be left with more archive files, and
an install program: an installation shell script.

If you downloaded the files as separate files, you should at least download
the \file{install.sh} script, and the libraries (in \file{libs.tar.gz}).

To install \fpc, all that you need to do now is give the following command:
\begin{verbatim}
./install.sh
\end{verbatim}
And then you must answer some questions. They're very simple, they're
mainly concerned with 2 things :
\begin{enumerate}
\item Places where you can install different things.
\item Deciding if you want to install certain components (such as sources
and demo programs).
\end{enumerate}
The script will automatically detect which components are present and can be
installed. It will only offer to install what has been found.
Because of this feature, you must keep the original names when downloading,
since the script expects this.

If you run the installation script as the \var{root} user, you can just accept all installation
defaults. If you don't run as \var{root}, you must take care to supply the
installation program with directory names where you have write permission,
as it will attempt to create the directories you specify.
In principle, you can install it wherever you want, though.

At the end of installation, the installation program will generate a
configuration file (\file{fpc.cfg}) for the \fpc compiler which 
reflects the settings that you chose. It will install this file in 
the \file{/etc} directory or in your home directory (with name
\file{.fpc.cfg}) if you do not have write permission in the \file{/etc}
directory. It will make a copy in the directory where you installed the 
libraries.

The compiler will first look for a file \file{.fpc.cfg} in your home 
directory before looking in the \file{/etc} directory.

%%%%%%%%%%%%%%%%%%%%%%%%%%%%%%%%%%%%%%%%%%%%%%%%%%%%%%%%%%%%%%%%%%%%%%%%%%
% Optional configuration
%%%%%%%%%%%%%%%%%%%%%%%%%%%%%%%%%%%%%%%%%%%%%%%%%%%%%%%%%%%%%%%%%%%%%%%%%%
\section{Optional configuration steps}
On any platform, after installing the compiler you may wish to set
some environment variables. The \fpc compiler recognizes the 
following variables :

\begin{itemize}
\item \verb|PPC_EXEC_PATH| contains the directory where support files for
the compiler can be found.
\item \verb|PPC_CONFIG_PATH| specifies an alternate path to find the \file{fpc.cfg}.
\item \verb|PPC_ERROR_FILE|  specifies the path and name of the error-definition file.
\item \verb|FPCDIR| specifies the root directory of the \fpc installation.
(e.g : \verb|C:\PP\BIN|)
\end{itemize}

These locations are, however, set in the sample configuration file which is
built at the end of the installation process, except for the
\verb|PPC_CONFIG_PATH| variable, which you must set if you didn't install
things in the default places.

\section{Before compiling}

Also distributed in \fpc is a README file. It contains the latest
instructions for installing \fpc, and should always be read first.

Furthermore, platform-specific information and common questions
are addressed in the \var{FAQ}. It should be read before reporting any 
bug.


%%%%%%%%%%%%%%%%%%%%%%%%%%%%%%%%%%%%%%%%%%%%%%%%%%%%%%%%%%%%%%%%%%%%%%%
% Testing the compiler
\section{Testing the compiler}

After the installation is completed and the optional environment variables 
are set as described above, your first program can be compiled.

Included in the \fpc distribution are some demonstration programs,
showing what the compiler can do.
You can test if the compiler functions correctly by trying to compile
these programs.

The compiler is called
\begin{itemize}
\item \file{fpc.exe} under \windows, \ostwo and \dos.
\item \file{fpc} under most other operating systems.
\end{itemize}
To compile a program (e.g \verb|demo\text\hello.pp|), copy the program
to your current working directory, and simply type :
\begin{verbatim}
  fpc hello
\end{verbatim}
at the command prompt. If you don't have a configuration file, then you may
need to tell the compiler where it can find the units, for instance as
follows:
\begin{verbatim}
fpc -Fuc:\pp\NNN\units\i386-go32v2\rtl hello
\end{verbatim}
under \dos, and under \linux you could type
\begin{verbatim}
fpc -Fu/usr/lib/fpc/NNN/units/i386-linux/rtl hello
\end{verbatim}
(replace \var{NNN} with the version number of \fpc that you are using).
This is, of course, assuming that you installed under \verb|C:\PP| or
\file{/usr/lib/fpc/NNN}, respectively.

If you got no error messages, the compiler has generated an executable
called \file{hello.exe} under \dos, \ostwo or \windows, or \file{hello} 
(no extension) under \unix and most other operating systems.

To execute the program, simply type :
\begin{verbatim}
  hello
\end{verbatim}
or
\begin{verbatim}
  ./hello
\end{verbatim}
on Unices (where the current directory usually is not in the PATH). 

If all went well, you should see the following friendly greeting:
\begin{verbatim}
Hello world
\end{verbatim}

%%%%%%%%%%%%%%%%%%%%%%%%%%%%%%%%%%%%%%%%%%%%%%%%%%%%%%%%%%%%%%%%%%%%%
% Usage
%%%%%%%%%%%%%%%%%%%%%%%%%%%%%%%%%%%%%%%%%%%%%%%%%%%%%%%%%%%%%%%%%%%%%
\chapter{Compiler usage}
\label{ch:Usage}

Here we describe the essentials to compile a program and a unit.
For more advanced uses of the compiler, see the section on configuring 
the compiler, and the \progref{}.

The examples in this section suppose that you have an \file{fpc.cfg} which
is set up correctly, and which contains at least the path setting for the
RTL units. In principle this file is generated by the installation program.
You may have to check that it is in the correct place. (see section
\ref{se:configfile} for more information on this.)

%%%%%%%%%%%%%%%%%%%%%%%%%%%%%%%%%%%%%%%%%%%%%%%%%%%%%%%%%%%%%%%%%%%%%%%
% Where the compiler looks for its files.
\section{File searching}
Before you start compiling a program or a series of units, it is
important to know where the compiler looks for its source files and other
files. In this section we discuss this, and we indicate how to influence
this.

\begin{remark}
The use of slashes (/) and backslashes (\verb+\+) as directory separators
is irrelevant, the compiler will convert to whatever character is used on
the current operating system. Examples will be given using slashes, since
this avoids problems on \unix systems (such as \linux).
\end{remark}

% Command line files.
\subsection{Command line files}
The file that you specify on the command line, such as in
\begin{verbatim}
fpc foo.pp
\end{verbatim}
will be looked for ONLY in the current directory. If you specify a directory
in the filename, then the compiler will look in that directory:
\begin{verbatim}
fpc subdir/foo.pp
\end{verbatim}
will look for \file{foo.pp} in the subdirectory \file{subdir} of the current
directory.

Under case sensitive file systems (such as \linux and \unix), the name of this 
file is case sensitive; under other operating systems (such as \dos, \windowsnt, \ostwo) 
this is not the case.

% Unit files.
\subsection{Unit files}
\label{se:unitsearching}

When you compile a unit or program that needs other units, the compiler will
look for compiled versions of these units in the following way:
\begin{enumerate}
\item It will look in the current directory.
\item It will look in the directory where the source file resides.
\item It will look in the directory where the compiler binary is.
\item It will look in all the directories specified in the unit search path.
\end{enumerate}

You can add a directory to the unit search path with the (\seeo{Fu})
option. Every occurrence of one of these options will {\em insert}
a directory to the unit search path. i.e. the last path on the command line
will be searched first.

The compiler adds several paths to the unit search path:
\begin{enumerate}
\item The contents of the environment variable \var{XXUNITS}, where \var{XX}
must be replaced with one of the supported targets: \var{GO32V2},
\var{LINUX},\var{WIN32}, \var{OS2}, \var{BEOS}, \var{FREEBSD}, \var{NETBSD}.
\item The standard unit directory. This directory is determined
from the \var{FPCDIR} environment variable. If this variable is not set,
then it is defaulted to the following:
\begin{itemize}
\item On \linux:
\begin{verbatim}
  /usr/local/lib/fpc/FPCVERSION
or
  /usr/lib/fpc/FPCVERSION
\end{verbatim}
whichever is found first.
\item On other OSes: the compiler binary directory, with '../' appended
to it, if it exists. For instance, on Windows, this would mean
\begin{verbatim}
C:\FPC\2.2\units\i386-win32
\end{verbatim}
This is assuming the compiler was installed in the directory
\begin{verbatim}
C:\FPC\2.2
\end{verbatim}
\end{itemize}
After this directory is determined , the following paths are added to the
search path:
\begin{enumerate}
\item FPCDIR/units/FPCTARGET
\item FPCDIR/units/FPCTARGET/rtl
\end{enumerate}
Here target must be replaced by the name of the target you are compiling
for: this is a combination of CPU and OS, so for instance
\begin{verbatim}
/usr/local/lib/fpc/2.2/units/i386-linux/
\end{verbatim}
or, when cross-compiling
\begin{verbatim}
/usr/local/lib/fpc/2.2/units/i386-win32/
\end{verbatim}
\end{enumerate}

The \var{-Fu} option accepts a single \var{*} wildcard, which will be
replaced by all directories found on that location, but {\em not} the
location itself.
For example, given the directories
\begin{verbatim}
rtl/units/i386-linux
fcl/units/i386-linux
packages/base
packages/extra
\end{verbatim}
the command
\begin{verbatim}
fpc -Fu"*/units/i386-linux"
\end{verbatim}
will have the same effect as
\begin{verbatim}
fpc -Furtl/units/i386-linux -Fufcl/units/i386-linux
\end{verbatim}
since both the \file{rtl} and \file{fcl} directories contain further
\file{units/i386-linux} subdirectories. The packages directory will not be
added, since it doesn't contain a \file{units/i386-linux} subdirectory.

The following command
\begin{verbatim}
fpc -Fu"units/i386-linux/*"
\end{verbatim}
will match any directory below the \file{units/i386-linux} directory, 
but will not match the \file{units/i386-linux} directory itself, so
you should add it manually if you want the compiler to look for files
in this directory as well:
\begin{verbatim}
fpc -Fu"units/i386-linux" -Fu"units/i386-linux/*"
\begin{verbatim}

Note that (for optimization) the compiler will drop any non-existing paths 
from the search path, i.e. the existence of the path (after wildcard and
environment variable expansion) will be tested.

You can see what paths the compiler will search by giving the compiler
the \var{-vu} option.

On systems where filenames are case sensitive (such as \unix and \linux), 
the compiler will :
\begin{enumerate}
\item Search for the original file name, i.e. preserves case.
\item Search for the filename all lowercased.
\item Search for the filename all uppercased.
\end{enumerate}
This is necessary, since Pascal is case-independent, and the statements 
\var{Uses Unit1;} or \var{uses unit1;} should have the same effect.

It will do this first with the extension \file{.ppu} (the compiled unit),
\file{.pp} and then with the extension \file{.pas}.

For instance, suppose that the file \file{foo.pp} needs the unit
\file{bar}. Then the command
\begin{verbatim}
fpc -Fu.. -Fuunits foo.pp
\end{verbatim}
will tell the compiler to look for the unit \file{bar} in the following
places:
\begin{enumerate}
\item In the current directory.
\item In the directory where the compiler binary is (not under \linux).
\item In the parent directory of the current directory.
\item In the subdirectory \file{units} of the current directory
\item In the standard unit directory.
\end{enumerate}

Also, unit names that are longer than 8 characters will first be looked for
with their full length. If the unit is not found with this name, the name
will be truncated to 8 characters, and the compiler will look again in the
same directories, but with the truncated name.


If the compiler finds the unit it needs, it will look for the source file of
this unit in the same directory where it found the unit.
If it finds the source of the unit, then it will compare the file times.
If the source file was modified more recent than the unit file, the
compiler will attempt to recompile the unit with this source file.

If the compiler doesn't find a compiled version of the unit, or when the
\var{-B} option is specified, then the compiler will look in the same
manner for the unit source file, and attempt to recompile it.

It is recommended to set the unit search path in the configuration file
\file{fpc.cfg}. If you do this, you don't need to specify the unit search
path on the command line every time you want to compile something.

% Include files.
\subsection{Include files}
If you include a file in your source with the \var{\{\$I filename\}}
directive, the compiler will look for it in the following places:

\begin{enumerate}
\item It will look in the path specified in the include file name.
\item It will look in the directory where the current source file is.
\item it will look in all directories specified in the include file search
path.
\end{enumerate}
You can add files to the include file search path with the \seeo{I} or
\seeo{Fi} options.

As an example, consider the following include statement in a file
\file{units/foo.pp}:
\begin{verbatim}

{$i ../bar.inc}

\end{verbatim}
Then the following command :
\begin{verbatim}
fpc -Iincfiles units/foo.pp
\end{verbatim}
will cause the compiler to look in the following directories for
\file{bar.inc}:
\begin{enumerate}
\item The parent directory of the current directory.
\item The \file{units} subdirectory of the current directory.
\item The \file{incfiles} subdirectory of the current directory.
\end{enumerate}

% Object files.
\subsection{Object files}
When you link to object files (using the \var{\{\$L file.o\}} directive,
the compiler will look for this file in the same way as it looks for include
files:

\begin{enumerate}
\item It will look in the path specified in the object file name.
\item It will look in the directory where the current source file is.
\item It will look in all directories specified in the object file search path.
\end{enumerate}
You can add files to the object file search path with the \seeo{Fo} option.

% Configuration file
\subsection{Configuration file}
\label{searchconfig}

Not all options must be given on the compiler command line. The compiler
can use a configuration file which can contain the same options as on the
command line. Unless you specify the \seeo{n} option, the compiler will look
for a configuration file \file{fpc.cfg} in the following places:

\begin{itemize}
\item Under \unix (such as \linux)
\begin{enumerate}
\item The current directory.
\item Your home directory, it looks for \file{.fpc.cfg}.
\item The directory specified in the environment 
variable \var{PPC\_CONFIG\_PATH}, and if it is not set, it will look in the 
\file{etc} directory above the compiler directory. (For instance, if the
compiler is in \file{/usr/local/bin}, it will look in \file{/usr/local/etc})
\item The directory \file{/etc}.
\end{enumerate}
\item Under all other OSes:
\begin{enumerate}
\item The current directory.
\item If it is set, the directory specified in the environment variable
\var{PPC\_CONFIG\_PATH}.
\item The directory where the compiler is.
\end{enumerate}
\end{itemize}

Versions prior to version 1.0.6 of the compiler used a configuration
file \file{ppc386.cfg}. This file is still searched, but its usage 
is considered deprecated. For compatibility, \file{fpc.cfg} will
be searched first, and if not found, the file \file{ppc386.cfg}
will be searched and used.


\subsection{About long filenames}
\fpc can handle long filenames on all platforms, except DOS.
On Windows, it will use support for long filenames if it is available
(which is not always the case on older versions of Windows).

If no support for long filenames is present, it will truncate unit names
to 8 characters.

It is not recommended to put units in directories that contain spaces in
their names, since the external GNU linker doesn't understand such filenames.

%%%%%%%%%%%%%%%%%%%%%%%%%%%%%%%%%%%%%%%%%%%%%%%%%%%%%%%%%%%%%%%%%%%%%%%
% Compiling a program
\section{Compiling a program}
Compiling a program is very simple. Assuming that you have a program source
in the file \file{prog.pp}, you can compile this with the following command:
\begin{verbatim}
  fpc [options] prog.pp
\end{verbatim}
The square brackets \var{[\ ]} indicate that what is between them is optional.

If your program file has the \file{.pp} or \file{.pas} extension,
you can omit this on the command line, e.g. in the previous example you
could have typed:
\begin{verbatim}
  fpc [options] prog
\end{verbatim}

If all went well, the compiler will produce an executable file. You can execute 
it straight away; you don't need to do anything else. 

You will notice that there is also another file in your directory, with
extension \file{.o}. This contains the object file for your program.
If you compiled a program, you can delete the object file (\file{.o}),
but don't delete it if you compiled a unit. This is because
the unit object file contains the code of the unit, and will be
linked in any program that uses it.



%%%%%%%%%%%%%%%%%%%%%%%%%%%%%%%%%%%%%%%%%%%%%%%%%%%%%%%%%%%%%%%%%%%%%%%
% Compiling a unit
\section{Compiling a unit}

Compiling a unit is not essentially different from compiling a program.
The difference is mainly that the linker isn't called in this case.

To compile a unit in the file \file{foo.pp}, just type :
\begin{verbatim}
  fpc  foo
\end{verbatim}
Recall the remark about file extensions in the previous section.

When all went well, you will be left with 2 (two) unit files:
\begin{enumerate}
\item \file{foo.ppu} - this is the file describing the unit you just
compiled.
\item \file{foo.o} - this file contains the actual code of the unit.
This file will eventually end up in the executables.
\end{enumerate}
Both files are needed if you plan to use the unit for some programs.
So don't delete them. If you want to distribute the unit, you must
provide both the \file{.ppu} and \file{.o} file. One is useless without the
other.

%%%%%%%%%%%%%%%%%%%%%%%%%%%%%%%%%%%%%%%%%%%%%%%%%%%%%%%%%%%%%%%%%%%%%%%
% Units libraries and smartlinking
\section{Units, libraries and smartlinking}
The \fpc compiler supports smartlinking and the creation of libraries.
However, the default behaviour is to compile each unit into one big object
file, which will be linked as a whole into your program.
Shared libraries can be created on any platform except MS-DOS.

It is also possible to take existing units and put them
together in 1 static or shared library (using the \file{ppumove} tool,
\sees{ppumove}).

%%%%%%%%%%%%%%%%%%%%%%%%%%%%%%%%%%%%%%%%%%%%%%%%%%%%%%%%%%%%%%%%%%%%%%%
% Reducing the size of your program
\section{Reducing the size of your program}

When you created your program, it is possible to reduce the size of the
resulting executable. This is possible, because the compiler leaves a 
lot of information in the program which, strictly speaking, isn't required 
for the execution of the program. 

The surplus of information can be removed with a small program
called \file{strip}.The usage is simple. Just type
\begin{verbatim}
strip prog
\end{verbatim}
On the command line, and the \file{strip} program will remove all unnecessary
information from your program. This can lead to size reductions of up to
30 \%.

%\begin{remark}
%In the \win version, \file{strip} is called \file{stripw}.
%\end{remark}
You can use the \var{-Xs} switch to let the compiler do this stripping
automatically at program compile time. (The switch has no effect when
compiling units.)

Another technique to reduce the size of a program is to use smartlinking.
Normally, units (including the system unit) are linked in as a whole.
It is however possible to compile units such that they can be smartlinked.
This means that only the functions and procedures that are actually used
are linked in your program, leaving out any unnecessary code. The compiler 
will turn on smartlinking with the \seeo{XX} switch. This technique is 
described in full in the programmers guide. 

%%%%%%%%%%%%%%%%%%%%%%%%%%%%%%%%%%%%%%%%%%%%%%%%%%%%%%%%%%%%%%%%%%%%%
% Problems
%%%%%%%%%%%%%%%%%%%%%%%%%%%%%%%%%%%%%%%%%%%%%%%%%%%%%%%%%%%%%%%%%%%%%
\chapter{Compiling problems}

%%%%%%%%%%%%%%%%%%%%%%%%%%%%%%%%%%%%%%%%%%%%%%%%%%%%%%%%%%%%%%%%%%%%%%%
% General problems
\section{General problems}
\begin{itemize}
\item \textbf{IO-error -2 at ...} : Under \linux you can get this message at
compiler startup. It means typically that the compiler doesn't find the
error definitions file. You can correct this mistake with the \seeo{Fr}
option under \linux.
\item \textbf {Error : File not found : xxx} or \textbf{Error: couldn't compile
unit xxx}: This typically happens when
your unit path isn't set correctly. Remember that the compiler looks for
units only in the current directory, and in the directory where the compiler
itself is. If you want it to look somewhere else too, you must explicitly
tell it to do so using the \seeo{Fu} option. Or you must set up a 
configuration file.
\end{itemize}

%%%%%%%%%%%%%%%%%%%%%%%%%%%%%%%%%%%%%%%%%%%%%%%%%%%%%%%%%%%%%%%%%%%%%%%
% Problems you may encounter under DOS
\section{Problems you may encounter under DOS}
\begin{itemize}
\item \textbf{No space in environment}.\\
An error message like this can occur if you call
\verb|SET_PP.BAT| in \file{AUTOEXEC.BAT}.\\
To solve this problem, you must extend your environment memory.
To do this, search a line in \file{CONFIG.SYS} like
\begin{verbatim}
SHELL=C:\DOS\COMMAND.COM
\end{verbatim}
and change it to the following:
\begin{verbatim}
SHELL=C:\DOS\COMMAND.COM /E:1024
\end{verbatim}
You may just need to specify a higher value, if this parameter is already set.
\item \textbf{ Coprocessor missing}\\
If the compiler writes
a message that there is no coprocessor, install
the coprocessor emulation.
\item \textbf{Not enough DPMI memory}\\
If you want to use the compiler with \var{DPMI} you must have at least
7-8 MB free \var{DPMI} memory, but 16 Mb is a more realistic amount.
\end{itemize}



%%%%%%%%%%%%%%%%%%%%%%%%%%%%%%%%%%%%%%%%%%%%%%%%%%%%%%%%%%%%%%%%%%%%%
% Configuration.
%%%%%%%%%%%%%%%%%%%%%%%%%%%%%%%%%%%%%%%%%%%%%%%%%%%%%%%%%%%%%%%%%%%%%
\chapter{Compiler configuration}
\label{ch:CompilerConfiguration}

The output of the compiler can be controlled in many ways. This can be done
essentially in two distinct ways:
\begin{itemize}
\item Using command line options.
\item Using the configuration file: \file{fpc.cfg}.
\end{itemize}
The compiler first reads the configuration file. Only then are the command line
options checked. This creates the possibility to set some basic options
in the configuration file, and at the same time you can still set some
specific options when compiling some unit or program. First we list the
command line options, and then we explain how to specify the command
line options in the configuration file. When reading this, keep in mind
that the options are case sensitive. 


%%%%%%%%%%%%%%%%%%%%%%%%%%%%%%%%%%%%%%%%%%%%%%%%%%%%%%%%%%%%%%%%%%%%%%%
% Using the command line options
\section{Using the command line options}

The available options for the current version of the compiler are listed by
category. Also, see \seec{commandlineoptions} for a listing as generated by 
the current compiler.

%
% General options
%

\subsection{General options}
\begin{description}
\item[-h] Print a list of all options and exit.
\olabel{h}
\item[-?] Same as \var{-h}, waiting after each screenfull for the enter key.
\item[-i] Print copyright information. You can supply a qualifier, 
\olabel{i} as \var{-ixxx} where xxx can be one of the following:
\begin{description}
\item[D] : Returns the compiler date.
\item[V] : Returns the compiler version.
\item[SO] : Returns the compiler OS.
\item[SP] : Returns the compiler processor.
\item[TO] : Returns the target OS.
\item[TP] : Returns the target processor.
\end{description}
\item[-l]  Print the Free Pascal logo and version number.
\olabel{l}
\item [-n] Ignore the default configuration file.
You can still pass a configuration file with the \var{@} option.
\olabel{n}
\end{description}

%
% Options for getting feedback
%
\subsection{Options for getting feedback}
\begin{description}
\item[-vxxx] Be verbose. \var{xxx} is a combination of the following :
\olabel{v}
\begin{itemize}
\item \var{e} : Show errors. This option is on by default.
\item \var{i} : Display some general information.
\item \var{w} : Issue warnings.
\item \var{n} : Issue notes.
\item \var{h} : Issue hints.
\item \var{i} : Issue informational messages.
\item \var{l} : Report number of lines processed (every 100 lines).
\item \var{u} : Show information on units being loaded.
\item \var{t} : Show names of files being opened.
\item \var{p} : Show names of procedures and functions being processed.
\item \var{c} : Notify on each conditional being processed.
\item \var{m} : Show names of macros being defined.
\item \var{d} : Show additional debugging information.
\item \var{0} : No messages. This is useful for overriding the default
          setting in the configuration file.
\item \var{b} : Show all procedure declarations if an overloaded function
error occurs.
\item \var{x} : Show information about the executable (Win32 platform only).
\item \var{r} : Format errors in RHIDE/GCC compatibility mode.
\item \var{a} : Show all possible information. (this is the same as specifying all options)
\item \var{b} : Tells the compiler to write filenames using the full path.
\item \var{v} : Write copious debugging information to file
\file{fpcdebug.txt}..
Mainly for the compiler developers.
\item \var{p} Write parse tree to file tree.log. (Intended for compiler  developers.)
\end{itemize}
\end{description}

%
% Options concerning files and directories
%
\subsection{Options concerning files and directories}
\begin{description}
\item [-exxx] Specify \file{xxx} as the directory containing the
executables for the programs \file{as} (the assembler) and \var{ld} (the linker).
\olabel{e}
\item[-FaXYZ] load units \var{XYZ} after the system unit, but before any other 
unit is loaded. \var{XYZ} is a comma-separated list of unit names. This can only be used
for programs, and has the same effect as if \var{XYZ} were inserted as the
first item in the program's \var{uses} clause.
\item[-FcXXX] Set the input codepage to \var{XXX}. Experimental.
\item[-FCxxx] Set the RC compiler (resource compiler) binary name to \file{xxx}.
\item [-FD] Same as \var{-e}.
\item [-Fexxx] Write errors, etc. to the file named \file{xxx}.
\olabel{Fe}
\item [-FExxx] Write the executable and units to directory \file{xxx} 
instead of the current directory. If this option
is followed by a  \var{-o} option \seeo{o}, and this option contains a path 
component, then the \var{-o} path will override the \var{-FE} setting.
\olabel{FE}
\item [-Ffxxx] Add \file{xxx} to the framework path (only for Darwin).
\item [-Fixxx] Add \file{xxx} to the include file search path.
\olabel{Fi}
\item [-Flxxx] Add \file{xxx} to the library search path. (This is also 
passed to the linker.)
\olabel{Fl}
\item[-FLxxx] (\linux only) Use \file{xxx} as the dynamic linker. The default is \file{/lib/ld-linux.so.2}, or
\file{/lib/ld-linux.so.1}, depending on which one is found first.
\olabel{FL}
\item[-Fmxxx] Load the unicode conversion table from file \file{x.txt} in
the directory where the compiler is located. Only used when \var{-Fc} is
also in effect.
\item[-Foxxx] Add \file{xxx} to the object file search path.
This path is used when looking for files that need to be linked in.
\olabel{Fo}
\item [-Frxxx] Specify \file{xxx} as the file which contain the compiler
messages. This will override the compiler's built-in default messages, which
are in english.
\olabel{Fr}
\item[-FRxxx] set the resource (.res) linker to \file{xxx}.
\item [-Fuxxx] Add \file{xxx} to the unit search path.
Units are first searched in the current directory.
If they are not found there then the compiler searches them in the unit path.
You must {\em always} supply the path to the system unit. The \file{xxx}
path can contain a single wildcard (*) which will be expanded to all
possible directory names found at that location. Note that the location
itself is not included in the list. See \sees{unitsearching} for more
information about this option.
\olabel{Fu}
\item [-FUxxx] Write units to directory \var{xxx} instead of the current 
directory. It overrides the \var{-FE} option.
\item [-Ixxx] \olabel{I} Add \file{xxx} to the include file search path.
This option has the same effect as \var{-Fi}.
%\item [-P] uses pipes instead of files when assembling. This may speed up
%the compiler on \ostwo and \linux. Only with assemblers (such as \gnu
%\file{as}) that support piping...
\end{description}

% Options controlling the kind of output.
\subsection{Options controlling the kind of output.}
\label{se:codegen}
For more information on these options, see \progref.
\begin{description}
\item [-a] \olabel{a} Do not delete the assembler files (not
applicable when using the internal assembler). This also applies
to the (possibly) generated batch script.
\item [-al] \olabel{al} Include the source code lines in the assembler
file as comments.
\item[-an] \olabel{an} Write node information in the
assember file (nodes are the way the compiler represents statements or parts
thereof internally). This is primarily intended for debugging
the code generated by the compiler.
\item[-ap] \olabel{ap} Use pipes instead of creating temporary assembler
files.  This may speed up the compiler on \ostwo and \linux. 
Only with assemblers (such as \gnu %\file{as}) that support piping, and not
if the internal assembler is used.
\item[-ar] \olabel{ar} List register allocation and
release info in the assembler file. This is primarily intended for debugging
the code generated by the compiler.
\item[-at] \olabel{at} List information about
temporary allocations and deallocations in the assembler file.
\item [-Axxx] \olabel{A} specify what kind of assembler should be generated. 
Here \var{xxx} is one of the following :
\begin{description}
\item[default] Use the built-in default.
\item[as] Assemble using \gnu as.
\item[nasmcoff] Coff (Go32v2) file using Nasm.
\item[nasmelf] Elf32 (\linux) file using Nasm.
\item[nasmwin32] \windows 32-bit file using Nasm.
\item[nasmwdosx] \windows 32-bit/DOSX file using Nasm.
\item[nasmobj] Object file using Nasm.
\item[masm] Object file using Masm (Microsoft).
\item[tasm] Object file using Tasm (Borland).
\item[elf] Elf32 (\linux) using internal writer.
\item[coff] Coff object file (Go32v2) using the internal binary object writer.
\item[pecoff] PECoff object file (Win32) using the internal binary object writer.
\end{description}
\item[-B] \olabel{B} Re-compile all used units, even
if the unit sources didn't change since the last compilation.
\item[-b] \olabel{b} Generate browser info. This information can
be used by an Integrated Development Environment (IDE) to provide information
on classes, objects, procedures, types  and variables in a unit.
\item[-bl] \olabel{bl} The same as \var{-b} but also generates
information about local variables, types and procedures.
\item[-Caxxx] Set the ABI (Application Binary Interface) to \file{xxx}. 
The \var{-i} option gives the possible values  for \file{xxx}.
\item[-Cb] Generate big-endian code.
\item[-Cc] Set the default calling convention used by the compiler.
\item [-CD] Create a dynamic library. This is used to transform units into
dynamically linkable libraries on \linux.
\item[-Ce] Emulate floating point operations.
\item[-Cfxxx] Set the used floating point processor to \file{xxx}.
\item[-CFNN] Set the minimal floating point precision to \var{NN}. Possible
values are 32 and 64.
\item[-Cg] Enable generation of PIC code. This should only be necessary when
generating libraries on \linux or other Unices.
\item [-Chxxx] \olabel {Ch} Reserves \var{xxx} bytes heap. \var{xxx} should
be between 1024 and 67107840.
\item [-Ci] \olabel{Ci} Generate Input/Output checking code. In case some
input/output code of your program returns an error status, the program will
exit with a run-time error. Which error is generated depends on the I/O error.
\item [-Cn] \olabel{Cn} Omit the linking stage.
\item [-Co] \olabel{Co} Generate Integer overflow checking code. In case of
integer errors, a run-time error will be generated by your program.
\item [-CO] \olabel{CO} Check for possible overflow of integer operations.
\item [-CpXXX] Set the processor type to \var{XXX}.
\item [-CPX=N] Set the packing for \file{X} to N. X can be \var{PACKSET},
\var{PACKENUM} or \var{PACKRECORD}, and N can be a value of 1,2,4,8 or one
of the keywords \var{DEFAULT} or \var{NORMAL}.
\item [-Cr] \olabel{Cr} Generate Range checking code. If your program
accesses an array element with an invalid index, or if it increases an
enumerated type beyond its scope, a run-time error will be generated.
\item [-CR] \olabel{CR} Generate checks when calling methods to verify
if the virtual method table for that object is valid.
\item [-Csxxx] \olabel{Cs} Set stack size to \var{xxx}.
\item [-Ct] \olabel{Ct} Generate stack checking code. If your program
performs a faulty stack operation, a run-rime error will be generated.
\item [-CX] \olabel{Cx} Create a smartlinked unit when writing a unit.
Smartlinking will only link in the code parts that are actually needed by
the program. All unused code is left out. This can lead to substantially
smaller binaries.
\item [-dxxx] \olabel{d} Define the symbol name \var{xxx}. This can be used
to conditionally compile parts of your code.
\item [-D] Generate a DEF file (for OS/2).
\item [-Dd] Set the description of the executable/library (\windows).
\item [-Dv] Set the version of the executable/library (\windows).
\item [-E] \olabel{E} Same as \var{-Cn}.
\item [-g] \olabel{g} Generate debugging information for debugging with
\file{gdb}.
\item [-gc] Generate checks for pointers. This must be used with the
\var{-gh} command line option. When this options is enabled, it will verify 
that all pointer accesses are within the heap.
%\item [-gd] \olabel{gd} Generate debugging info for \file{dbx}.
\item [-gg] Same as \var{-g}.
\item [-gh] Use the heaptrc unit (see \unitsref). (Produces a report
about heap usage after the program exits)
\item [-gl] Use the lineinfo unit (see \unitsref). (Produces file
name/line number information if the program exits due to an error.)
\item[-goXXX] set debug information options. One of the options is
\var{dwarfsets}: It enables dwarf set debug information (this does not work
with \var{gdb} versions prior to 6.5.
\item [-gp] Preserve case in stabs symbol names. Default is to uppercase all
names.
\item [-gs] Write stabs debug information.
\item [-gt] Trash local variables. This writes a random value to local
variables at procedure start. This can be used to detect uninitialized
variables.
\item [-gv] Emit info for valgrind.
\item [-gw] Emit dwarf debugging info (version 2).
\item [-gw2] Emit dwarf debugging info (version 2).
\item [-gw3] Emit dwarf debugging info (version 3).
\item[-kxxx] Pass \var{xxx} to the linker. 
\item[-Oxxx] \olabel{O} Optimize the compiler's output; \var{xxx} can have one
of the following values :
\begin{description}
\item[aPARAM=VALUE] Specify alignment of structures and code. \var{PARAM}
determines what should be aligned; \var{VALUE} specifies the alignment
boundary. See the Programmer's Guide for a description of the possible
values.
\item[g] Optimize for size, try to generate smaller code.
\item[G] Optimize for time, try to generate faster code (default).
\item[r] Keep certain variables in registers (experimental, use with
caution).
\item[u] Uncertain optimizations
\item[1] Level 1 optimizations (quick optimizations).
\item[2] Level 2 optimizations (\var{-O1} plus some slower optimizations).
\item[3] Level 3 optimizations (\var{-O2} plus \var{-Ou}).
\item[Pn] (Intel only) Specify processor: \var{n} can be one of
\begin{description}
\item[1] Optimize for 386/486
\item[2] Optimize for Pentium/PentiumMMX (tm)
\item[3] Optimizations for PentiumPro/PII/Cyrix 6x86/K6 (tm)
\end{description}
\end{description}
The exact effect of these optimizations can be found in the \progref.
\item [-oxxx] \olabel{o} Use \var{xxx} as the name of the output
file (executable). For use only with programs. The output filename can contain a
path, and if it does, it will override any previous \var{-FE} setting. If
the output filename does not contain a path, the \var{-FE} setting is
observed.
\item [-pg] \olabel{gp} Generate profiler code for \file{gprof}. This will
define the symbol \var{FPC\_PROFILE}, which can be used in conditional
defines.
\item [-s] \olabel{s} Do not call the assembler and linker.
Instead, the compiler writes a script, \file{PPAS.BAT} under \dos, or
\file{ppas.sh} under \linux, which can then be executed to produce an
executable. This can be used to speed up the compiling process or to debug
the compiler's output. This option can take an extra parameter, mainly
used for cross-compilation. It can have one of the following values:
\begin{description}
\item[h] Generate script to link on host. The generated script can be run on
the compilation platform (host platform).
\item[t] Generate script to link on target platform. The generated script
can be run on the target platform. (where the binary is intended to be run)
\item[r] Skip register allocation phase (optimizations will be disabled).
\end{description}
\item[-Txxx] \olabel{T} Specify the target operating system. \var{xxx} can be one of
the following:
\begin{itemize}
\item \textbf{emx} : OS/2 via EMX (and DOS via EMX extender).
\item \textbf{freebsd} : FreeBSD.
\item \textbf{go32v2} : \dos and version 2 of the DJ DELORIE extender.
\item \textbf{linux} : \linux.
\item \textbf{netbsd} : NetBSD.
\item \textbf{netware} : Novell Netware Module (clib).
\item \textbf{netwlibc} : Novell Netware Module (libc).
\item \textbf{openbsd} : OpenBSD.
\item \textbf{os2} : OS/2 (2.x) using the \var{EMX} extender.
\item \textbf{sunos} : SunOS/Solaris.
\item \textbf{watcom} : Watcom compatible DOS extender
\item \textbf{wdosx} : WDOSX extender.
\item \textbf{win32} : \windows 32 bit.
\item \textbf{wince} : \windows for handhelds (ARM processor).
\end{itemize}
\item [-uxxx] \olabel{u} Undefine the symbol \var{xxx}. This is the opposite
of the \var{-d} option.
\item [-Ur] \olabel{Ur} Generate release unit files. These files will not be
recompiled, even when the sources are available. This is useful when making
release distributions. This also overrides the \var{-B} option for release 
mode units.
\item[-W] Set some \windows or \ostwo attributes of the generated binary. It
can be one or more of the following
\begin{description}
\item[Bhhh] Set preferred base address to hhh (a hexadecimal address)
\item[C] Generate a console application (+) or a gui application (-).
\item[D] Force use of Def file for exports.
\item[F] Generate a FS application (+) or a console application (-).
\item[G] Generate a GUI application (+) or a console application (-).
\item[N] Do not generate a relocation section.
\item[R] Generate a relocation section.
\item[T] Generate a TOOL application (+) or a console application (-).
\end{description}
\item [-Xx] \olabel{X} Specify executable options. This tells the compiler what
kind of executable should be generated. The parameter \var{x}
can be one of the following:
\begin{itemize}
\item \textbf{c} : (\linux only) Link with the C library. You should only use this when
  you start to port \fpc to another operating system. \olabel{Xe}
\item \textbf{d} Do not use the standard library path. This is needed for
cross-compilation, to avoid linking with the host platform's libraries.
\item \textbf{D} : Link with dynamic libraries (defines the
\var{FPC\_LINK\_DYNAMIC} symbol) \olabel{XD}
\item \textbf{e} use external (GNU) linker.
\item \textbf{g} Create debug information in a separate file and add a debuglink section to executable.
\item \textbf{i} use internal linker.
\item \textbf{MXXX} : Set the name of the program entry routine.
The default is 'main'.
\item \textbf{m} : Generate linker map file.
\item \textbf{PXXX} : Prepend binutils names with  \var{XXX} for cross-compiling.
\item \textbf{rXXX} : Set library path to \var{XXX}.
\item \textbf{Rxxx} Prepend \file{xxx} to all linker search paths. (used for
cross compiling).
\item \textbf{s} : Strip the symbols from the executable. \olabel{Xs}
\item \textbf{S} : Link with static units (defines the \var{FPC\_LINK\_STATIC} symbol).
\olabel{XS}
\item \textbf{t} : Link static (passes the \var{-static} option to the linker). \olabel{Xt}
\item \textbf{X} : Link with smartlinked units (defines the
\var{FPC\_LINK\_SMART} symbol). \olabel{XX}
\end{itemize}
\end{description}

%
%

% Options concerning the sources (language options)

\subsection{Options concerning the sources (language options)}
\label{se:sourceoptions}
For more information on these options, see \progref
\begin{description}
\item[-Mmode] \olabel{M} Set language mode to \var{mode}, which can be one of the
following:
\begin{description}
\item[delphi] Try to be Delphi compatible. This is more strict
than the \var{objfpc} mode, since some \fpc extensions are switched off.
\item[fpc] Free Pascal dialect (default).
\item[gpc] Try to be gpc compatible.
\item[macpas] Try to be compatible with Macintosh Pascal dialects.
\item[objfpc] Switch on some Delphi extensions. This is different from 
Delphi mode, because some \fpc constructs are still available.
\item[tp] Try to be TP/BP 7.0 compatible. This means no function overloading
etc.
\end{description}
\item [-Rxxx] \olabel{R} Specify what kind of assembler you use in
your \var{asm} assembler code blocks. Here \var{xxx} is one of the following:
\begin{description}
\item [att\ ] \var{asm} blocks contain AT\&T-style  assembler.
This is the default style.
\item [intel] \var{asm} blocks contain Intel-style assembler.
\item [default] Use the default assembler for the specified target.
\item [direct] \var{asm} blocks should be copied as is in the assembler,
only replacing certain variables. 
\end{description}
\item [-S2] \olabel{Stwo} Switch on Delphi 2 extensions (\var{objfpc} mode). 
Deprecated, use \var{-Mobjfpc} instead.
\item [-Sa] \olabel{Sa} Include assert statements in compiled code. Omitting 
this option will cause assert statements to be ignored.
\item [-Sc] \olabel{Sc} Support C-style operators, i.e. \var{*=, +=, /= and
-=}.
\item [-Sd] \olabel{Sd} Try to be Delphi compatible. Deprecated, use
\var{-Mdelphi} instead. 
\item [-SeN] \olabel{Se} The compiler stops after the N-th error. Normally,
the compiler tries to continue compiling after an error, until 50 errors are
reached, or a fatal error is reached, and then it stops. With this switch,
the compiler will stop after the N-th error (if N is omitted, a default of 1
is assumed). Instead of a number, one of \var{n}, \var{h} or \var{w} can also be
specified. 
In that case the compiler will consider notes, hints or warnings as errors and 
stop when one is encountered.
\item [-Sg] \olabel{Sg} Support the \var{label} and \var{goto} commands. By
default these are not supported. You must also specify this option if you
use labels in assembler statements. (if you use the \var{AT\&T} style
assember)
\item [-Sh] Use ansistrings by default for strings. If this option is
specified, the compiler will interpret the \var{string} keyword as an
ansistring. Otherwise it is supposed to be a shortstring (TP style).
\item [-Si] \olabel{Si} Support \var{C++} style INLINE.
\item [-SIXXX] Set interfaces style to XXX.
\item [-Sk] Load the Kylix compatibility unit (\file{fpcylix}).
\item [-Sm] \olabel{Sm} Support C-style macros.
\item [-So] \olabel{So} Try to be Borland TP 7.0 compatible. Deprecated, use
\var{-Mtp} instead.
\item [-Sp] \olabel{Sp} Try to be \file{gpc} (\gnu Pascal compiler)
compatible. Deprecated, use \var{-Mgpc} instead.
\item [-Ss] \olabel{Ss} The name of constructors must be \var{init}, and the
name of destructors should be \var{done}.
\item [-St] \olabel{St} Allow the \var{static} keyword in objects.
\item [-Sx] Enable exception keywords (default in Delphi/Objfpc mode). This
will mark all exception related keywords as keywords, also in \tp or
\file{FPC} mode. This can be used to check for code which should be
mode-neutral as much as possible.
\item [-Un] \olabel{Un} Do not check the unit name. Normally, the unit name
is the same as the filename. This option allows them to be different.
\item [-Us] \olabel{Us} Compile a system unit. This option causes the
compiler to define only some very basic types.
\end{description}


%%%%%%%%%%%%%%%%%%%%%%%%%%%%%%%%%%%%%%%%%%%%%%%%%%%%%%%%%%%%%%%%%%%%%%%
% Using the configuration file
\section{Using the configuration file}
\label{se:configfile}
Using the configuration file \file{fpc.cfg} is an alternative to command
line options. When a configuration file is found, it is read, and the lines
in it are treated as if you typed them on the command line. They are treated
before the options that you type on the command line.

You can specify comments in the configuration file with the \var{\#} sign.
Everything from the \var{\#} on will be ignored.

The algorithm to determine which file is used as a configuration file
is decribed in \ref{searchconfig} on page \pageref{searchconfig}.

When the compiler has finished reading the configuration file, it continues
to treat the command line options. 

One of the command line options allows you to specify a second configuration
file: Specifying \file{@foo} on the command line will open file \file{foo},
and read further options from there. When the compiler has finished reading
this file, it continues to process the command line.

The configuration file allows a type of preprocessing. It understands the
following directives, which you should place starting on the first column of a line:
\begin{description}
\item [\#IFDEF]
\item [\#IFNDEF]
\item [\#ELSE]
\item [\#ENDIF]
\item [\#DEFINE]
\item [\#UNDEF]
\item [\#WRITE]
\item [\#INCLUDE]
\item [\#SECTION]
\end{description}
They work the same way as their \{\$...\}  counterparts in Pascal source code. 
All the default defines used to compile source code are also defined while 
processing the configuration file. For example, if the target compiler is an 
intel 80x86 compatible linux platform, both \var{cpu86} and \var{linux} will be 
defined while interpreting the configuration file. For the possible default 
defines when compiling, consult Appendix G of the \progref.


What follows is a description of the different directives.

\subsection{\#IFDEF}
Syntax:
\begin{verbatim}
#IFDEF name
\end{verbatim}
Lines following \var{\#IFDEF} are read only if the keyword \var{name}
following it is defined.

They are read until the keywords \var{\#ELSE} or \var{\#ENDIF} are
encountered, after which normal processing is resumed.

Example :
\begin{verbatim}
#IFDEF VER2_2_0
-Fu/usr/lib/fpc/2.2.0/linuxunits
#ENDIF
\end{verbatim}
In the above example, \file{/usr/lib/fpc/2.2.0/linuxunits} will be added to
the path if you're compiling with version 2.2.0 of the compiler.

\subsection{\#IFNDEF}
Syntax:
\begin{verbatim}
#IFNDEF name
\end{verbatim}
Lines following \var{\#IFNDEF} are read only if the keyword \var{name}
following it is not defined.

They are read until the keywords \var{\#ELSE} or \var{\#ENDIF} are
encountered, after which normal processing is resumed.

Example :
\begin{verbatim}
#IFNDEF VER2_2_0
-Fu/usr/lib/fpc/2.2.0/linuxunits
#ENDIF
\end{verbatim}
In the above example, \file{/usr/lib/fpc/2.2.0/linuxunits} will be added to
the path if you're NOT compiling with version 2.2.0 of the compiler.

\subsection{\#ELSE}
Syntax:
\begin{verbatim}
#ELSE
\end{verbatim}
\var{\#ELSE} can be specified after a \var{\#IFDEF} or \var{\#IFNDEF}
directive as an alternative.
Lines following \var{\#ELSE} are read only if the preceding \var{\#IFDEF}
or \var{\#IFNDEF} was not accepted.

They are skipped until the keyword \var{\#ENDIF} is
encountered, after which normal processing is resumed.

Example :
\begin{verbatim}
#IFDEF VER2_2_2
-Fu/usr/lib/fpc/2.2.2/linuxunits
#ELSE
-Fu/usr/lib/fpc/2.2.0/linuxunits
#ENDIF
\end{verbatim}
In the above example, \file{/usr/lib/fpc/2.2.2/linuxunits} will be added to
the path if you're compiling with version 2.2.2 of the compiler,
otherwise \file{/usr/lib/fpc/2.2.0/linuxunits} will be added to the path.

\subsection{\#ENDIF}
Syntax:
\begin{verbatim}
#ENDIF
\end{verbatim}
\var{\#ENDIF} marks the end of a block that started with \var{\#IF(N)DEF},
possibly with an \var{\#ELSE} between them.

\subsection{\#DEFINE}
Syntax:
\begin{verbatim}
#DEFINE name
\end{verbatim}
\var{\#DEFINE} defines a new keyword. This has the same effect as a
\var{-dname}  command line option.

\subsection{\#UNDEF}
Syntax:
\begin{verbatim}
#UNDEF name
\end{verbatim}
\var{\#UNDEF} un-defines a keyword if it existed.
This has the same effect as a \var{-uname}  command line option.

\subsection{\#WRITE}
Syntax:
\begin{verbatim}
#WRITE Message Text
\end{verbatim}
\var{\#WRITE} writes \var{Message Text} to the screen.
This can be useful to display warnings if certain options are set.

Example:
\begin{verbatim}
#IFDEF DEBUG
#WRITE Setting debugging ON...
-g
#ENDIF
\end{verbatim}
If \var{DEBUG} is defined, this will produce a line
\begin{verbatim}
Setting debugging ON...
\end{verbatim}
and will then switch on debugging information in the compiler.

\subsection{\#INCLUDE}
Syntax:
\begin{verbatim}
#INCLUDE filename
\end{verbatim}
\var{\#INCLUDE} instructs the compiler to read the contents of
\file{filename} before continuing to process options in the current file.

This can be useful if you want to have a particular configuration file
for a project (or, under \linux, in your home directory), but still want to
have the global options that are set in a global configuration file.

Example:
\begin{verbatim}
#IFDEF LINUX
#INCLUDE /etc/fpc.cfg
#ELSE
#IFDEF GO32V2
#INCLUDE c:\pp\bin\fpc.cfg
#ENDIF
#ENDIF
\end{verbatim}
This will include \file{/etc/fpc.cfg} if you're on a \linux machine,
and will include \verb+c:\pp\bin\fpc.cfg+
on a \dos machine.

\subsection{\#SECTION}
Syntax:
\begin{verbatim}
#SECTION name
\end{verbatim}
The \var{\#SECTION} directive acts as a \var{\#IFDEF} directive, only
it doesn't require an \var{\#ENDIF} directive. The special name \var{COMMON}
always exists, i.e. lines following \var{\#SECTION COMMON} are always read.

%%%%%%%%%%%%%%%%%%%%%%%%%%%%%%%%%%%%%%%%%%%%%%%%%%%%%%%%%%%%%%%%%%%%%
% Variable subsitution in paths
\section{Variable substitution in paths}
To avoid having to edit your configuration files too often,
the compiler allows you to specify the following variables in
the paths that you feed to the compiler:
\begin{description}
\item[FPCFULLVERSION] is replaced by the compiler's version string.
\item[FPCVERSION] is replaced by the compiler's version string.
\item[FPCDATE] is replaced by the compiler's date.
\item[FPCTARGET] is replaced by the compiler's target (combination of CPU-OS)
\item[FPCCPU] is replaced by the compiler's target CPU.
\item[FPCOS] is replaced by the compiler's target OS.
\end{description}
To have these variables subsituted, just insert them with a \var{\$}
prepended, as follows:
\begin{verbatim}
-Fu/usr/lib/fpc/$FPCVERSION/rtl/$FPCOS
\end{verbatim}
This is equivalent to
\begin{verbatim}
-Fu/usr/lib/fpc/2.2.2/rtl/linux
\end{verbatim}
if the compiler version is \var{2.2.2} and the target OS is \linux{}.

These replacements are valid on the command line and also in the
configuration file.

On the linux command line, you must be careful to escape the \var{\$} since
otherwise the shell will attempt to expand the variable for you, which may have
undesired effects.

%%%%%%%%%%%%%%%%%%%%%%%%%%%%%%%%%%%%%%%%%%%%%%%%%%%%%%%%%%%%%%%%%%%%%
% IDE.
%%%%%%%%%%%%%%%%%%%%%%%%%%%%%%%%%%%%%%%%%%%%%%%%%%%%%%%%%%%%%%%%%%%%%

%
%   $Id: ide.tex,v 1.10 2005/04/29 07:52:18 michael Exp $
%   This file is part of the FPC documentation.
%   Copyright (C) 2000 by Florian Klaempfl
%
%   The FPC documentation is free text; you can redistribute it and/or
%   modify it under the terms of the GNU Library General Public License as
%   published by the Free Software Foundation; either version 2 of the
%   License, or (at your option) any later version.
%
%   The FPC Documentation is distributed in the hope that it will be useful,
%   but WITHOUT ANY WARRANTY; without even the implied warranty of
%   MERCHANTABILITY or FITNESS FOR A PARTICULAR PURPOSE.  See the GNU
%   Library General Public License for more details.
%
%   You should have received a copy of the GNU Library General Public
%   License along with the FPC documentation; see the file COPYING.LIB.  If not,
%   write to the Free Software Foundation, Inc., 59 Temple Place - Suite 330,
%   Boston, MA 02111-1307, USA.
%
%%%%%%%%%%%%%%%%%%%%%%%%%%%%%%%%%%%%%%%%%%%%%%%%%%%%%%%%%%%%%%%%%%%%%
% The IDE
%%%%%%%%%%%%%%%%%%%%%%%%%%%%%%%%%%%%%%%%%%%%%%%%%%%%%%%%%%%%%%%%%%%%%
\chapter{The IDE}

The IDE (\textbf{I}ntegrated \textbf{D}evelopment \textbf{E}nvironment)
provides a comfortable user interface to the compiler. It contains an
editor with syntax highlighting, a debugger, symbol browser etc.
The IDE is a text-mode application which has the same look and feel
on all supported operating systems. It is modelled after the IDE of Turbo
Pascal, so many people should feel comfortable using it.

Currently, the IDE is available for \dos, \windows and \linux.

%%%%%%%%%%%%%%%%%%%%%%%%%%%%%%%%%%%%%%%%%%%%%%%%%%%%%%%%%%%%%%%%%%%%%%%
% First steps with the IDE
\section{First steps with the IDE}
%
% Starting the IDE
%
\subsection{Starting the IDE}
The IDE is started by entering the command:
\begin{verbatim}
fp
\end{verbatim}
at the command line. It can also be started from a graphical user
interface such as \windows.
\begin{remark}
Under \windows, it is possible to switch between windowed mode and
full screen mode by pressing \key{Alt-Enter}.
\end{remark}
%
% IDE command-line options.
%
\subsection{IDE command line options}
When starting the IDE, command line options can be passed:
\begin{verbatim}
fp [-option] [-option] ... <file name> ...
\end{verbatim}
\var{Option} is one of the following switches (the option letters
are case insensitive):
\begin{description}
\item [-N] (\dos only) Do not use long file names. \windows 95 and later
versions of \windows provide an interface to DOS applications to access
long file names.
The IDE uses this interface by default to access files. Under certain
circumstances, this can lead to problems. This switch tells the IDE not to
use the long filenames.
\item [-Cfilename] Read IDE options from \file{filename}. 
There should be no whitespace between the file name and the \var{-C}.
\item [-F] Use alternative graphic characters. This can be used to run the
IDE on \linux in an X-term or through a telnet session.
\item [-R] After starting the IDE, change automatically to the directory
which was active when the IDE exited the last time.
\item [-S] Disable the mouse. When this option is used, the use of a mouse is
disabled, even if a mouse is present.
\item[-Tttyname] (\linux/Unix only) Send program output to tty \var{ttyname}.
This avoids having to continually switch between program output and the IDE.
\end{description}
The files given at the command line are loaded into edit windows automatically.

\begin{remark}
Under DOS/Win32, the first character of a command line option can be a \var{/}
character instead of a \var{-} character. So \var{/S} is equivalent to \var{-S}.
\end{remark}

\subsection{The IDE screen}

After start up, the screen of the IDE can look like \seefig{idestart}.

\FPCpic{The IDE screen immediately after startup}{}{idestart}

At the top of the screen the \emph{menu bar} is visible, at the bottom
the \emph{status bar}. The empty space between them is called the
\emph{desktop}.

The status bar shows the keyboard shortcuts for frequently used
commands, and allows quick access to these commands by clicking
them with the mouse.
At the right edge of the status bar, the current amount of unused
memory is displayed. This is only an indication, since the IDE
tries to allocate more memory from the operating system if it
runs out of memory.

The menu provides access to all of the IDE's functionality, and
at the right edge of the menu, a clock is displayed.

The IDE can be exited by selecting \menu{File|Exit} in the menu
\footnote{\menu{File|Exit} means select the item 'Exit' in the menu 'File'.}
or by pressing \key{Alt-X}.

\begin{remark}
If a file \file{fp.ans} is found in the current directory,
then it is loaded and used to paint the background.
This file should contain ANSI drawing commands to draw on a screen.
\end{remark}

%%%%%%%%%%%%%%%%%%%%%%%%%%%%%%%%%%%%%%%%%%%%%%%%%%%%%%%%%%%%%%%%%%%%%%%
% Navigating in the IDE
\section{Navigating in the IDE}
The IDE can be navigated both with the keyboard and with a mouse, if the
system is equipped with a mouse.
%
% Using the keyboard
%
\subsection{Using the keyboard}
All functionality of the IDE is available through use of the keyboard.
\begin{itemize}
\item It is used for typing and navigating through the sources.
\item Editing commands such as copying and pasting text.
\item Moving and resizing windows.
\item It can be used to access the menu, by pressing \key{ALT} and the
appropriate highlighted menu letter, or by pressing \key{F10} and
navigating through the menu with the arrow keys.
More information on the menu can be found in \sees{idemenu}.
\item Many commands in the IDE are bound to shortcuts, i.e. typing a special
combination of keys will execute a command immediately.
\end{itemize}
\begin{remark}
\begin{itemize}
\item When working in a \linux X-Term or through a telnet session, the
key combination with \key{Alt} may not be available. To remedy this, the
\key{Ctrl-Z} combination can be typed first. This means that e.g. \key{Alt-X}
can be replaced by \key{Ctrl-Z X}.
\item Alternatively, you can try the key combination ESC-X for \key{Alt-X}
when working on \linux.
\item A complete reference of all keyboard shortcuts can be found in
\sees{keyshortcuts}.
\end{itemize}
\end{remark}
%
% Using the mouse
%
\subsection{Using the mouse}
\label{suse:mouseusage}
If the system is equipped with a mouse, it can be used to work with the
IDE. The left button is used to select menu items, press buttons, select
text blocks etc.

The right mouse button is used to access the local menu, if available.
Holding down the \key{Ctrl} or \key{Alt} key and clicking the right
button will execute user defined functions. See \sees{prefmouse}.

\begin{remark}
\begin{enumerate}
\item Occasionally, the manual uses the term "drag the mouse". This
means that the mouse is moved while the left mouse button is being
pressed.
\item
The action of mouse buttons may be reversed, i.e. the actions of the left
mouse button can be assigned to the right mouse button and vice versa
\footnote{See \sees{prefmouse} for more information on how to reverse the
actions of the mouse buttons.}. Throughout the manual, it is assumed
that the actions of the mouse buttons are not reversed.
\item
The mouse is not always available, even if a mouse is installed:
\begin{itemize}
\item The IDE is running under \linux through a telnet connection from
a \windows machine.
\item The IDE is running under \linux in an X-term under X-windows. In this
case it depends on the terminal program: under Konsole (the KDE terminal) it
works.
\end{itemize}
\item On Windows, the console has an option 'Quick edit', allowing text
to be copied to the clipboard by selecting text in the console window. If this
mode is enabled, the mouse will not work. The 'Quick edit' option should be
disabled in the console window's properties in order for the IDE to receive mouse
events.
\end{enumerate}
\end{remark}
%
% Navigating in dialogs
%
\subsection{Navigating in dialogs}
\label{se:navigatingdialogs}
Dialogs usually have a lot of elements in them such as buttons, edit fields,
memo fields, list boxes and so on. To activate one of these fields, choose
one of the following methods:
\begin{enumerate}
\item Click on the element with the mouse.
\item Press the \key{Tab} key till the focus reaches the element.
\item Press the highlighted letter in the element's label. If the focus
is currently on an element that allows editing, then \key{Alt} should be
pressed simultaneously with the highlighted letter. For a button, the action
associated with the button will then be executed.
\end{enumerate}
Inside edit fields, list boxes and memos, navigation is carried out with the
usual arrow key commands.

%%%%%%%%%%%%%%%%%%%%%%%%%%%%%%%%%%%%%%%%%%%%%%%%%%%%%%%%%%%%%%%%%%%%%%%
% Windows
\section{Windows}
\label{se:windows}
Nowadays, working with windowed applications should be no problem for
most \windows and \linux users. Nevertheless, the following section
describes how the windows work in order to derive the most benefit from the
Free Pascal IDE.
%
% Window basics
%
\subsection{Window basics}
\label{se:windowbasics}

A common IDE window is displayed in \seefig{idewin}.

\FPCpic{A common IDE window}{}{idewin}

The window is surrounded by a so-called \emph{frame}, the white double
line around the window.

At the top of the window 4 things are displayed:
\begin{itemize}
\item
At the upper left corner of the window, a \emph{close icon} is shown.
When clicked, the window will be closed. It can also be closed by
 pressing \key{Alt-F3} or by selecting the menu item \menu{Window|Close}.
All open windows can be closed by selecting the menu item
\menu{Window|Close all}.
\item In the middle, the title of the window is displayed.
\item Almost at the upper right corner, a number is visible.
This number identifies the editor window, and pressing \key{Alt-Number}
will jump to this window. Only the first 9 windows will get such a number.
\item At the upper right corner, a small green arrow is visible.
Clicking this arrow zooms the window so it covers the whole desktop.
Clicking this arrow on a zoomed window will restore the old size of the
window. Pressing the \key{F5} key has the same effect as clicking
that arrow. The same effect can be achieved with the menu item
\menu{Window|Zoom}.
Windows and dialogs which aren't resizeable can't be zoomed, either.
\end{itemize}

The right edge and bottom edges of a window contain scrollbars.
They can be used to scroll the window contents with the mouse.
The arrows at the ends of the scrollbars can be clicked to scroll the
contents line by line. Clicking on the dotted area between the arrows
and the cyan-coloured rectangle will scroll the window's content
page by page. By dragging the rectangle the content can be scrolled
continuously.

The star and the numbers in the lower left corner of the window
display information about the contents of the window. They
are explained in the section about the editor, see \sees{editingtext}.

%
% Sizing+moving windows
%
\subsection{Sizing and moving windows}
\label{se:windowsizingmoving}
A window can be moved and sized using the mouse and the keyboard.

To move a window:
\begin{itemize}
\item Using the mouse, click on the title bar and drag the window
with the mouse.
\item Using the keyboard, go into the size/move mode
by pressing \key{Ctrl-F5} or selecting the menu item
\menu{Window|Size/Move}. The window  frame will change to green to 
indicate that the IDE is in  size/move mode. Now the cursor keys 
can be used to move the  window. Press \key{Enter} to leave the 
size/move mode. In this case, the window will keep its size and 
position. Alternatively, pressing \key{Esc} will restore the old 
position.
\end{itemize}
To resize a window:
\begin{itemize}
\item Using the mouse, click on the lower right corner of the window
and drag it.
\item Using the keyboard, go into the size/move mode by pressing \key{Ctrl-F5} or selecting the menu item
\menu{Window|Size/Move}. The window frame will change to green to 
indicate that the IDE is in size/move mode. Now hold down the SHIFT 
key and press one of the cursor keys in order to resize the window. 
Press \key{Enter} to leave the size/move mode.
Pressing \key{Esc} will restore the old size.
\end{itemize}
Not all windows can be resized. This applies, for example, to
\emph{dialog windows} (\sees{dialogwindow}).

A window can also be hidden. To hide a window, the \key{Ctrl-F6} key
combination can be used, or the \menu{Window|Hide} menu may be selected.
To restore a Hidden window, it is necessary to select it from the window
list. More information about the window list can be found in the next
section.
%
% Multiple windows
%
\subsection{Working with multiple windows}
\label{se:multiplewindows}
When working with larger projects, it is likely that multiple windows
will appear on the desktop. However, only one of these windows will be
the active window; all other windows will be inactive.

An inactive window is identified by a grey frame. An inactive window can
be made active in one of several ways:
\begin{itemize}
\item Using the mouse, activate a window by clicking on it.
\item Using the keyboard, pressing \key{F6} will step through all open
windows. To activate the previously activated window, \key{Shift-F6} can
be used.
\item The menu item \menu{Window|Next} can be used to activate the next
window in the list of windows, while \var{Window|Previous} will select
the previous window.
\item If the window has a number in the upper right corner, it can be
activated by pressing \key{Alt-<number>}.
\item Pressing \key{Alt-0} will pop up a dialog with all
available windows which allows a quick activation of windows which
don't have a number.
\end{itemize}

The windows can be ordered and placed on the IDE desktop by zooming and
resizing them with the mouse or keyboard. This is a time-consuming task,
and particularly difficult with the keyboard. Instead, the menu items
\menu{Window|Tile} and \menu{Window|Cascade} can be used:
\begin{description}
\item[Tile] will divide the whole desktop space evenly between all resizable
windows.
\item[Cascade] puts all the windows in a cascaded arrangement.
\end{description}

In very rare cases the screen of the IDE may become mixed up. In this
case the whole IDE screen can be refreshed by selecting the menu item
\menu{Window|Refresh display}.
%
% Dialog windows
%
\subsection{Dialog windows}
\label{se:dialogwindow}
In many cases the IDE displays a dialog window to get user input.
The main difference to normal windows is that other windows cannot be
activated while a dialog is active. Also the menu is not accessible while in
a dialog. This behaviour is called \emph{modal}. To activate another window,
the modal window or dialog must be closed first.

A typical dialog window is shown in \seefig{idedlg}.

\FPCpic{A typical dialog window}{}{idedlg}

%%%%%%%%%%%%%%%%%%%%%%%%%%%%%%%%%%%%%%%%%%%%%%%%%%%%%%%%%%%%%%%%%%%%%%%
% The menu
\section{The Menu}
\label{se:idemenu}
The main menu (the gray bar at the top of the IDE) provides access to all the
functionality of the IDE. It also contains a clock, displaying the current
time. The menu is always available, except when a dialog is opened. If a
dialog is opened, it must be closed first in order to access the menu.

In certain windows, a local menu is also available. The local menu will
appear where the cursor is, and provides additional commands that are
context-sensitive.
%
% Accessing the menu
%
\subsection{Accessing the menu}
The menu can be accessed in a number of ways:
\begin{enumerate}
\item By using the mouse to select items. The mouse cursor should be located
over the desired menu item, and a left mouse click will then select it.
\item By pressing \key{F10}. This will switch the IDE focus to the menu.
The arrow keys can then be used to navigate in the menu. The
\key{Enter} key should be used to select items.
\item To access menu items directly, \key{Alt-<highlighted menu letter>}
can be used to select a menu item. Afterwards submenu entries can be selected
by pressing the highlighted letter, but without \key{Alt}.
E.g. \key{Alt-S G} is a fast way to display the \emph{goto line} dialog.
\end{enumerate}
Every menu item is explained by a short text in the status bar.

When a local menu is available, it can be accessed by pressing
the right mouse button or \key{Alt-F10}.

To exit any menu without taking any action, press the \key{ESC} key
twice.

In the following, all menu entries and their actions are described.
%
% The file menu
%
\subsection{The File menu}
\label{se:menufile}
The \menu{File} menu contains all menu items that allow the user 
to load and save files, as well as to exit the IDE.
\begin{description}
\item[New] Opens a new, empty editor window.
\item[New from template] Prompts for a template to be used, asks to fill in
any parameters, and then starts a new editor window with the template.
\item[Open] (\key{F3}) Presents a file selection dialog, and opens
the selected file in a new editor window.
\item[Print] print the contents of the current edit window.
\item[Print setup] set up the printer properties.
\item[Reload] Reload a file from disk.
\item[Save] (\key{F2}) Saves the contents of the current edit window
with the current filename. If the current edit window does not yet have
a filename, a dialog is presented to enter a new filename.
\item[Save as] Presents a dialog in which a filename can be entered. The
current window's contents are then saved to this new filename, and the
filename is stored for further save actions.
\item[Save all] Saves the contents of all edit windows.

\item[Change dir] Presents a dialog in which a directory can be selected.
The current working directory is then changed to the selected directory.
\item[Command shell] Executes a command shell. After the shell is exited, the
IDE resumes. Which command shell is executed depends on the system.
\item[Exit] (\key{ALT-X}) Exits the IDE. If any unsaved files are
in the editor, the IDE will ask if these files should be saved.
\end{description}
Under the \menu{Exit} menu appear some filenames of recently used files.
These entries can be used to quickly reload these files in the editor.

%
% The edit menu
%
\subsection{The Edit menu}
\label{se:menuedit}
The \menu{Edit} menu contains entries for accessing the clipboard, and
undoing or redoing editing actions. Most of these functions have shortcut
keys associated with them.
\begin{description}
\item[Undo] (\key{ALT-BKSP}) Reverses the effect of the last editing action.
The editing actions are stored in a buffer.
Selecting this mechanism will move backwards through this buffer, i.e.
multiple undo levels are possible. 
However, any selections that may have been made are not reproduced.
\item[Redo] Repeats the last action that was just undone with Undo.
Redo can redo multiple undone actions.
%\item[Dump undo]
%Shows the contents of the UNDO list in the messages window.
%\item[Undo all]
%Undo all actions in the undo buffer. If a new empty file was started, this
%action should clear the window contents again.
%\item[Redo all]
%Redo all editing actions that were undone.
\item[Cut] (\key{Shift-DEL}) Deletes the selected text from the window and
copies it to the clipboard. Any previous clipboard contents are lost. 
The new clipboard contents are available for pasting elsewhere.
\item[Copy] (\key{Ctrl-INS}) Copies the current selection to the clipboard.
Any previous clipboard contents are lost. The new clipboard
contents are available for pasting elsewhere.
\item[Paste] (\key{Shift-INS}) Inserts the current clipboard contents in the
text at the cursor position. The clipboard contents remain as they were.  
\item[Clear] (\key{Ctrl-DEL}) Clears (i.e. deletes) the current
selection.
\item[Select All] Selects all text in the current window. The selected
text can then be cut or copied to the clipboard.
\item[Unselect] undo the selection.
\item[Show clipboard] Opens a window in which the current clipboard contents
are shown.
\end{description}
When running an IDE under \windows, the \menu{Edit} menu has two
additional entries. The IDE maintains a separate clipboard which does
not share its contents with the \windows clipboard. To access the \windows
clipboard, the following two entries are also present:
\begin{description}
\item[Copy to Windows] Copy the selection to the \windows clipboard.
\item[Paste from Windows] Insert the contents of the \windows clipboard 
(if it contains text) in the edit window at the current cursor position.
\end{description}

%
% The Search menu
%
\subsection{The Search menu}
\label{se:menusearch}
The \menu{Search} menu provides access to the search and replace dialogs, as well as
access to the symbol browser of the IDE.
\begin{description}
\item[Find] (\key{Ctrl-Q F}) Presents the search dialog. A search text
can be entered, and when the dialog is closed, the entered text is searched
for in the active window. If the text is found, it will be selected.
\item[Replace] (\key{Ctrl-Q A}) Presents the search and replace dialog.
After the dialog is closed, the search text will be replaced by the replace
text in the active window.
\item[Search again] (\key{CTRL-L}) Repeats the last search or search and replace action,
 using  the same parameters.
\item[Go to line number] (\key{Alt-G}) Prompts for a line number, and
then jumps to this line number.
\end{description}
When the program and units are compiled with browse information, then
the following menu entries are also enabled:
\begin{description}
\item[Find procedure]
Not yet implemented.
\item[Objects]
Asks for the name of an object and opens a browse window for this object.
\item[Modules]
Asks for the name of a module and opens a browse window for this module.
\item[Globals]
Asks for the name of a global symbol and opens a browse window for this
global symbol.
\item[Symbol]
Opens a window with all known symbols, so a symbol can be selected. After
the symbol is selected, a browse window for that symbol is opened.
\end{description}
%
% The Run menu
%
\subsection{The Run menu}
\label{se:menurun}
The \menu{Run} menu contains all entries related to running a program,
\begin{description}
\item[Run] (\key{Ctrl-F9})
If the sources were modified, compiles the program. If the compile is
successful, the program is executed. If the primary file  was set, then
that is used to determine which program to execute. See \sees{menucompile}
for more information on how to set the primary file.
\item[Step over] (\key{F8})
Run the program until the next source line is reached. If any calls to
procedures are made, these will be executed completely as well.
\item[Trace into] (\key{F7})
Execute the current line. If the current line contains a call to another
procedure, the process will stop at the entry point of the called procedure.
\item[Goto cursor] (\key{F4})
Run the program until the execution point matches the line where the cursor
is.
\item[Until return] Runs the current procedure until it exits.
\item[Run directory] Set the working directory to change to when 
executing the program.
\item[Parameters]
Permits the entry of parameters that will be passed to the program when 
it is being executed.
\item[Program reset] (\key{Ctrl-F2}) if the program is being run or
debugged, the debug session is aborted, and the running program is killed.
\end{description}
%
% The compile menu
%
\subsection{The Compile menu}
\label{se:menucompile}
The \menu{Compile} menu contains all entries related to compiling a program or
unit.
\begin{description}
\item[Compile] (\key{Alt-F9}) Compiles the contents of the active window,
irrespective of the primary file setting.
\item[Make] (\key{F9}) Compiles the contents of the active window, and
any files that the unit or program depends on and that were modified since
the last compile.
If the primary file was set, the primary file is compiled instead.
\item[Build]
Compiles the contents of the active window, and any files that the unit or
program depends on, whether they were modified or not.
If the primary file was set, the primary file is compiled instead.
\item[Target] Sets the target operating system for which the program should be compiled.
\item[Primary file] Sets the primary file. If set, any run or compile command
will act on the primary file instead of on the active window. The primary
file need not be loaded in the IDE for this to have effect.
\item[Clear primary file]
Clears the primary file. After this command, any run or compile action will
act on the active window.
%\item[Information] Displays some information about the current program.
\item[Compiler messages] (\key{F12}) Displays the compiler messages
window. This window will display the messages generated by the compiler
during the most recent compile.
\end{description}
%
% The debug menu
%
\subsection{The Debug menu}
\label{se:menudebug}
The \menu{Debug} menu contains menu entries to aid in debugging a program, such as
setting breakpoints and watches.
\begin{description}
\item[Output] Show user program output in a window.
\item[User screen] (\key{Alt-F5})
Switches to the screen as it was last left by the running program.
\item[Add watch] (\key{Ctrl-F7}) Adds a watch. A watch is an expression
that can be evaluated by the IDE and will be shown in a special window.
Usually this is the content of some variable.
\item[Watches]
Shows the current list of watches in a separate window.
\item[Breakpoint] (\key{Ctrl-F8})
Sets a breakpoint at the current line. When debugging, program execution
will stop at this breakpoint.
\item[Breakpoint list]
Shows the current list of breakpoints in a separate window.
\item[Evaluate]
\item[Call stack] (\key{Ctrl-F3})
Shows the call stack. The call stack is the list of addresses (and
filenames and line numbers, if this information was compiled in) of
procedures that are currently being called by the running program.
\item[Disassemble] Shows the call stack.
\item[Registers]
Shows the current content of the CPU registers.
\item[Floating point unit] 
Shows the current content of the FPU registers.
\item[Vector unit]
Shows the current content of the MMX (or equivalent) registers.
\item[GDB window]
Shows the GDB debugger console. This can be used to interact with the debugger
directly; here arbitrary GDB commands can be typed and the result will be
shown in the window.
\end{description}
%
% The tools menu
%
\subsection{The Tools menu}
\label{se:menutools}
The \menu{Tools} menu defines some standard tools. If new tools are defined by the
user, they are appended to this menu as well.
\begin{description}
\item[Messages] (\key{F11}) Shows the messages window.
This window contains the output from one of the tools. For more information,
see \sees{toolsmessages}.
\item[Goto next] (\key{Alt-F8}) Goes to the next message.
\item[Goto previous] (\key{Alt-F7}) Goes to the previous message
\item[Grep] (\key{SHIFT-F2}) Prompts for a regular expression and options
to be given to grep, and then executes \file{grep} with the given expression and
options. For this to work, the \file{grep} program must be installed on the
system, and be in a directory that is in the \var{PATH}. For more
information, see \sees{grep}.
\item[Calculator]
Displays the calculator. For more information, see \sees{calculator}.
\item[Ascii table] Displays the \var{ASCII} table. For more information, see
\sees{asciitable}.
\end{description}
%
% The Options menu
%
\subsection{The Options menu}
\label{se:menuoptions}
The \menu{Options} menu is the entry point for all dialogs that are used to set
options for the compiler and the IDE, as well as the user preferences.
\begin{description}
\item[Mode] Presents a dialog to set the current mode of the compiler. The
current mode is shown at the right of the menu entry. For more information,
see \sees{compilermode}.
\item[Compiler] Presents a dialog that can be used to set common compiler
options. These options will be used when compiling a program or unit.
\item[Memory sizes]
Presents a dialog where the stack size and the heap size for the program can
be set. These options will be used when compiling a program.
\item[Linker]
Presents a dialog where some linker options can be set. These options will
be used when a program or library is compiled.
\item[Debugger]
Presents a dialog where the debugging options can be set. These options
are used when compiling units or programs. Note that the debugger will not
work unless debugging information is generated for the program.
\item[Directories]
Presents a dialog where the various directories needed by the compiler can
be set. These directories will be used when a program or unit is compiled.
\item[Browser]
Presents a dialog where the browser options can be set. The browser options
affect the behaviour of the symbol browser of the IDE.
\item[Tools]
Presents a dialog to configure the tools menu. For more information, see
\sees{addingtools}.
\item[Environment]
Presents a dialog to configure the behaviour of the IDE. A sub menu is
presented with the various aspects of the IDE:
\begin{description}
\item[Preferences]
General preferences, such as whether to save files automatically or not, 
and which files should be saved. The video mode can also be set here.
\item[Editor]
Controls various aspects of the edit windows.
\item[CodeComplete]
Used to set the words which can be automatically completed when typing in
the editor windows.
\item[Codetemplates]
Used to define code templates, which can be inserted in an edit window.
\item[Desktop]
Used to control the behaviour of the desktop, i.e. several features can be
switched on or off.
\item[Keyboard \& Mouse] Can be used to select the cut/copy/paste
convention, control the actions of the mouse, and to assign commands 
to various mouse actions.
%\item[Startup]
%Not yet implemented.
%\item[Colors]
%Here the various colors used in the IDE and the editor windows can be set.
\item[Learn keys] Let the IDE learn keystrokes to be assigned to various 
commands. This is useful mostly on \linux and Unix-like platforms, where the
actual keys sent to the IDE depend on the terminal emulation.
\end{description}
\item[Open]
Presents a dialog in which a file containing editor preferences can be selected.
After the dialog is closed, the preferences file will be read and the
preferences will be applied.
\item[Save]
Saves the current options in the default file.
\item[Save as]
Saves the current options in an alternate file. A file selection dialog box
will be presented in which the alternate settings file can be specified.
\end{description}
Please note that options are not saved automatically. They should be saved
explicitly with the \menu{Options|\-Save} command.
%
% The window menu
%
\subsection{The Window menu}
\label{se:menuwindow}
The \menu{Window} menu provides access to some window functions. 
More information on all these functions can be found in \sees{windows}
\begin{description}
\item[Tile]
Tiles all opened windows on the desktop.
\item[Cascade]
Cascades all opened windows on the desktop.
\item[Close all]
Closes all opened windows.
\item[Size/move] (\key{Ctrl-F5})
Puts the IDE in Size/move mode; after this command the active window can be
moved and resized using the arrow keys.
\item[Zoom] (\key{F5})
Zooms or unzooms the current window.
\item[Next] (\key{F6})
Activates the next window in the window list.
\item[Previous] (\key{SHIFT-F6})
Activates the previous window in the window list.
\item[Hide] (\key{Ctrl-F6})
Hides the active window.
\item[Close] (\key{ALT-F3})
Closes the active window.
\item[List] (\key{Alt-0})
Shows the list of opened windows. From there a
window can be activated, closed, shown and hidden.
\item[Refresh display]
Redraws the screen.
\end{description}
%
% The Help menu
%
\subsection{The Help menu}
\label{se:menuhelp}
The \menu{Help} menu provides entry points to all the help functionality of
the IDE, as well as the means to customize the help system.
\begin{description}
\item[Contents]
Shows the help table of contents
\item[Index] (SHIFT-F1)
Jumps to the help Index.
\item[Topic search]  (CTRL-F1)
Jumps to the topic associated with the currently highlighted text.
\item[Previous topic] (ALT-F1)
Jumps to the previously visited topic.
\item[Using help]
Displays help on using the help system.
\item[Files]
Allows the configuration of the help menu. With this menu item, help files can be added to the help
system.
\item[About]
Displays information about the IDE. See \sees{about} for more information.
\end{description}

%%%%%%%%%%%%%%%%%%%%%%%%%%%%%%%%%%%%%%%%%%%%%%%%%%%%%%%%%%%%%%%%%%%%%%%
% Editing text
\section{Editing text}
\label{se:editingtext}
In this section, the basics of editing (source) text are explained. The IDE
works like many other text editors in this respect, so mainly the
distinguishing points of the IDE will be explained.

\subsection{Insert modes}
Normally, the IDE is in insert mode. This means that any text that is typed
will be inserted before text that is present after the cursor.

In overwrite mode, any text that is typed will replace existing text.

When in insert mode, the cursor is a flat blinking line. If the IDE is in
overwrite mode, the cursor is a cube with the height of one line. Switching between
insert mode and overwrite mode happens with the \key{Insert} key or with the
\key{Ctrl-V} key.
%
% blocks
%
\subsection{Blocks}
\label{se:blocks}
The IDE handles selected text just as the \tp IDE handles it. This is
slightly different from the way e.g. \windows applications handle selected
text.

Text can be selected in 3 ways:
\begin{enumerate}
\item Using the mouse, dragging the mouse over existing text selects it.
\item Using the keyboard, press \key{Ctrl-K B} to mark the beginning of
the selected text, and \key{Ctrl-K K} to mark the end of the selected
text.
\item Using the keyboard, hold the \key{Shift} key depressed while
navigating with the cursor keys.
\end{enumerate}

There are also some special select commands:
\begin{enumerate}
\item The current line can be selected using \key{Ctrl-K L}.
\item The current word can be selected using \key{Ctrl-K T}.
\end{enumerate}

In the \fpc IDE, selected text is persistent. After selecting a range of
text, the cursor can be moved, and the selection will not be destroyed;
hence the term 'block' is more appropriate for the selection, and will be
used henceforth...

Several commands can be executed on a block:
\begin{itemize}
\item Move the block to the cursor location (\key{Ctrl-K V}).
\item Copy the block to the cursor location (\key{Ctrl-K C}).
\item Delete the block (\key{Ctrl-K Y}).
\item Write the block to a file (\key{Ctrl-K W}).
\item Read the contents of a file into a block (\key{Ctrl-K R}).
If there is already a block, this block is not replaced by this command.
The file is inserted at the current cursor position, and then the
inserted text is selected.
\item Indent a block (\key{Ctrl-K I}).
\item Undent a block (\key{Ctrl-K U}).
\item Print the block contents (\key{Ctrl-K P}).
\end{itemize}
When searching and replacing, the search can be restricted to the block
contents.

%
% Bookmarks
%
\subsection{Setting bookmarks}
\label{se:bookmarks}
The IDE provides a feature which allows the setting of a bookmark at the current
cursor position. Later, the cursor can be returned to this position
by pressing a keyboard shortcut.

Up to 9 bookmarks per source file can be set up; they are set by
\key{Ctrl-K <Number>} (where number is the number of the bookmark).
To go to a previously set bookmark, press \key{Ctrl-Q <Number>}.

\begin{remark}
Currently, the bookmarks are not saved when the IDE is exited. 
This may change in future implementations of the IDE.
\end{remark}

%
% Jumping to a source line
%
\subsection{Jumping to a source line}
It is possible to go directly to a specific source line. To do this, open
the {\em goto line} dialog via the \menu{Search|Goto line number} menu.

In the dialog that appears, the line number the IDE should jump to can be
entered. The goto line dialog is shown in \seefig{gotoline}.

\FPCpic{The goto line dialog.}{ide}{gotoline}

%
% Syntax highlighting and code completion
%
\subsection{Syntax highlighting}
\label{se:syntaxhighlighting}
The IDE is capable of syntax highlighting, i.e. the color of certain
Pascal elements can be set. As text is entered in an editor window,
the IDE will try to recognise the elements, and set the color of the
text accordingly.


The syntax highlighting can be customized in the colors preferences dialog,
using the menu option \menu{Options|\-Environment|\-Colors}. In the colors dialog, the
group "Syntax" must be selected. The item list will then display the
various syntactical elements that can be colored:
\begin{description}
\item[Whitespace] The empty text between words. Note that for whitespace,
only the background color will be used.
\item[Comments] All styles of comments in Free Pascal.
\item[Reserved words] All reserved words of Free Pascal. (See also \refref).
\item[Strings] Constant string expressions.
\item[Numbers] Numbers in decimal notation.
\item[Hex numbers] Numbers in hexadecimal notation.
\item[Assembler] Any assembler blocks.
\item[Symbols] Recognised symbols (variables, types).
\item[Directives] Compiler directives.
\item[Tabs] Tab characters in the source can be given a different color than
other whitespace.
\end{description}
The editor uses some default settings, but experimentation is the best way
to find a suitable color scheme. A good color scheme helps in detecting errors
in sources, since errors will result in wrong syntax highlighting.

% Code completion
\subsection{Code Completion}
\label{se:codecompletion}
Code completion means the editor will try to guess the text as it
is being typed. It does this by checking what text is typed, and as soon
as the typed text can be used to identify a keyword in a list of keywords,
the keyword will be presented in a small colored box under the typed text.
Pressing the \key{Enter} key will complete the word in the text.

There is no code completion yet for filling in function arguments, or choosing
object methods as in e.g. the Lazarus or \delphi IDEs.

Code completion can be customized in the Code completion dialog, reachable
through the menu option \menu{Options|\-Preferences|\-Codecomple}.
The list of keywords that can be completed can be maintained here.
The code completion dialog is shown in \seefig{codecomp}.

\FPCpic{The code completion dialog.}{ide}{codecomp}

The dialog shows in alphabetical order the currently defined keywords 
that are available for completion. The following buttons are available:
\begin{description}
\item[Ok] Saves all changes and closes the dialog.
\item[Edit] Pops up a dialog that allows the editing of the currently
highlighted keyword.
\item[New] Pops up a dialog that allows the entry of a new keyword which will be
added to the list.
\item[Delete] Deletes the currently highlighted keyword from the list.
\item[Cancel] Discards all changes and closes the dialog.
\end{description}
All keywords are saved and are available the next time the IDE is started.
Duplicate names are not allowed. If an attempt is made to add a duplicate
name to the list, an error will follow.

% Code templates
\subsection{Code Templates}
Code templates are a way to insert large pieces of code at once. Each
code templates is identified by a unique name. This name can be used to
insert the associated piece of code in the text.

For example, the name \var{ifthen} could be associated to the following
piece of code:
\begin{verbatim}
If | Then
  begin
  end
\end{verbatim}
A code template can be inserted by typing its name, and pressing \key{Ctrl-J}
when the cursor is positioned right after the template name.

If there is no template name before the cursor, a dialog will pop up to
allow selection of a template.

If a vertical bar (|) is present in the code template, the cursor is positioned
on it, and the vertical bar is deleted. In the above example, the cursor would be
positioned between the \var{if} and \var{then}, ready to type an expression.

Code templates can be added and edited in the code templates dialog, reachable via
the menu option \menu{Options|\-Environment|\-CodeTemplates}.
The code templates dialog is shown in \seefig{codetemp}.

\FPCpic{The code templates dialog.}{ide}{codetemp}

The top listbox in the code templates dialog shows the names of all
known templates. The bottom half of the dialog shows the text associated
with the currently highlighted code template.
The following buttons are available:
\begin{description}
\item[Ok] Saves all changes and closes the dialog.
\item[Edit] Pops up a dialog that allows the editing of the currently
highlighted code template. Both the name and text can be edited.
\item[New] Pops up a dialog that allows the entry of a new code template
which will be added to the list. A name must be entered for the new
template.
\item[Delete] Deletes the currently highlighted code template from the list.
\item[Cancel] Discards all changes and closes the dialog.
\end{description}
All templates are saved and are available the next time the IDE is started.
\begin{remark}
Duplicates are not allowed. If an attempt is made to add a duplicate name
to the list, an error will occur.
\end{remark}

%%%%%%%%%%%%%%%%%%%%%%%%%%%%%%%%%%%%%%%%%%%%%%%%%%%%%%%%%%%%%%%%%%%%%%%
% Searching in the text
\section{Searching and replacing}
\label{se:searching}
The IDE allows you to search for text in the active editor window.
To search for text, one of the following can be done:
\begin{enumerate}
\item Select \menu{Search|Find} in the menu.
\item Press \key{Ctrl-Q F}.
\end{enumerate}
After that, the dialog shown in \seefig{search} will pop up,
and the following options can be entered:

\FPCpic{The search dialog.}{ide}{search}

\begin{description}
\item[Text to find] The text to be searched for. If a block was active when
the dialog was started, the first line of this block is proposed.
\item[Case sensitive] When checked, the search is case sensitive.
\item[Whole words only] When checked, the search text must appear in the
text as a complete word.
\item[Direction] The direction in which the search must be conducted,
starting from the specified origin.
\item[Scope] Specifies if the search should be on the whole file, or just the selected
text.
\item[Origin] Specifies if the search should start from the cursor position or the start
of the scope.
\end{description}
After the dialog has closed, the search is performed using the given options.

A search can be repeated (using the same options) in one of 2 ways:
\begin{enumerate}
\item Select \menu{Search|Search again} from the menu.
\item Press \key{Ctrl-L}.
\end{enumerate}

It is also possible to replace occurrences of a text with another text.
This can be done in a similar manner to searching for a text:
\begin{enumerate}
\item Select \menu{Search|Replace} from the menu.
\item Press \key{Ctrl-Q A}.
\end{enumerate}
A dialog, similar to the search dialog will pop up, as shown in \seefig{replace}.

\FPCpic{The replace dialog.}{ide}{replace}

In this dialog, in addition to the things that can be entered in the
search dialog, the following things can be entered:
\begin{description}
\item [New text] Text that will replace the found text.
\item [Prompt on replace] Before a replacement is made, the IDE will ask for
confirmation.
\end{description}
If the dialog is closed with the 'OK' button, only the next occurrence of
the search text will be replaced.
If the dialog is closed with the 'Change All' button, all occurrences of
the search text will be replaced.

%%%%%%%%%%%%%%%%%%%%%%%%%%%%%%%%%%%%%%%%%%%%%%%%%%%%%%%%%%%%%%%%%%%%%%%
% The symbol browser
\section{The symbol browser}
\label{se:browser}
The symbol browser allows searching all occurrences of a symbol. A symbol
can be a variable, type, procedure or constant that occurs in the
program or unit sources.

To enable the symbol browser, the program or unit must be compiled with
browser information. This can be done by setting the browser information
options in the compiler options dialog.

The IDE allows to browse several types of symbols:
\begin{description}
\item[Procedures] Allows quick jumping to a procedure definition or
implementation.
\item[Objects] Quickly browse for an object.
\item[Modules] Browse a module.
\item[Globals] Browse any global symbol.
\item[Arbitrary symbol] Browse an arbitrary symbol.
\end{description}
In all cases, first a symbol to be browsed must be selected. After that,
a browse window appears. In the browse window, all locations where the
symbol was encountered are shown. Selecting a location and pressing the
space bar will cause the editor to jump to that location; the line
containing the symbol will be highlighted.

If the location is in a source file that is not yet displayed, a new
window will be opened with the source file loaded.

After the desired location has been reached, the browser window can be closed
with the usual commands.

The behaviour of the browser can be customized with the browser options
dialog, using the \menu{Options|Browser} menu.
The browser options dialog looks like \seefig{obrowser}.

\FPCpic{The browser options dialog.}{ide}{obrowser}

The following options can be set in the browser options dialog:
\begin{description}
\item[Symbols] Here the types of symbols displayed in the browser can be
selected:
\begin{description}
\item[Labels] Labels are shown.
\item[Constants] Constants are shown.
\item[Types] Types are shown.
\item[Variables] Variables are shown.
\item[Procedures] Procedures are shown.
\item[Inherited]
\end{description}
\item[Sub-browsing] Specifies what the browser should do when displaying the
members of a complex symbol such as a record or class:
\begin{description}
\item[New browser] The members are shown in a new browser window.
\item[Replace current] The contents of the current window are replaced with
the members of the selected complex symbol.
\end{description}
\item[Preferred pane] Specifies what pane is shown in the browser when it is
initially opened:
\begin{description}
\item[Scope]
\item[Reference]
\end{description}
\item[Display] Determines how the browser should display the symbols:
\begin{description}
\item[Qualified symbols]
\item[Sort always] Sorts the symbols in the browser window.
\end{description}
\end{description}

%%%%%%%%%%%%%%%%%%%%%%%%%%%%%%%%%%%%%%%%%%%%%%%%%%%%%%%%%%%%%%%%%%%%%%%
% Running programs
\section{Running programs}
\label{se:running}
A compiled program can be run straight from the IDE. This can be done
in one of several ways:
\begin{enumerate}
\item select the \menu{Run|Run} menu, or
\item press \key{Ctrl-F9}.
\end{enumerate}
If command line parameters should be passed to the program, then these
can be set through the \menu{Run|Parameters} menu.
The program parameters dialog looks like \seefig{params}.

\FPCpic{The program parameters dialog.}{ide}{params}

Once the program has started, it will continue to run, until
\begin{enumerate}
\item the program quits normally,
\item an error happens,
\item a breakpoint is encountered, or
\item the program is reset by the user.
\end{enumerate}
The last alternative is only possible if the program is compiled
with debug information.

Alternatively, it is possible to position the cursor somewhere in a
source file, and run the program till the execution reaches the
source line where the cursor is located. This can be done by
\begin{enumerate}
\item selecting \menu{Run|Goto Cursor} in the menu,
\item pressing \key{F4}.
\end{enumerate}
Again, this is only possible if the program was compiled with debug
information.

The program can also executed line by line. Pressing \key{F8} will
execute the next line of the program. If the program wasn't started
yet, it is started. Repeatedly pressing \key{F8} will execute the program
line by line, and the IDE will show the line to be executed
in an editor window. If somewhere in the code a call occurs to a subroutine,
then pressing \key{F8} will cause the whole routine to be executed before
control returns to the IDE. If the code of the subroutine should be stepped
through as well, then \key{F7} should be used instead. Using \key{F7} will
cause the IDE to execute line by line any subroutine that is encountered.

If a subroutine is being stepped through, then the \menu{Run|Until return} menu
will execute the program till the current subroutine ends.

If the program should be stopped before it quits by itself, then this can be
done by
\begin{enumerate}
\item selecting \menu{Run|Program reset} from the menu, or
\item pressing \key{Ctrl-F2}.
\end{enumerate}
The running program will then be aborted.

%%%%%%%%%%%%%%%%%%%%%%%%%%%%%%%%%%%%%%%%%%%%%%%%%%%%%%%%%%%%%%%%%%%%%%%
% Debugging programs
\section{Debugging programs}
\label{se:debugging}
To debug a program, it must be compiled with debug information. Compiling a
program with debug information allows you to:
\begin{enumerate}
\item Execute the program line by line.
\item Run the program up to a certain point (a breakpoint).
\item Inspect the contents of variables or memory locations while the
program is running.
\end{enumerate}
%
% Using breakpoints
%
\subsection{Using breakpoints}
Breakpoints will cause a running program to stop when the execution
reaches the line where the breakpoint was set. At that moment, control
is returned to the IDE, and it is possible to continue execution.

To set a breakpoint on the current source line, use the
\menu{Debug|Breakpoint} menu entry, or press \key{Ctrl-F8}.

A list of current breakpoints can be obtained through the
\menu{Debug|Breakpoint list} menu. The breakpoint list window
is shown in \seefig{brklist}.

\FPCpic{The breakpoint list window}{ide}{brklist}

In the breakpoint list window, the following things can be done:
\begin{description}
\item[New] Shows the breakpoint property dialog where the properties
for a new breakpoint can be entered.
\item[Edit] Shows the breakpoint property dialog where the properties of
the highlighted breakpoint can be changed.
\item[Delete] Deletes the highlighted breakpoint.
\end{description}
The dialog can be closed with the 'Close' button.
The breakpoint properties dialog is shown in \seefig{brkprop}

\FPCpic{The breakpoint properties dialog}{ide}{brkprop}

The following properties can be set:
\begin{description}
\item[Type] Set the type of the breakpoint. The following types of breakpoints
exist:
\begin{description}
\item[function] Function breakpoint. The program will stop when the function
with the given name is reached.
\item[file-line] Source line breakpoint. The program will stop when the
source file with given name and line is reached.
\item[watch] Expression breakpoint. An expression may be entered, and the
program will stop as soon as the expression changes.
\item[awatch] (access watch) Expression breakpoint. An expression that references a
memory location may be entered, and the program will stop as soon as
the memory indicated by the expression is accessed.
\item[Address] stop as soon as an address is reached.
\item[rwatch] (read watch) Expression breakpoint. An expression that references a
memory location may be entered, and the program will stop as soon as
the memory indicated by the expression is read.
\end{description}
\item[Name] Name of the function or file where to stop.
\item[Conditions] Here an expression can be entered which must evaluate to
\var{True} for the program to stop at the breakpoint. The expressions that
can be entered must be valid GDB expressions.
\item[Line] Line number in the file where to stop. Only for breakpoints of
type file-line.
\item[Ignore count] The number of times the breakpoint will be ignored
before the program stops.
\end{description}
\begin{remark}
\begin{enumerate}
\item Because the IDE uses GDB to do its debugging, it is necessary to enter all
expressions in {\em uppercase}.
\item Expressions that reference memory locations should be no longer than 16
bytes on \linux or go32v2 on an Intel processor, since the Intel processor's
debug registers are used to monitor these locations.
\item Memory location watches will not function on Win32 unless a special
patch is applied.
\end{enumerate}
\end{remark}

%
% Using watches
%
\subsection{Using watches}
When debugging information is compiled in the program, watches can be used.
Watches are expressions which can be evaluated by the IDE and shown in a
separate window. When program execution stops (e.g. at a breakpoint) all
watches will be evaluated and their current values will be shown.

Setting a new watch can be done with the \menu{Debug|Add watch} menu
command or by pressing \key{Ctrl-F7}. When this is done, the watch
property dialog appears, and a new expression can be entered.
The watch property dialog is shown in \seefig{watch}.

\FPCpic{The watch property dialog}{ide}{watch}

In the dialog, the expression can be entered. Any possible previous value
and current value are shown.
\begin{remark}
Because the IDE uses GDB to do its debugging, it is necessary to enter all
expressions in {\em uppercase} in \freebsd.
\end{remark}
A list of watches and their present value is available in the watches
window, which can be opened with the \menu{Debug|Watches} menu.
The watch list window is shown in \seefig{watchlst}.

\FPCpic{The watch list window.}{ide}{watchlst}

Pressing \key{Enter} or the space bar will show the watch property dialog
for the currently highlighted watch in the watches window.

The list of watches is updated whenever the IDE resumes control when
debugging a program.
%
% The call stack
%
\subsection{The call stack}
\label{se:callstack}
The call stack helps in showing the program flow. It shows the list of
procedures that are being called at this moment, in reverse order.
The call stack window can be shown using the \menu{Debug|Call Stack} menu.
It will show the address or procedure name of all currently active
procedures with their filename and addresses. If parameters were passed
they will be shown as well. The call stack is shown in \seefig{callstck}.

\FPCpic{The call stack window.}{ide}{callstck}

By pressing the space bar in the call stack window, the line corresponding
to the call will be highlighted in the edit window.

% The GDB Window
\subsection{The GDB window}
\label{se:gdbwindow}
The GDB window provides direct interaction with the GDB debugger.
In it, GDB commands can be typed as they would be typed in GDB.
The response of GDB will be shown in the window.

Some more information on using GDB can be found in \sees{usinggdb}, but
the final reference is of course the GDB manual itself
\footnote{Available from the Free Software Foundation website.}.
The GDB window is shown in \seefig{gdbwin}.

\FPCpic{The GDB window}{ide}{gdbwin}

%%%%%%%%%%%%%%%%%%%%%%%%%%%%%%%%%%%%%%%%%%%%%%%%%%%%%%%%%%%%%%%%%%%%%%%
% The tools menu
\section{Using Tools}
\label{se:toolsmenu}
The tools menu provides easy access to external tools. It also has
three pre-defined tools for programmers: an ASCII table, a grep tool
and a calculator. The output of the external tools can be accessed through
this menu as well.

%
% The messages window.
%
\subsection{The messages window}
\label{se:toolsmessages}
The output of the external utilities is redirected by the IDE and it
will be displayed in the messages window. The messages window is
displayed automatically, if an external tool was run. The
messages window can also be displayed manually by selecting the
menu item \menu{Tools|Messages} or by pressing the \key{F11} key.
The messages window is shown in \seefig{messages}.

\FPCpic{The messages window}{ide}{messages}

If the output of the tool contains filenames and line numbers,
the messages window can be used to navigate the source as in a browse
window:
\begin{enumerate}
\item Pressing \key{Enter} or double clicking the output line will jump
to the specified source line and close the messages window.
\item Pressing the space bar will jump to the specified source line, but
will leave the messages window open, with the focus on it. This allows the
quick selection of another message line with the arrow keys and jump to
another location in the sources.
\end{enumerate}
The algorithm which extracts the file names and line numbers from
the tool output is quite sophisticated, but in some cases it may
fail\footnote{Suggestions for improvement, or better yet, patches
that improve the algorithm, are always welcome.}.
%
% Grep
%
\subsection{Grep}
\label{se:grep}
One external tool in the Tools menu is already predefined: a
menu item to call the \file{grep} utility (\menu{Tools|Grep} or
\key{Shift-F2}). \file{Grep} searches for a given string in files and
returns the lines which contain the string. The search string can
even be a regular expression. For this menu item to work, the
\file{grep} program must be installed, since it is not distributed
with \fpc.

The messages window displayed in \seefig{messages} in the previous
section shows the output of a typical \file{grep} session. The messages
window can be used in combination with \file{grep} to find special
occurrences in the text.

\file{Grep} supports regular expressions. A regular expression is a
string with special characters which describe a whole class of
expressions. The command line in \dos or \linux has limited
support for regular expressions: entering \var{ls *.pas}
(or \var{dir *.pas}) to get a list of all Pascal files in a
directory. \file{*.pas} is something similar to a regular expression.
It uses a wildcard to describe a whole class of strings: those which
end on "\file{.pas}".
Regular expressions offer much more: for example \var{[A-Z][0-9]+}
describes all strings which begin with an upper case letter followed by
one or more digits.

It is outside the scope of this manual to describe regular expressions
in great detail. Users of a \linux system can get more information on grep
using \var{man grep} on the command line.
%
% The ASCII table.
%
\subsection{The ASCII table}
\label{se:asciitable}
The tools menu also provides an ASCII table (\menu{Tools|Ascii table}).
The ASCII table can be used to look up ASCII codes as well as to
insert characters into the window which was active when invoking the
table. 

To reveal the ASCII code of a character in the table, move the
cursor onto this character or click it with the mouse. The decimal
and hex values of the character are shown at the bottom on the
ASCII table window.

To insert a character into an editor window either:
\begin{enumerate}
\item using the mouse, double click it, or,
\item using the keyboard, press \key{Enter} while the cursor is on it.
\end{enumerate}
This is especially useful for pasting graphical characters in a constant
string.

The ASCII table remains active till another window is explicitly activated;
thus multiple characters can be inserted at once.
The ASCII table is shown in \seefig{ascii}.

\FPCpic{The ASCII table}{ide}{ascii}

%
% The calculator
%
\subsection{The calculator}
\label{se:calculator}
The calculator allows quick calculations without leaving the IDE. It is a simple
calculator, since it does not take care of operator precedence, and
bracketing of operations is not (yet) supported.

The result of the calculations can be pasted into the text using the
\key{Ctrl-Enter} keystroke. The calculator dialog is shown in
\seefig{calc}.

\FPCpic{The calculator dialog}{ide}{calc}

The calculator supports all basic mathematical operations such as
addition, subtraction, division and multiplication. They are summarised in
\seet{calculatorbasic}.
\begin{FPCltable}{p{8cm}lll}{Basic mathematical operations}{calculatorbasic}
Operation & Button & Key \\ \hline
Add two numbers & \var{+} & \key{+} \\
Subtract two numbers & \var{\-} & \key{\-} \\
Multiply two numbers & \var{*} & \key{*} \\
Divide two numbers & \var{/} & \key{/} \\
Delete the last typed digit & \var{<-} & \key{Backspace} \\
Clear display & \var{C} & \key{C} \\
Change the sign & \var{+\-} & \\
Do per cent calculation & \var{\%} & \key{\%} \\ \hline
Get result of operation & \var{=} & \key{Enter} \\ \hline
\end{FPCltable}

But also more sophisticated mathematical operations such as exponentiation
and logarithms are supported. The advanced mathematical operations are
shown in \seet{calculatoradvanced}.
\begin{FPCltable}{p{8cm}lll}{Advanced mathematical operations}{calculatoradvanced}
Operation & Button & Key \\ \hline
Calculate power & \var{x\^{}y} & \\
Calculate the inverse value & \var{1/x} & \\
Calculate the square root & \var{sqr} & \\
Calculate the natural logarithm &  \var{log} & \\
Square the display contents & \var{x\^{}2} & \\ \hline.
\end{FPCltable}

Like many calculators, the calculator in the IDE also supports storing
a single value in memory, and several operations can be done on this memory
value. The available operations are listed in \seet{calculatormemory}
\begin{FPCltable}{p{8cm}lll}{Advanced calculator commands}{calculatormemory}
Operation & Button & Key \\ \hline
Add the displayed number to the memory & \var{M+} & \\
Subtract the displayed number from the memory & \var{M-} & \\
Move the memory contents to the display & \var{M->} & \\
Move the display contents to the memory & \var{M<-} & \\
Exchange display and memory contents & \var{M<->} & \\ \hline
\end{FPCltable}
%
% Adding new tools
%
\subsection{Adding new tools}
\label{se:addingtools}
The tools menu can be extended with any external program which is command line
oriented. The output of such a program will be caught and displayed in the
messages window.

Adding a tool to the tools menu can be done using the \menu{Options|Tools} menu.
This will display the tools dialog. The tools dialog is shown in \seefig{otools}.

\FPCpic{The tools configuration dialog}{ide}{otools}

In the tools dialog, the following actions are available:
\begin{description}
\item[New] Shows the tool properties dialog where the
properties of a new tool can be entered.
\item[Edit] Shows the tool properties dialog where the
properties of the highlighted tool can be edited.
\item[Delete] Removes the currently highlighted tool.
\item[Cancel] Discards all changes and closes the dialog.
\item[OK] Saves all changes and closes the dialog.
\end{description}
The definitions of the tools are written in the desktop
configuration file. So unless auto-saving of the desktop file
is enabled, the desktop file should be saved explicitly after
the dialog is closed.

\subsection{Meta parameters}
When specifying the command line for the called tool, meta parameters can
be used. Meta parameters are variables and they are replaced
by their contents before passing the command line to the tool.

\begin{description}
\item[\$CAP]
Captures the output of the tool.
\item[\$CAP\_MSG()]
Captures the output of the tool and puts it in the messages window.
\item[\$CAP\_EDIT()]
Captures the output of the tool and puts it in a separate editor window.
\item[\$COL]
Replaced by the column of the cursor in the active editor window. If there is no
 active window or the active window is a dialog, then it is replaced by 0.
\item[\$CONFIG]
Replaced by the complete filename of the current configuration file.
\item[\$DIR()]
Replaced by the full directory of the filename argument, including the trailing
directory separator. e.g.
\begin{verbatim}
  $DIR('d:\data\myfile.pas')
\end{verbatim}
would return \verb|d:\data\|.
\item[\$DRIVE()]
Replaced by the drive letter of the filename argument. e.g.
\begin{verbatim}
  $DRIVE('d:\data\myfile.pas')
\end{verbatim}
would return \file{d:}.
\item[\$EDNAME]
Replaced by the complete file name of the file in the active edit window.
If there is no active edit window, this is an empty string.
\item[\$EXENAME]
Replaced by the executable name that would be created if the make command
was used. (i.e. from the 'Primary File' setting or the active edit window).
\item[\$EXT()]
Replaced by the extension of the filename argument.
The extension includes the dot.
e.g.
\begin{verbatim}
  $EXT('d:\data\myfile.pas')
\end{verbatim}
would return \file{.pas}.
\item[\$LINE]
Replaced by the line number of the cursor in the active edit window.
If no edit window is present or active, this is 0.
\item[\$NAME()]
Replaced by the name part (excluding extension and dot) of the filename
argument.
e.g.
\begin{verbatim}
  $NAME('d:\data\myfile.pas')
\end{verbatim}
would return \file{myfile}.
\item[\$NAMEEXT()]
Replaced by the name and extension part of the filename argument.
e.g.
\begin{verbatim}
  $NAMEEXT('d:\data\myfile.pas')
\end{verbatim}
would return \file{myfile.pas}.
\item[\$NOSWAP]
Does nothing in the IDE; it is provided only for compatibility with \tp.
\item[\$PROMPT()]
Prompt displays a dialog box that allows editing of all arguments that
come after it. Arguments that appear before the \var{\$PROMPT} keyword
are not presented for editing.

\var{\$PROMPT()} can also take an optional filename argument. If present, \var{\$PROMPT()} will load
a dialog description from the filename argument. E.g.
\begin{verbatim}
$PROMPT(cvsco.tdf)
\end{verbatim}
would parse the file \file{cvsco.tdf}, construct a dialog with it and
display it. After the dialog closed, the information entered by the user
is used to construct the tool command line.

See \sees{commanddialogs} for more information on how to create a dialog
description.
\item[\$SAVE]
Before executing the command, the active editor window is saved, even if it is not modified.
\item[\$SAVE\_ALL]
Before executing the command, all unsaved editor files are saved without prompting.
\item[\$SAVE\_CUR]
Before executing the command the contents of the active editor window are
saved without prompting if they are modified.
\item[\$SAVE\_PROMPT]
Before executing the command, a dialog is displayed asking whether any
unsaved files should be saved before executing the command.
\item[\$WRITEMSG()]
Writes the parsed tool output information to a file with name as in the argument.
\end{description}	

\subsection{Building a command line dialog box}
\label{se:commanddialogs}
When defining a tool, it is possible to show a dialog to the user, asking for
additional arguments, using the \var{\$PROMPT(filename)} command-macro.
The \fpc distribution contains some ready-made dialogs, such as a 'grep' dialog, a 'cvs checkout' dialog
and a 'cvs check in' dialog. The files for these dialogs are in the binary
directory and have an extension \file{.tdf}.

In this section, the file format for the dialog description file is explained.
The format of this file resembles a windows \file{.INI} file, where each section
in the file describes an element (or control) in the dialog.
An \var{OK} and a \var{Cancel} button will be added to the bottom of the dialog,
so these should not be specified in the dialog definition.

A special section is the \var{Main} section. It describes how the result of
the dialog will be passed to the command line, and the total size of the dialog.

\begin{remark}
Keywords that contain a string value should have the string value enclosed
in double quotes as in
\begin{verbatim}
Title="Dialog title"
\end{verbatim}
\end{remark}

The \var{Main} section should contain the following keywords:
\begin{description}
\item[Title] The title of the dialog. This will appear in the frame title of the dialog.
The string should be enclosed in quotes.
\item[Size] The size of the dialog, this is formatted as \var{(Cols,Rows)}, so
\begin{verbatim}
Size=(59,9)
\end{verbatim}
means the dialog is 59 characters wide, and 9 lines high. This size does not include
the border of the dialog.
\item[CommandLine] specifies how the command line will be passed to the
program, based on the entries made in the dialog. The text typed here will be passed
on after replacing some control placeholders with their values.

A control placeholder is the name of some control in the dialog, enclosed in
percent (\var{\%}) characters. The name of the control will be replaced with
the text associated with the control. Consider the following example:
\begin{verbatim}
CommandLine="-n %l% %v% %i% %w% %searchstr% %filemask%"
\end{verbatim}
Here the values associated with the controls named \var{l, v, i, w} and
\var{searchstr} and \var{filemask} will be inserted in the command line
string.
\item[Default]
The name of the control that is the default control, i.e. the control
that is to have the focus when the dialog is opened.
\end{description}
The following is an example of a valid main section:
\begin{verbatim}
[Main]
Title="GNU Grep"
Size=(56,9)
CommandLine="-n %l% %v% %i% %w% %searchstr% %filemask%"
Default="searchstr"
\end{verbatim}

After the \var{Main} section, a section must be specified for each control that
should appear on the dialog. Each section has the name of the control it
describes, as in the following example:
\begin{verbatim}
[CaseSensitive]
Type=CheckBox
Name="~C~ase sensitive"
Origin=(2,6)
Size=(25,1)
Default=On
On="-i"
\end{verbatim}
Each control section  must have at least the following keywords associated
with it:
\begin{description}
\item[Type] The type of control. Possible values are:
\begin{description}
\item[Label] A plain text label which will be shown on the dialog.
A control can be linked to this label, so it will be focused when
the user presses the highlighted letter in the label caption (if any).
\item[InputLine] An edit field where a text can be entered.
\item[CheckBox] A checkbox which can be in an on or off state.
\end{description}
\item[Origin] Specifies where the control should be located in the dialog.
The origin is specified as \var{(left,top)} and the top-left corner of
the dialog has coordinate \var{(1,1)} (not counting the frame).
\item[Size] Specifies the size of the control, which should be specified
as \var{(Cols,Rows)}.
\end{description}

Each control has some specific keywords associated with it;
they will be described below.

A label (\var{Type=Label}) has the following extra keywords associated
with it:
\begin{description}
\item[Text] the text displayed in the label. If one of the letters should
be highlighted so it can be used as a shortcut, then it should be enclosed
in tilde characters (\~{}). E.g. in
\begin{verbatim}
Text="~T~ext to find"
\end{verbatim}
the \var{T} will be highlighted.
\item[Link] The name of a control in the dialog may be specified.
If specified, pressing the label's highlighted letter in combination
with the \key{Alt} key will put the focus on the control specified here.
\end{description}
A label does not contribute to the text of the command line; it is for
informational and navigational purposes only. The following is an
example of a label description section:
\begin{verbatim}
[label2]
Type=Label
Origin=(2,3)
Size=(22,1)
Text="File ~m~ask"
Link="filemask"
\end{verbatim}

An edit control (\var{Type=InputLine}) allows entry of arbitrary text.
The text of the edit control will be pasted in the command line if it
is referenced there. The following keyword can be specified in a
inputline control section:
\begin{description}
\item[Value] A standard value (text) for the edit control can be
specified. This value will be filled in when the dialog appears.
\end{description}
The following is an example of an input line section:
\begin{verbatim}
[filemask]
Type=InputLine
Origin=(2,4)
Size=(22,1)
Value="*.pas *.pp *.inc"
\end{verbatim}

A checkbox control (\var{Type=CheckBox}) presents a checkbox which
can be in one of two states, \var{on} or \var{off}. With each of
these states, a value can be associated which will be passed on to
the command line. The following keywords can appear in a checkbox
type section:
\begin{description}
\item[Name] The text that appears after the checkbox.
If there is a highlighted letter in it, this letter can be used
to set or unset the checkbox using the \key{Alt}-letter combination.
\item[Default] Specifies whether the checkbox is checked or not when
the dialog appears (value \var{on} or \var{off}).
\item[On] The text associated with this checkbox if it is in the checked
state.
\item[Off] The text associated with this checkbox if it is in the
unchecked state.
\end{description}
The following is an example of a valid checkbox description:
\begin{verbatim}
[i]
Type=CheckBox
Name="~C~ase sensitive"
Origin=(2,6)
Size=(25,1)
Default=On
On="-i"
\end{verbatim}
If the checkbox is checked, then the value \var{-i} will be added on
the command line of the tool. If it is unchecked, no value will be added.

%%%%%%%%%%%%%%%%%%%%%%%%%%%%%%%%%%%%%%%%%%%%%%%%%%%%%%%%%%%%%%%%%%%%%%%
% Project management
\section{Project management and compiler options}
\label{se:projectmanagement}
Project management in Pascal is much easier than with C. The
compiler knows from the source which units, sources etc. it needs.
So the \fpc IDE does not need a full featured project manager like
some C development environments offer. Nevertheless there are some
settings in the IDE which apply to projects.
%
% The primary file
%
\subsection{The primary file}
\label{se:primaryfile}
Without a primary file the IDE compiles/runs the source of the active
window when a program is started. If a primary file is specified,
the IDE always compiles/runs this source, even if another
source window is active. With the menu item \menu{Compile|Primary file...}
a file dialog can be opened where the primary file can be selected.
Only the menu item \menu{Compile|Compile} compiles the active window
regardless. This is useful if a large project is being edited, and 
only the syntax of the current source should be checked.

The menu item \menu{Compiler|Clear primary file} restores the default
behaviour of the IDE, i.e. the 'compile' and 'run' commands apply to the
active window.
%
% The directory dialog
%
\subsection{The directory dialog}
In the directory dialog, the directories can be specified where the
compiler should look for units, libraries, object files. It also says
where the output files should be stored. Multiple directories (except
for the output directory) can be entered, separated by semicolons.
The directories dialog is shown in \seefig{odirs}.

\FPCpic{The directories configuration dialog}{ide}{odirs}

The following directories can be specified:
\begin{description}
\item[EXE \& PPU directories] Specifies where the compiled units and
executables will go. (\seeo{FE} on the command line.)
\item[Object directories] Specifies where the compiler looks for external
object files. (\seeo{Fo} on the command line.)
\item[Library directories] Specifies where the compiler (more exactly, the
linker) looks for external libraries. (\seeo{Fl} on the command line.)
\item[Include directories] Specifies where the compiler will look for
include files, included with the \var{\{\$i \}} directive.
(\seeo{Fi} or \seeo{I} on the command line.)
\item[Unit directories] Specifies where the compiler will look for compiled
units. The compiler always looks first in the current directory, and also in
some standard directories. (\seeo{Fu} on the command line.)
\end{description}
%
% The target operating system.
%
\subsection{The target operating system}
The menu item \menu{Compile|Target} allows  specification of the target
operating system for which the sources will be compiled.
Changing the target doesn't affect any compiler switches or
directories. It does affect some defines defined by the compiler.
The settings here correspond to the option on the command line 
\seeo{T}. A sample compilation target dialog is shown in \seefig{target}:
the actual dialog will show only those targets that the IDE actually 
supports.

\FPCpic{The compilation target dialog}{ide}{target}

The following targets can be set (the list depends on the platform for
which the IDE was compiled):
\begin{description}
\item[Dos (go32v1)] This switch will dissapear in time as this target is no
longer being maintained.
\item[Dos (go32v2)] Compile for \dos, using version 2 of the Go32 extender.
\item[FreeBSD] Compile for \freebsd.
\item[Linux] Compile for \linux.
\item[OS/2] Compile for OS/2 (using the EMX extender).
\item[Windows] Compile for \windows.
\end{description}
The currently selected target operating system is shown in the 
\menu{Target} menu item in the \menu{Compile} menu. Initially, 
this will be set to the operating system for which the IDE was compiled.
%
% Other compiler options
%
\subsection{Compiler options}
The menu \menu{Options|Compiler} allow the settting of options that affect the
compilers behaviour. When this menu item is chosen, a dialog pops up that
displays several tabs.

There are six tabs:
\begin{description}
\item[Syntax] Here options can be set that affect the various syntax aspects
of the code. They correspond mostly to the \var{-S} option on the command
line (\sees{sourceoptions}).
\item[Code generation] These options control the generated code; they are
mostly concerned with the \var{-C} and \var{-X} command line options.
\item[Verbose] These set the verbosity of the compiler when compiling. The
messages of the compiler are shown in the compiler messages window (can be
called with \key{F12}).
\item[Browser] Options concerning the generated browser information. Browser
information needs to be generated for the symbol browser to work.
\item[Assembler] Options concerning the reading of assembler blocks (-R on
the command line) and the generated assembler (\var{-A} on the command line)
\item[Processor] Here the target processor can be selected.
\end{description}

On each tab page, there are two entry boxes: the first for
Conditional defines and the second for additional compiler arguments.
The symbols, and arguments, should be separated with semi-colons.

The syntax tab of the compiler options dialog is shown in \seefig{ocompa}.

\FPCpic{The syntax options tab}{ide}{ocompa}

In the syntax options dialog, the following options can be set:
\begin{description}
\item[Stop after first error]  when checked, the compiler stops after the
first error. Normally the compiler continues compiling till a fatal error is
reached. (\seeo{Se} on the command line)
\item[Allow label and goto] Allow the use of label declarations and goto
statements (\seeo{Sg} on the command line).
\item[Enable macros] Allow the use of macros (\seeo{Sm}).
\item[Allow inline] Allow the use of inlined functions (\seeo{Sc} on
the command line).
\item[Include assertion code] Include \var{Assert} statements in the code.
\item[Load kylix compat. unit] Load the Kylix compatibility unit.
\item[Allow STATIC in objects] Allow the \var{Static} modifier for object
methods (\seeo{St} on the command line) 
\item[C-like operators]
Allows the use of some extended operators such as \var{+=, -=} etc.
(\seeo{Sc} on the command line).
\item[Compiler mode] select the appropriate compiler mode:
\begin{description}
\item[Free Pascal Dialect] The default \fpc compiler mode (\var{FPC}).
\item[Object pascal extensions on] 
Enables the use of classes and exceptions (\seeo{Sd} on the command line).
\item[Turbo pascal compatible] Try to be more \tp compatible (\seeo{So} on
the command line).
\item[Delphi compatible] Try to be more \delphi compatible (\seeo{Sd} on
the command line).
\item[Macintosh Pascal dialect] Try to be Macintosh pascal compatible.
\end{description}
%\item[Strict var-strings] Not used.
%\item[Extended syntax] Not used.
%\item[Allow MMX operations] Allow MMX operations.
\end{description}

The code generation tab of the compiler options dialog is shown in
\seefig{ocompb}.

\FPCpic{The code generation options tab}{ide}{ocompb}

In the code generation dialog, the following options can be set:
\begin{description}
\item[Run-time checks] Controls what run-time checking code is generated. If
such a check fails, a run-time error is generated.
The following checking code can be generated:
\begin{description}
\item[Range checking] Checks the results of enumeration and subset
type operations (\seeo{Cr} command line option).
\item[Stack checking] Checks whether the stack limit is not
reached (\seeo{Cs} command line option).
\item[I/O checking] Checks the result of IO operations
(\seeo{Ci} command line option).
\item[Integer overflow checking] Checks the result of integer operations
(\seeo{Co} command line option).
\item[Object method call checking] Check the validity of the method pointer
prior to calling it.
\item[Position independent code] Generate PIC code.
\item[Create smartlinkable units] Create smartlinkable units.
\end{description}
\item[Optimizations] What optimizations should be used when compiling:
\begin{description}
\item[Generate faster code] Corresponds to the \var{-OG} command line option.
\item[Generate smaller code] Corresponds to the \var{-Og} command line option.
%\item[Use register variables] Corresponds to the \var{-Or} command line option.
%\item[Uncertain optimizations] Corresponds to the \var{-Ou} command line option.
%\item[Level 1 optimizations] Corresponds to the \var{O1} command line option.
%\item[Level 2 optimizations] Corresponds to the \var{O1} command line option.
\end{description}
\end{description}
More information on these switches can be found in \sees{codegen}.

The processor tab of the compiler options dialog is shown in
\seefig{ocompf}.

In the processor dialog, the target processor can be set. The
compiler can use different optimizations for different processors.

\FPCpic{The processor selection tab}{ide}{ocompf}

The verbose tab of the compiler options dialog is shown in
\seefig{ocompc}.

\FPCpic{The verbosity options tab}{ide}{ocompc}

In this dialog, the following verbosity options can be set
(on the command line: \seeo{v}):
\begin{description}
\item[Warnings] Generate warnings. 
Corresponds to \var{-vw} on the command line.
\item[Notes] Generate notes. 
Corresponds to \var{-vn} on the command line.
\item[Hints] Generate hints. 
Corresponds to \var{-vh} on the command line.
\item[General info] Generate general information. 
Corresponds to \var{-vi} on the command line.
\item[User,tried info] Generate information on used and tried files. 
Corresponds to \var{-vut} on the command line.
\item[All] Switch on full verbosity. 
Corresponds to \var{-va} on the command line.
\item[Show all procedures if error] If an error using overloaded procedure
occurs, show all procedures. 
Corresponds to \var{-vb} on the command line.
\end{description}

The browser tab of the compiler options dialog is shown in \seefig{ocompd}.

\FPCpic{The browser options tab}{ide}{ocompd}

In this dialog, the browser options can be set:
\begin{description}
\item[No browser] (default) No browser information is generated by the
compiler.
\item[Only global browser] Browser information is generated for global
symbols only, i.e. symbols defined not in a procedure or function (\var{-b} on the command line)
\item[Local and global browser] Browser information is generated for all
symbols, i.e. also for symbols that are defined in procedures or functions
 (\var{-bl} on the command line)
\end{description}
\begin{remark}
If no browser information is generated, the symbol browser of the IDE will
not work.
\end{remark}
The assembler tab of the compiler options dialog is shown in
\seefig{ocompe}. The actual dialog may vary, as it depends on the 
target CPU the IDE was compiled for.

\FPCpic{The assembler options tab}{ide}{ocompe}

In this dialog, the assembler reader and writer options can be set:
\begin{description}
\item[Assembler reader] This permits setting the style of the assembler blocks
in the sources:
\begin{description}
%\item[Direct assembler] The assembler blocks are copied as-is to the output
%(\var{-Rdirect} on the command line).
\item[AT\&T assembler] The assembler is written in \var{AT\&T} style
assembler (\var{-Ratt} on the command line).
\item[Intel style assembler] The assembler is written in \var{Intel} style
assembler blocks (\var{-Rintel} on the command line).
\end{description}
remark that this option is global, but locally the assembler style can be
changed with compiler directives.
\item[Assembler info] When writing assembler files, this option decides
which extra information is written to the assembler file in comments:
\begin{description}
\item[List source] The source lines are written to the assembler files
together with the generated assembler (\var{-al} on the command line).
\item[List register allocation] The compiler's internal register
allocation/deallocation information is written to the assembler file
(\var{-ar} on the command line).
\item[List temp allocation] The temporary register allocation/deallocation
is written to the assembler file. (\var{-at} on the command line).
\item[List node allocation] The node allocation/deallocation
is written to the assembler file. (\var{-an} on the command line).
\item[use pipe with assembler] use a pipe on unix systems when feeding the
assembler code to an external assembler.
\end{description}
The latter three of these options are mainly useful for debugging the
compiler itself, it should rarely be necessary to use these.
\item[Assembler output] This option tells the compiler what assembler output
should be generated.
\begin{description}
\item[Use default output] This depends on the target.
\item[Use GNU as] Assemble using \gnu \file{as} (\var{-Aas} on the
command line).
\item[Use NASM coff] Produce NASM coff assembler (go32v2, \var{-Anasmcoff} on the
command line)
\item[Use NASM elf] Produce NASM elf assembler (\linux, \var{-Anasmelf} on
the command line).
\item[Use NASM obj] Produce NASM obj assembler (\var{-Anasmobj} on the
command line).
\item[Use MASM] Produce MASM (Microsoft assembler) assembler (\var{-Amasm} on the
command line).
\item[Use TASM] Produce TASM (Turbo Assembler) assembler (\var{-Atasm} on the
command line).
\item[Use coff] Write binary coff files directly using the internal
assembler (go32v2, \var{-Acoff} on the command line).
\item[Use pecoff] Write binary pecoff files files directly using the
internal writer. (Win32)
\end{description}
\end{description}
%
% Linker options
%
\subsection{Linker options}
The linker options can be set in the menu \menu{Options|Linker}. 
It permits the specification of how libraries and units are linked, 
and how the linker should be called.
The linker options dialog is shown in \seefig{olinker}.

\FPCpic{The linker options dialog}{ide}{olinker}

The following options can be set:
\begin{description}
\item[Call linker after] If this option is set then a script is written
which calls the linker. This corresponds to the \var{s} option on the
command line (\seeo{s}).
\item[Only link to static library] Only use static libraries.

\item[Preferred library type] With this option, the type of library to be
linked in can be set:
\begin{description}
\item[Target default] This depends on the platform.
\item[Dynamic libraries] Tries to link in units in dynamic libraries.
(option \var{-XD} on the command line.)
\item[Static libraries] Tries to link in units in static libraries.
(option \var{-XS} on the command line.)
\item[Smart libraries] Tries to link in units in smart-linked libraries.
(option \var{-XX} on the command line.)
\end{description}
\end{description}
%
% Memory sizes dialog
%
\subsection{Memory sizes}
The memory sizes dialog (reachable via \menu{options|Memory sizes}) permits
the entry of the memory sizes for the project.
The memory sizes dialog is shown in \seefig{omemsize}.

\FPCpic{The memory sizes dialog}{ide}{omemsize}

The following sizes can be entered:
\begin{description}
\item[Stack size] Sets the size of the stack in bytes
(option \var{-Cs} on the command line). This size may be ignored on some
systems.
\item[Heap size] Sets the size of the heap in bytes; (option \var{-Ch} on
the command line). Note that the heap grows dynamically as much as the OS
allows.
\end{description}

%
% Debugging options
%
\subsection{Debug options}
\label{se:debugoptions}
In the debug options dialog (reachable via \menu{Options|Debugger}), some 
options for inclusion of debug information in the binary can be set; 
it is also possible to add additional compiler options in this dialog.
The debug options dialog is shown in \seefig{odebug}.
\FPCpic{The debug options dialog}{ide}{odebug}

The following options can be set:
\begin{description}
\item[Debugging information] tells the compiler which debug information
should be compiled in. One of the following options can be chosen:
\begin{description}
\item[Strip all debug symbols from executable] Will strip all debug and
symbol information from the binary. (option \var{-Xs} on the command line).
\item[Skip debug information generation] Do not generate debug information
at all.
\item[Generate debug symbol information] Include debug information in the
binary (option \var{-g} on the command line). Please note that no debug
information for units in the Run-Time Library will be included, unless a
version of the RTL compiled with debug information is available. Only units
specific to the current project will have debug information included.
\item[Generate also backtrace line information] Will compile with debug
information, and will additionally include the \file{lineinfo} unit in the
binary, so that in case of an error the backtrace will contain the file names and
line numbers of procedures in the call-stack. (Option \var{-gl} on the
command line.)
\item[Generate valgrind compatible debug info] Generate debug information
that can be read with valgrind (a memory checking tool).
\end{description}
\item[Profiling switches] Tells the compiler whether or not profile code
should be included in the binary.
\begin{description}
\item[No profile information] Has no effect, as it is the default.
\item[Generate Profile code for gprof] If checked, profiling code is
included in the binary (option \var{-p} on the command line).
\end{description}
%\item[Addition compiler args] Here arbitrary options can be entered as they
%would be entered on the command line, they will be passed on to the compiler
%as typed here.
\item[Use another TTY for Debuggee]
An attempt will be made to redirect the output of the program
being debugged to another window (terminal), whose file name should
be entered here.
\end{description}

%
% The switches mode.
%
\subsection{The switches mode}
\label{se:compilermode}
The IDE allows saving a set of compiler settings under a common name. 
It provides 3 names under which the switches can be saved:
\begin{description}
\item[Normal] For normal (fast) compilation.
\item[Debug] For debugging; intended to set most debug switches on. Also
useful for setting conditional defines that e.g. allow including some
debug code.
\item[Release] For a compile of the program as it should be released, debug
information should be off, the binary should be stripped, and optimizations
should be used.
\end{description}
Selecting one of these modes will load the compiler options as they were
saved the last time the selected mode was active, i.e. it doesn't
specifically set or unset options.

When setting and saving compiler options, be sure to select the correct
switch mode first; it makes little sense to set debug options while the
release switch is active.
The switches mode dialog is shown in \seefig{oswitch}.

\FPCpic{The switches mode dialog}{ide}{oswitch}

%%%%%%%%%%%%%%%%%%%%%%%%%%%%%%%%%%%%%%%%%%%%%%%%%%%%%%%%%%%%%%%%%%%%%%%
% Customize the IDE
\section{Customizing the IDE}
The IDE is configurable over a wide range of parameters: colors can be changed, 
screen resolution. The configuration settings can be reached via the
sub-menu \var{Environment} in the \var{Options} menu.
%
% general preferences
%
\subsection{Preferences}
The {\em preferences dialog} is called by the menu item
\menu{Options|Environment|Preferences}.
The preferences dialog is shown in \seefig{oeprefs}.

\FPCpic{The preferences dialog}{ide}{oeprefs}

\begin{description}
\item[Video mode]
The drop down list at the top of the dialog allows selecting a video mode.
The available video modes depend on the system on which the IDE
is running.
\begin{remark}
\begin{enumerate}
\item The video mode must be selected by pressing space or clicking
on it. If the drop down list is opened while leaving the dialog,
the new video mode will not be applied.
\item For the \dos version of the IDE, the following should be noted:
When using VESA modes, the display refresh rate may be very low.
On older graphics card (1998 and before), it is possible to use the
{\em UniVBE} driver from {\em SciTech}\footnote{It can be downloaded from
\seeurl{http://www.informatik.fh-muenchen.de/~ifw98223/vbehz.htm}
{http://www.informatik.fh-muenchen.de/\~{}ifw98223/vbehz.htm}}
% It is quite outdated
%(last update somewhere in 1998).
%For newer graphics cards which support VESA 3.0, you can try to get one
%of the TSR programs
%\footnote{\textbf{T}erminate and \textbf{S}tay \textbf{R}esisdent}
% available at the net to customize the refresh rate.
%%%%!!!!!!!! footnote with URL
\end{enumerate}
\end{remark}
\item[Desktop File]
Specifies where the desktop file is saved: the current directory, or the
directory where the config file was found.
\item[Auto save]
Here it is possible to set which files are saved when a program is run or
when the IDE is exited:
\begin{description}
\item[Editor files] The contents of all open edit windows will be saved.
\item[Environment] The current environment settings will be saved.
\item[Desktop] The desktop file with all desktop settings (open windows,
history lists, breakpoints etc.) will be saved.
\end{description}
\item[Options]
Some special behaviours of the IDE can be specified here:
\begin{description}
\item[Auto track source]
\item[Close on go to source] When checked, the messages window is closed
when the 'go to source line' action is executed.
\item[Change dir on open] When a file is opened, the directory of that file
is made the current working directory.
\end{description}
\end{description}
%
% Desktop customization
%
\subsection{The desktop}
\label{se:prefdesktop}
The desktop preferences dialog allows to specify what elements of the
desktop are saved across sessions, i.e. they are saved when the IDE is left,
and they are again restored when the IDE is started the next time.
They are saved in the file \file{fp.dsk}.
The desktop preferences dialog is shown in \seefig{oedesk}.

\FPCpic{The desktop preferences dialog}{ide}{oedesk}

The following elements can be saved and restored across IDE sessions:
\begin{description}
\item[History lists] Most entry boxes have a history list where previous
entries are saved and can be selected. When this option is checked, these
entries are saved in the desktop file. On by default.
\item[Clipboard content]
When checked, the contents of the clipboard are also saved to disk. Off by
default.
\item[Watch expressions]
When checked, all watch expressions are saved in the desktop file. Off by
default.
\item[Breakpoints]
When checked, all breakpoints with their properties are saved in the
desktop file. Off by default.
\item[Open windows]
When checked, the list of files in open editor windows is saved in the
desktop file, and the windows will be restored the next time the IDE
is run. On by default.
\item[Symbol information]
When checked, the information for the symbol browser is saved in the desktop
file. Off by default.
\item[CodeComplete wordlist]
When checked, the list of codecompletion words is saved. On by default.
\item[CodeTemplates]
When checked, the defined code templates are saved. On by default.
\end{description}

\begin{remark}
The format of the desktop file changes between editor versions. So
when installing a new version, it may be necessary to delete the
\file{fp.dsk} files wherever the IDE searches for them.
\end{remark}

%
% Editor customization
%
\subsection{The Editor}
Several aspects of the editor window behaviour can be set in this dialog.
The editor preferences dialog is shown in \seefig{oeeditor}. Note that
some of these options affect only newly opened windows, not already 
opened windows (e.g. Vertical Blocks, Highlight Column/Row).

\FPCpic{The editor preferences dialog}{ide}{oeeditor}

The following elements can be set in the editor preferences dialog:
\begin{description}
\item[Create backup files]
Whenever an editor file is saved, a backup is made of the old file. On by
default.
\item[Insert mode] Start with insert mode.
\item[Auto indent mode]
Smart indenting is on. This means that pressing \key{Enter} will position the
cursor on the next line in the same column where text starts on the current
line. On by default.
\item[Use tab characters]
When the tab key is pressed, use a tab character. Normally, when the tab key
is pressed, spaces are inserted. When this option is checked, tab characters
will be inserted instead. Off by default.
\item[Backspace unindents]
Pressing the \key{Bksp} key will unindent if the beginning of the text on
the current line is reached, instead of deleting just the previous
character. On by default.
\item[Persistent blocks]
When a selection is made, and the cursor is moved, the selection is not
destroyed, i.e. the selected block stays selected. On by default.
\item[Syntax highlight]
Use syntax highlighting on the files that have an extension which appears in
the list of highlight extensions. On by default.
\item[Block insert cursor]
The insert cursor is a block instead of an underscore character. By default
the overwrite cursor is a block. This option reverses that behaviour. Off by
default.
\item[Vertical blocks]
When selecting blocks spanning several lines, the selection doesn't 
contain the entirety of the lines within the block; instead, it 
contains the lines as far as the column on which the cursor is located.
Off by default.
\item[Highlight column]
When checked, the current column (i.e. the column where the cursor is) is
highlighted. Off by default.
\item[Highlight row]
When checked, the current row (i.e. the row where the cursor is) is
highlighted. Off by default.
\item[Auto closing brackets]
When an opening bracket character is typed, the closing bracket is also
inserted at once. Off by default.
\item[Keep trailing spaces]
When saving a file, the spaces at the end of lines are stripped off. This
option disables that behaviour; i.e. any trailing spaces are also saved
to file. Off by default.
\item[Codecomplete enabled]
Enable code completion. On by default.
\item[Enable folds]
Enable code folding. Off by default.
\item[Tab size]
The number of spaces that are inserted when the \key{Tab} key is pressed.
The default value is 8.
\item[Indent size]
The number of spaces a block is indented when calling the block indent function.
The default value is 2.
\item[Highlight extensions]
When syntax highlighting is on, the list of file masks entered here will be
used to determine which files are highlighted. File masks should be
separated with semicolon (;) characters. The default is
\file{*.pas;*.pp;*.inc}.
\item[File patterns needing tabs]
Some files (such as makefiles) need actual tab characters instead of spaces.
Here a series of file masks can be entered to indicate files for which tab 
characters will always be used. Default is \file{make*;make*.*}.
\end{description}
\begin{remark}
These options will not be applied to already opened windows; only newly
opened windows will have these options.
\end{remark}
%
% Mouse customization
%
\subsection{Keyboard \& Mouse}
\label{se:prefmouse}
The Keyboard \& mouse options dialog is called by the menu item
\menu{Options|Environment|Keyboard \& Mouse}. It allows adjusting the behaviour of the
keyboard and mouse as well as the sensitivity of the mouse.
The keyboard and mouse options dialog is shown in \seefig{oemouse}.

\FPCpic{The Keyboard \& mouse options dialog}{ide}{oemouse}

\begin{description}
\item[Keys for copy, cut and paste] Set the keys to use for clipboard
operations:
\begin{itemize}
\item CUA-91 convention (Shift+Del,Ctrl+Ins,Shift+Ins)
\item Microsoft convention (Ctrl+X,Ctrl+C,Ctrl+V)
\end{itemize}
\item[Mouse double click]
The slider can be used to adjust the double click speed. Fast means that the
time between two clicks is very short; slow means that the time between two
mouse clicks can be quite long.
\item[Reverse mouse buttons]
the behaviour of the left and right mouse buttons can be swapped by
by checking the checkbox; this is especially useful for left-handed people.
\item[Ctrl+Right mouse button]
Assigns an action to a right mouse button click while holding the
\key{Ctrl} key pressed.
\item[Alt+right mouse button]
Assigns an action to right mouse button click while holding the
\key{Alt} key pressed.
\end{description}

The following actions can be assigned to \key{Ctrl}-Right mouse button or
\key{Alt}-right mouse button:
\begin{description}
\item[Nothing] No action is associated to the event.
\item [Topic search] The keyword at the mouse cursor is searched in the
help index.
\item [Go to cursor] The program is executed until the line where
the mouse cursor is located.
\item [Breakpoint] Set a breakpoint at the mouse cursor position.
\item [Evaluate] Evaluate the value of the variable at the mouse
cursor.
\item [Add watch] Add the variable at the mouse cursor to the
watch list.
\item [Browse symbol] The symbol at the mouse cursor is displayed
in the browser.
\end{description}

%
% Color customization
%
%\subsection{Colors}
%\label{se:prefcolors}
%Almost all elements of the IDE such as borders input fields, buttons and so
%on can have their color set in this dialog. The dialog sets the colors for
%all elements at once, i.e. it is not so that the color of one particular
%button can be set.
%
%The syntax highlighting colors for the editor windows of the IDE can also
%be set in this dialog.
%The colors dialog is shown in \seefig{oecolors}.
%
%\FPCpic{The colors dialog}{ide}{oecolors}
%
%The following elements are visible in the color dialog:
%\begin{description}
%\item[Group]
%Here the group to be customized is displayed; A group is a specific window
%or series of windows in the editor. A special group is {\em Syntax} which
%sets the colors for syntax highlighting.
%\begin{description}
%\item[Browser] Sets the colors for the symbol browser window.
%\item[Clock] Sets the colors for the clock in the menu.
%\item[Desktop] Sets the colors for the desktop.
%\item[Dialogs] Sets the colors for the dialog windows.
%\item[Editor] Sets the colors for the editor windows.
%\item[Help] Sets the colors for the help windows.
%\item[Menus] Sets the colors used in the menus.
%\item[Syntax] Sets the colors used when performing syntax highlighting in the
%editor windows.
%\end{description}
%\item[item]
%Here the item for the current group can be selected. The foreground and
%background of this item can be set using the color selectors on the right of
%the dialog.
%\item[Foreground]
%Sets the foreground color of the selected item.
%\item[background]
%Sets the background color of the selected item.
%\item[Sample text]
%This shows the colors of the selected item in a sample text.
%\end{description}
%Setting a good color scheme is important especially for syntax highlighting;
%a good syntax highlighting scheme helps in eliminating errors when typing,
%without needing to compile the sources.
%%
%%%%%%%%%%%%%%%%%%%%%%%%%%%%%%%%%%%%%%%%%%%%%%%%%%%%%%%%%%%%%%%%%%%%%%%
% The help system
\section{The help system}

More information on how to handle the IDE, or about the use of various
calls in the RTL, explanations regarding the syntax of a Pascal statement,
can be found in the \emph{help system}. The help system is activated
by pressing \key{F1}.

\subsection{Navigating in the help system}
The help system contains hyperlinks; these are sensitive locations that
lead to another topic in the help system. They are marked by a different
color. The hyperlinks can be activated in one of two ways:
\begin{enumerate}
\item by directly clicking the one you want with the mouse, or
\item by using the \key{Tab} and \key{Shift-Tab} keys to move between
the different hyperlinks of a page and then pressing the \key{Enter} 
key to activate the one you want.
\end{enumerate}

When \key{Shift-F1} is pressed, the contents of the help system are
displayed. To go back to the previous help topic, press \key{Alt-F1}.
This also works if the help window isn't displayed on the desktop; the help
window will then be activated.

%
% Working with help files.
%
\subsection{Working with help files}
The IDE contains a help system which can display the following file formats:
\begin{description}
\item[TPH] The help format for the Turbo Pascal help viewer.
\item[INF] The OS/2 help format.
\item[NG] The Norton Guide Help format.
\item[HTML] HTML files.
\end{description}
In future some more formats may be added. However, the above formats should
cover already a wide spectrum of available help files.

\begin{remark}
Concerning the support for HTML files the following should be noted:
\begin{enumerate}
\item
The HTML viewer of the  help system is limited, it can only handle the
most basic HTML files (graphics excluded), since it is only designed
to display the \fpc help files. \footnote{...but feel free to improve it and send patches to the
\fpc development team...}.
\item
When the HTML help viewer encounters a graphics file, it will try and find a
file with the same name but an extension of \file{.ans}; If this file is
found, this will be interpreted as a file with ANSI escape sequences, and
these will be used to display a text image. The displays of the IDE dialogs
in the IDE help files are made in this way.
\end{enumerate}
\end{remark}

The menu item \menu{Help|Files} permits help files to be added to, 
and deleted from, the list of files in the help table of contents.
The help files dialog is displayed in \seefig{helpfils}.

\FPCpic{The help files dialog}{ide}{helpfils}

The dialog lists the files that will be presented in the table of contents
window of the help system. Each entry has a small descriptive title and a
filename next to it. The following actions are available when adding help
files:
\begin{description}
\item[New] Adds a new file. IDE will display a prompt, in which the
location of the help file should be entered.

If the added file is an HTML file, a dialog box will be displayed
which asks for a title. This title will then be included in the
contents of help.
\item[Delete] Deletes the currently highlighted file from the help system.
It is \emph{not} deleted from the hard disk; only the help system entry is
removed.
\item[Cancel] Discards all changes and closes the dialog.
\item[OK] Saves the changes and closes the dialog.
\end{description}

The \fpc documentation in HTML format can be added to the IDE's help system.
This way the documentation can be viewed from within the IDE. If \fpc has
been installed using the installer, the installer should have added the
FPC documentation to the list of help files, if the documentation was
installed as well.

%
% The about dialog.
%
\subsection{The about dialog}
\label{se:about}
The {\em about dialog}, reachable through (\menu{Help|About...}) shows some
information about the IDE, such as the version number, the date it was built,
what compiler and debugger it uses. When reporting bugs about the IDE, please
use the information given by this dialog to identify the version of the IDE
that was used.

It also displays some copyright information.

%%%%%%%%%%%%%%%%%%%%%%%%%%%%%%%%%%%%%%%%%%%%%%%%%%%%%%%%%%%%%%%%%%%%%%%
% Keyboard shortcuts
\section{Keyboard shortcuts}
\label{se:keyshortcuts}
A lot of keyboard shortcuts used by the IDE are compatible with
WordStar and should be well known to Turbo Pascal users.

Below are the following tables:
\begin{enumerate}
\item In \seet{shortcutsgeneral} some shortcuts for handling the IDE windows
and Help are listed.
\item In \seet{shortcutscompiler} the shortcuts for compiling, running and
debugging a program are presented.
\item In \seet{shortcutsnavigation} the navigation keys are described.
\item In \seet{shortcutsedit} the editing keys are listed.
\item In \seet{shortcutsblock} all block command shortcuts are listed.
\item In \seet{shortcutsselection} all selection-changing shortcuts are
presented.
\item In \seet{shortcutsmisc} some general shortcuts,
which do not fit in the previous categories, are presented.
\end{enumerate}

\begin{FPCltable}{p{5cm}ll}{General}{shortcutsgeneral}
Command & Shortcut key & Alternative \\ \hline
Help & \key{F1} & \\
Goto last help topic & \key{Alt-F1} & \\
Search word at cursor position in help & \key{Ctrl-F1} & \\
Help index & \key{Shift-F1} & \\
Close active window & \key{Alt-F3} & \\
Zoom/Unzoom window & \key{F5} & \\
Move/Zoom active window & \key{Ctrl-F5} & \\
Switch to next window & \key{F6} & \\
Switch to last window & \key{Shift-F6} & \\
Menu & \key{F10} & \\
Local menu & \key{Alt-F10} & \\
List of windows & \key{Alt-0} & \\
Active another window & \key{Alt-<digit>} & \\
Call \file{grep} utility & \key{Shift-F2} & \\
Exit IDE & \key{Alt-X} & \\
\end{FPCltable}

%%%%%%%%%%%%%%%%%%%%%%%%%%%%%%%%%%%%%%%%%%%%%%%%%%%%%%%%%%%%%%%%%%%%%%%%%%%%
\begin{FPCltable}{p{5cm}ll}{Compiler}{shortcutscompiler}
Command & Shortcut key & Alternative \\
\hline
Reset debugger/program & \key{Ctrl-F2} & \\
Display call stack & \key{Ctrl-F3} & \\
Run as far as the cursor & \key{F4} & \\
Switch to user screen & \key{Alt-F5} & \\
Trace into & \key{F7} & \\
Add watch & \key{Ctrl-F7} & \\
Step over & \key{F8} & \\
Set breakpoint at current line & \key{Ctrl-F8} & \\
Make & \key{F9} & \\
Run & \key{Ctrl-F9} & \\
Compile the active source file & \key{Alt-F9} & \\
Message & \key{F11} & \\
Compiler messages & \key{F12} & \\
\end{FPCltable}

%%%%%%%%%%%%%%%%%%%%%%%%%%%%%%%%%%%%%%%%%%%%%%%%%%%%%%%%%%%%%%%%%%%%%%%%%%%%
\begin{FPCltable}{p{5cm}ll}{Text navigation}{shortcutsnavigation}
Command & Shortcut key & Alternative \\
\hline
Char left & \key{Arrow left} & \key{Ctrl-S} \\
Char right & \key{Arrow right} & \key{Ctrl-D} \\
Line up & \key{Arrow up} & \key{Ctrl-E} \\
Line down & \key{Arrow down} & \key{Ctrl-X} \\
Word left & \key{Ctrl-Arrow left} & \key{Ctrl-A} \\
Word right & \key{Ctrl-Arrow right} & \key{Ctrl-F} \\
Scroll one line up & \key{Ctrl-W} & \\
Scroll one line down & \key{Ctrl-Z} & \\
Page up & \key{PageUp} & \key{Ctrl-R} \\
Page down & \key{PageDown} & \\
Beginning of Line & \key{Pos1} & \key{Ctrl-Q-S} \\
End of Line & \key{End} & \key{Ctrl-Q-D} \\
First line of window & \key{Ctrl-Home} & \key{Ctrl-Q-E} \\
Last line of window & \key{Ctrl-End} & \key{Ctrl-Q-X} \\
First line of file & \key{Ctrl-PageUp} & \key{Ctrl-Q-R} \\
Last line of file & \key{Ctrl-PageDown} & \key{Ctrl-Q-C} \\
Last cursor position & \key{Ctrl-Q-P} & \\
Find matching block delimiter & \key{Ctrl-Q-[} & \\
Find last matching block delimiter & \key{Ctrl-Q-]} & \\
\end{FPCltable}
%%%%%%%%%%%%%%%%%%%%%%%%%%%%%%%%%%%%%%%%%%%%%%%%%%%%%%%%%%%%%%%%%%%%%%%%%%%%
\begin{FPCltable}{p{5cm}ll}{Edit}{shortcutsedit}
Command & Shortcut key & Alternative \\
\hline
Delete char & \key{Del} & \key{Ctrl-G} \\
Delete left char & \key{Backspace} & \key{Ctrl-H} \\
Delete line & \key{Ctrl-Y} & \\
Delete til end of line & \key{Ctrl-Q-Y} & \\
Delete word & \key{Ctrl-T} & \\
Insert line & \key{Ctrl-N} & \\
Toggle insert mode & \key{Insert} & \key{Ctrl-V} \\
\end{FPCltable}

%%%%%%%%%%%%%%%%%%%%%%%%%%%%%%%%%%%%%%%%%%%%%%%%%%%%%%%%%%%%%%%%%%%%%%%%%%%%
\begin{FPCltable}{p{5cm}ll}{Block commands}{shortcutsblock}
Command & Shortcut key & Alternative \\
\hline
Goto Beginning of selected text & \key{Ctrl-Q-B} & \\
Goto end of selected text & \key{Ctrl-Q-K} & \\
Select current line & \key{Ctrl-K-L} & \\
Print selected text & \key{Ctrl-K-P} & \\
Select current word & \key{Ctrl-K-T} & \\
Delete selected text & \key{Ctrl-Del} & \key{Ctrl-K-Y} \\
Copy selected text to cursor position & \key{Ctrl-K-C} & \\
Move selected text to cursor position & \key{Ctrl-K-V} & \\
Copy selected text to clipboard & \key{Ctrl-Ins} & \\
Move selected text to the clipboard & \key{Shift-Del} & \\
Indent block one column & \key{Ctrl-K-I} & \\
Unindent block one column & \key{Ctrl-K-U} & \\
Insert text from clipboard & \key{Shift-Insert} & \\
Insert file & \key{Ctrl-K-R} & \\
Write selected text to file & \key{Ctrl-K-W} & \\
Uppercase current block & \key{Ctrl-K-N} & \\
Lowercase current block & \key{Ctrl-K-O} & \\
Uppercase word & \key{Ctrl-K-E} & \\
Lowercase word & \key{Ctrl-K-F} & \\
\end{FPCltable}

%%%%%%%%%%%%%%%%%%%%%%%%%%%%%%%%%%%%%%%%%%%%%%%%%%%%%%%%%%%%%%%%%%%%%%%%%%%%
\begin{FPCltable}{p{5cm}ll}{Change selection}{shortcutsselection}
Command & Shortcut key & Alternative \\
\hline
Mark beginning of selected text & \key{Ctrl-K-B} & \\
Mark end of selected text& \key{Ctrl-K-K} & \\
Remove selection & \key{Ctrl-K-Y} & \\
Extend selection one char to the left & \key{Shift-Arrow left} & \\
Extend selection one char to the right & \key{Shift-Arrow right} & \\
Extend selection to the beginning of the line & \key{Shift-Pos1} & \\
Extend selection to the end of the line & \key{Shift-End} & \\
Extend selection to the same column in the last row & \key{Shift-Arrow up} & \\
Extend selection to the same column in the next row & \key{Shift-Arrow down} & \\
Extend selection to the end of the line & \key{Shift-End} & \\
Extend selection one word to the left & \key{Ctrl-Shift-Arrow left} & \\
Extend selection one word to the right & \key{Ctrl-Shift-Arrow right} & \\
Extend selection one page up & \key{Shift-PageUp} & \\
Extend selection one page down & \key{Shift-PageDown} & \\
Extend selection to the beginning of the file & \key{Ctrl-Shift-Pos1} &
\key{Ctrl-Shift-PageUp} \\
Extend selection to the end of the file & \key{Ctrl-Shift-End} &
\key{Ctrl-Shift-PageUp} \\
\end{FPCltable}

%%%%%%%%%%%%%%%%%%%%%%%%%%%%%%%%%%%%%%%%%%%%%%%%%%%%%%%%%%%%%%%%%%%%%%%%%%%%
\begin{FPCltable}{p{5cm}ll}{Misc. commands}{shortcutsmisc}
Command & Shortcut key & Alternative \\
\hline
Save file & \key{F2} & \key{Ctrl-K-S} \\
Open file & \key{F3} & \\
Search & \key{Ctrl-Q-F} & \\
Search again & \key{Ctrl-L}\ & \\
Search and replace & \key{Ctrl-Q-A} & \\
Set mark & \key{Ctrl-K-n} (where n can be 0..9) & \\
Goto mark & \key{Ctrl-Q-n} (where n can be 0..9) & \\
Undo & \key{Alt-Backspace} & \\
\end{FPCltable}


%%%%%%%%%%%%%%%%%%%%%%%%%%%%%%%%%%%%%%%%%%%%%%%%%%%%%%%%%%%%%%%%%%%%%
% Porting.
%%%%%%%%%%%%%%%%%%%%%%%%%%%%%%%%%%%%%%%%%%%%%%%%%%%%%%%%%%%%%%%%%%%%%

\chapter{Porting and portable code}

\section{Free Pascal compiler modes}
The \fpc team tries to create a compiler that can compile as much as
possible code produced for \tp, \delphi{} or the Mac pascal compilers: this
should make sure that porting code that was written for one of these
compilers is as easy as possible.

At the same time, the \fpc developers have introduced a lot of extensions
in the Object Pascal language. To reconcile these different goals, and to
make sure that people can produce code which can still be compiled by
the \tp and \delphi compilers, the compiler has a concepts of 'compiler
modes'. In a certain compiler mode, the compiler has certain functionalities
switched on or off. This allows to introduce a compatibility mode in which
only features supported by the original compiler are supported. Currently,
5 modes are supported:
\begin{description}
\item[FPC] This is the original \fpc compiler mode: here all language
constructs except classes, interfaces and exceptions are supported. 
Objects are supported in this mode. This is the default mode of the
compiler.
\item[OBJFPC] This is the same mode as \var{FPC} mode, but it also includes
classes, interfaces and exceptions.
\item[TP] Turbo Pascal compatibility mode. In this mode, the compiler tries
to mimic the Turbo Pascal compiler as closely as possible. Obviously, only
32-bit or 64-bit code can be compiled.
\item[DELPHI] Delphi compatibility mode. In this mode, the compiler tries
to resemble the Delphi compiler as best as it can: All Delphi 7 features are
implemented. Features that were implemented in the .NET versions of Delphi
are {\em not} implemented.
\item[MACPAS] the Mac Pascal compatibility mode. In this mode, the compiler
attempts to allow all constructs that are implemented in Mac pascal. In
particular, it attempts to compile the universal interfaces.
\end{description}

The compiler mode can be set on a per-unit basis: each unit can have its
own compiler mode, and it is possible to use units which have been compiled
in different modes intertwined. The mode can be set in one of 2 ways:
\begin{enumerate}
\item On the command line, with the -M switch. 
\item In the source file, with the \var{\{\$MODE \}} directive. 
\end{enumerate}
Both ways take the name of the mode as an argument. If the unit or program
source file does not specify a mode, the mode specified on the command-line
is used. If the source file specifies a mode, then it overrides the mode
given on the command-line. 

Thus compiling a unit with the \var{-M} switch as follows:
\begin{verbatim}
fpc -MOBJFPC myunit
\end{verbatim}
is the same as having the following mode directive in the unit:
\begin{verbatim}
{$MODE OBJFPC}
Unit myunit;
\end{verbatim}
The \var{MODE} directive should always be located before the uses clause of the unit
interface or program uses clause, because setting the mode may result in the
loading of an additional unit as the first unit to be loaded.

Note that the \var{\{\$MODE \}} directive is a global directive, i.e. it is
valid for the whole unit; Only one directive can be specified.

The mode has no influence on the availability of units: all available
units can be used, independent of the mode that is used to compile 
the current unit or program.

\section{Turbo Pascal}
\fpc was originally designed to resemble Turbo Pascal as closely as possible. 
There are, of course, restrictions. Some of these are due to the fact that
Turbo Pascal was developed  for 16-bit architectures whereas  \fpc is
a 32-bit/64-bit compiler. Other restrictions result from the fact that \fpc works
on more than one operating system.

In general we can say that if you keep your program code close to ANSI
Pascal, you will have no problems porting from Turbo Pascal, or even Delphi, to
\fpc. To a large extent, the constructs defined by Turbo Pascal are
supported. This is even more so if you use the \var{-So} or \var{-S2}
switches.

In the following sections we will list the Turbo Pascal and Delphi 
constructs which are not supported in \fpc, and we will list in what
ways \fpc extends Turbo Pascal.

%%%%%%%%%%%%%%%%%%%%%%%%%%%%%%%%%%%%%%%%%%%%%%%%%%%%%%%%%%%%%%%%%%%%%%%
% Things that will not work
\subsection{Things that will not work}

Here we give a list of things which are defined/allowed in Turbo Pascal, but
which are not supported by \fpc. Where possible, we indicate the reason.
\begin{enumerate}
\item Duplicate case labels are permitted in Turbo Pascal, but not
in \fpc. This is actually a bug in Turbo Pascal, and so
support for it will not be implemented in Free Pascal.
\item In \tp, parameter lists of previously defined functions and 
procedures did not have to match exactly. In Free Pascal, they must. 
The reason for this is the function overloading mechanism of
\fpc. However, the \seeo{M} option overcomes this restriction.
\item The Turbo Pascal variables \var{MEM, MEMW, MEML} and \var{PORT} for memory and port
access are not available in the system unit. This is due to the operating system. Under
\dos, the extender unit (\file {GO32.PPU}) implements the mem constuct.
Under \linux, the \file{ports} unit implements such a construct for the
\var{Ports} variable.
\item Turbo Pascal allows you to create procedure and variable names
using words that are not permitted in that role in Free Pascal.
This is because there are certain words that are reserved in
Free Pascal (and Delphi) that are not reserved in Turbo Pascal, such as:
\var{PROTECTED, PUBLIC, PUBLISHED, TRY, FINALLY, EXCEPT, RAISE}.
Using the \var{-Mtp} switch will solve this problem if
you want to compile Turbo Pascal code that uses these words
(\seec{reserved} for a list of all reserved words).

\item The Turbo Pascal reserved words \var{FAR, NEAR} are ignored. 
This is because their purpose was limited to a 16-bit environment 
and \fpc is a 32-bit/64-bit compiler.
\item The Turbo Pascal \var{INTERRUPT} directive will work only on the \fpc \dos target.
Other operating systems do not allow handling of interrupts by user
programs.
\item By default the \fpc compiler uses  \var{AT\&T} assembler syntax.
This is mainly because \fpc uses \gnu \var{as}. However, other assembler
forms are available. For more information, see the \progref.
\item Turbo Pascal's Turbo Vision is available in \fpc under the name of 
FreeVision, which should be almost 100\% compatible with Turbo Vision.
\item Turbo Pascal's 'overlay' unit is not available. It also isn't necessary, since
\fpc is a 32/64-bit compiler, so program size shouldn't be an issue.
%\item Turbo Pascal has fewer reserved words than Free Pascal. 
%(see appendix \ref{ch:reserved} for a list of all reserved words.)
\item The command line parameters of the compiler are different.
\item Compiler switches and directives are mostly the same, but some extra
exist.
\item Units are not binary compatible. That means that you cannot use a
\file{.tpu} unit file, produced by Turbo Pascal, in a \fpc project.
\item The \fpc \var{TextRec} structure (for internal description of files) is not
binary compatible with TP or Delphi.
\item Sets are by default 4 bytes in Free Pascal; this means that some typecasts
which were possible in Turbo Pascal are no longer possible in Free Pascal.
However, there is a switch to set the set size, see \progref for more
information.
\item A file is opened for output only (using \var{fmOutput}) when it is
opened with \var{Rewrite}. In order to be able to read from it, it should
be reset with \var{Reset}.
\item Turbo Pascal destructors allowed parameters. This is not
permitted in Free Pascal: by default, in \fpc, Destructors cannot have parameters. 
This restriction can be removed by using the \var{-So} switch.
\item Turbo Pascal permits more than one destructor for an object. In \fpc,
there can be only one destructor. This restriction can also be removed by 
using the \var{-So} switch.
\item The order in which expressions are evaluated is not necessarily the
same. In the following expression:
\begin{verbatim}
a := g(2) + f(3);
\end{verbatim}
it is not guaranteed that \var{g(2)} will be evaluated before \var{f(3)}.
\item In \fpc, you need to use the address @ operator when assigning procedural
variables.
\end{enumerate}

%%%%%%%%%%%%%%%%%%%%%%%%%%%%%%%%%%%%%%%%%%%%%%%%%%%%%%%%%%%%%%%%%%%%%%%
% Things which are extra
\subsection{Things which are extra}
Here we give a list of things which are possible in \fpc, but which
didn't exist in Turbo Pascal or Delphi.
\begin{enumerate}
\item \fpc functions can also return complex types, such as records and arrays.
\item  In \fpc, you can use the function return value in the function itself, as a
variable. For example:
\begin{verbatim}
function a : longint;

begin
   a:=12;
   while a>4 do
     begin
        {...}
     end;
end;
\end{verbatim}
The example above would work with TP, but the compiler would assume
that the \var{a>4} is a recursive call. If a recursive call is actually what
is desired, you must append \var{()} after the function name:
\begin{verbatim}
function a : longint;

begin
   a:=12;
   { this is the recursive call }
   if a()>4 then
     begin
        {...}
     end;
end;
\end{verbatim}
\item In \fpc, there is partial support of Delphi constructs. (See the \progref for
more information on this).
\item The \fpc \var{exit} call accepts a return value for functions.
\begin{verbatim}
function a : longint;

begin
   a:=12;
   if a>4 then
     begin
        exit(a*67); {function result upon exit is a*67 }
     end;
end;
\end{verbatim}
\item \fpc supports function overloading. That is, you can define many
functions with the same name, but with different arguments. For example:
\begin{verbatim}
procedure DoSomething (a : longint);
begin
{...}
end;

procedure DoSomething (a : real);
begin
{...}
end;
\end{verbatim}
You can then call procedure \var{DoSomething} with an argument of type
\var{Longint} or \var{Real}.\\
This feature has the consequence that a previously declared function must
always be defined with the header completely the same:
\begin{verbatim}
procedure x (v : longint); forward;

{...}

procedure x;{ This will overload the previously declared x}
begin
{...}
end;
\end{verbatim}
This construction will generate a compiler error, because the compiler
didn't find a definition of \var{procedure x (v : longint);}. Instead you
should define your procedure x as:
\begin{verbatim}
procedure x (v : longint);
{ This correctly defines the previously declared x}
begin
{...}
end;
\end{verbatim}
The command line option \seeo{So}  disables overloading. When you use it, the above will
compile, as in Turbo Pascal.
\item Operator overloading. \fpc allows operator overloading, e.g. you can
define the '+' operator for matrices.
\item On FAT16 and FAT32 systems, long file names are supported.
\end{enumerate}

%%%%%%%%%%%%%%%%%%%%%%%%%%%%%%%%%%%%%%%%%%%%%%%%%%%%%%%%%%%%%%%%%%%%%%%
% Turbo Pascal compatibility mode
\subsection{Turbo Pascal compatibility mode}
When you compile a program with the \var{-Mtp} switch, the compiler will
attempt to mimic the Turbo Pascal compiler in the following ways:
\begin{itemize}
\item Assigning a procedural variable doesn't require an @ operator. One of
the differences between Turbo Pascal and \fpc is that the latter requires
you to specify an address operator when assigning a value to a procedural
variable. In Turbo Pascal compatibility mode, this is not required.
\item Procedure overloading is disabled. If procedure overloading is
disabled, the function header doesn't need to repeat the function header.

\item Forward defined procedures don't need the full parameter list when
they are defined. Due to the procedure overloading feature of \fpc, you must
always specify the parameter list of a function when you define it, even
when it was declared earlier with \var{Forward}. In Turbo Pascal
compatibility mode, there is no function overloading; hence you can omit the
parameter list:
\begin{verbatim}
Procedure a (L : Longint); Forward;

...

Procedure a ; { No need to repeat the (L : Longint) }

begin
 ...
end;

\end{verbatim}
\item Recursive function calls are handled differently. Consider the
following example:
\begin{verbatim}
Function expr : Longint;

begin
  ...
  Expr:=L:
  Writeln (Expr);
  ...
end;
\end{verbatim}
In Turbo Pascal compatibility mode, the function will be called recursively
when the \var{writeln} statement is processed. In \fpc, the function result
will be printed. In order to call the function recursively under \fpc, you
need to implement it as follows :
\begin{verbatim}
Function expr : Longint;

begin
  ...
  Expr:=L:
  Writeln (Expr());
  ...
end;
\end{verbatim}
\item Any text after the final \var{End.} statement is ignored. Normally,
this text is processed too.
\item You cannot assign procedural variables to untyped pointers; so the
following is invalid:
\begin{verbatim}
 a: Procedure;
 b: Pointer;
begin
 b := a; // Error will be generated.
\end{verbatim}
\item The @ operator is typed when applied on procedures.
\item You cannot nest comments.
\end{itemize}

\begin{remark}
The \var{MemAvail} and \var{MaxAvail} functions are no longer available in
\fpc as of version 2.0. The reason for this incompatibility follows:

On modern operating systems, \footnote{The DOS extender GO32V2 falls under this 
definition of "modern" because it can use paged memory and run in 
multitasked environments.} the idea of "Available Free Memory" is not valid for an
application.
The reasons are:
\begin{enumerate} 
\item One processor cycle after an application asked the OS how much memory is free,
another application may have allocated everything.
\item It is not clear what "free memory" means: does it include swap memory,
does it include disk cache memory (the disk cache can grow and shrink on
modern OS'es), does it include memory allocated to other applications but
which can be swapped out, etc.
\end{enumerate}

Therefore, programs using \var{MemAvail} and \var{MaxAvail} functions 
should be rewritten so they no longer use these functions, because
it does not make sense any more on modern OS'es. There are 3 possibilities:
\begin{enumerate}
\item Use exceptions to catch out-of-memory errors.
\item Set the global variable "ReturnNilIfGrowHeapFails" to \var{True}
and check after each allocation whether the pointer is different from
\var{Nil}.
\item Don't care and declare a dummy function called \var{MaxAvail} 
which always returns \var{High(LongInt)} (or some other constant).
\end{enumerate}

\end{remark}

%%%%%%%%%%%%%%%%%%%%%%%%%%%%%%%%%%%%%%%%%%%%%%%%%%%%%%%%%%%%%%%%%%%%%%%
% A note about long file names.
\subsection{A note on long file names under \dos}
Under \windows 95 and higher, long filenames are supported. Compiling
for the \windows target ensures that long filenames are supported in all
functions that do file or disk access in any way.

Moreover, \fpc supports the use of long filenames in the system unit and
the \file{Dos} unit also for go32v2 executables. The system unit contains the
boolean variable \var{LFNsupport}. If it is set to \var{True} then all
system unit functions and \file{Dos} unit functions will use long file names
if they are available. This should be so on \windows 95 and 98, but
not on \windows NT or \windows 2000. The system unit will check this 
by calling \dos function \var{71A0h} and checking whether long filenames 
are supported on the \file{C:} drive.

It is possible to disable the long filename support by setting the
\var{LFNSupport} variable to \var{False}; but in general it is recommended
to compile programs that need long filenames as native \windows applications.

\section{Porting Delphi code}

Porting Delphi code should be quite painless. The \var{Delphi} mode of the 
compiler tries to mimic Delphi as closely as possible. 
This mode can be enabled using the \var{-Mdelphi} command line switch, 
or by inserting the following code in the sources before the \var{unit} 
or \var{program} clause:
\begin{verbatim}
{$IFDEF FPC}
{$MODE DELPHI}
{$ENDIF FPC}
\end{verbatim}
This ensures that the code will still compile with both Delphi and FPC.

Nevertheless, there are some things that will not work. 
Delphi compatibility is relatively complete up to Delphi 7. 
New constructs in higher versions of Delphi 
(notably, the versions that work with .NET) are not supported.

\subsection{Missing language constructs}
At the level of language compatibility, FPC is very compatible with Delphi:
it can compile most of FreeCLX, the free Widget library that was shipped 
with Delphi 6, Delphi 7 and Kylix.

Currently, the only missing language constructs are:
\begin{enumerate}
\item \var{Dynamic} methods are actually the same as \var{virtual}.
\item \var{Const} for a parameter to a procedure does not necessarily 
mean that the variable or value is passed by reference.
\item Packages are not supported.
\end{enumerate}

There are some inline assembler constructs which are not supported, 
and since \fpc is designed to be platform independent, it is quite 
unlikely that these constructs will be supported in the future.

Note that the \var{-Mobjfpc} mode switch is to a large degree Delphi 
compatible, but is more strict than Delphi. The most notable differences
are:
\begin{enumerate}
\item Parameters or local variables of methods cannot have the same 
names as properties of the class in which they are implemented.
\item The address operator is needed when assigning procedural variables (or event handlers).
\item AnsiStrings are not switched on by default.
\end{enumerate}

\subsection{Missing calls / API incompatibilities}
Delphi is heavily bound to Windows. Because of this, it introduced a lot of 
Windows-isms in the API (e.g. file searching and opening, loading libraries).

\fpc was designed to be portable, so things that are very Windows 
specific are missing, although the \fpc team tries to minimize this.
The following are the main points that should be considered:

\begin{itemize}
\item By default, \fpc generates console applications. This means that
you must explicitly enable the GUI application type for Windows:
\begin{verbatim}
{$APPTYPE GUI}
\end{verbatim}
\item The \file{Windows} unit provides access to most of the core 
Win32 API. Some calls may have different parameter lists: instead 
of declaring a parameter as passed by reference (var), a pointer 
is used (as in C). For most cases, \fpc provides overloaded versions 
of such calls.
\item Widestrings. Widestring management is not automatic in \fpc, 
since various platforms have different ways of dealing with widestring 
encodings and Multi-Byte Character Sets. 
FPC supports Widestrings, but may not use the same encoding as on Windows.

Note that in order to have correct widestring management, you need to
include the \file{cwstring} unit on Unix/\linux platforms: This unit
initializes the widestring manager with the necessary callbacks which
use the C library to implement all needed widestring functionality.
\item Threads: At this moment, \fpc does not offer native thread 
management on all platforms; on Unix, linking to the C library is 
needed to provide thread management in an FPC application. 
This means that a \file{cthreads} unit must be included to enable 
threads.
\item A much-quoted example is the \var{SetLastOSError} call. 
This is not supported, and will never be supported.
\item Filename Case sensitivity: Pascal is a case-insensitive language,
so the uses clause should also be case insensitive. Free Pascal ensures
case insensitive filenames by also searching for a lowercase version 
of the file. Kylix does not do this, so this could create problems
if two differently cased versions of the same filename are in the path.
\item RTTI is NOT stored in the same way as for Delphi. The format is mostly compatible, but may differ. This should not be a problem if the API of the TypeInfo unit 
is used and no direct access to the RTTI information is attempted.
\item By default, sets are of different size than in Delphi, but set size
can be specified using directives or command line switches.
\item Likewise, by default enumeration types are of different size than in
Delphi. Here again, the size can be specified using directives or command
line switches.
\item In general, one should not make assumptions about the internal
structure of complex types such as records, objects, classes and their
associated structure. For example, the VMT table layout is different, the
alignment of fields in a record may be different, etc. 
\item The same is true for basic types: on other processors the high and low
bytes of a word or integer may not be at the same location as on an Intel
processor (the endianness is different).
\item Names of local variables and method arguments are not allowed to 
match the name of a property or field of the class: this is bad practise,
as there can be confusion as to which of the two is meant.
\end{itemize}

\subsection{Delphi compatibility mode}
Switching on Dephi compatibility mode has the following effect:
\begin{enumerate}
\item Support for Classes, exceptions and threadvars is enabled.
\item The \file{objpas} is loaded as the first unit. This unit redefines
some basic types: \var{Integer} is 32-bit for instance.
\item The address operator (@) is no longer needed to set event handlers
(i.e. assign to procedural variables or properties).
\item Names of local variables and method parameters in classes can match 
the name of properties or field of the class.
\item The \var{String} keyword implies \var{AnsiString} by default. 
\item Operator overloading is switched off.
\end{enumerate}

\subsection{Best practices for porting}

When encountering differences in Delphi/FPC calls, the best thing to 
do is not to insert IFDEF statements whenever a difference is 
encountered, but to create a separate unit which is only used 
when compiling with FPC. The missing/incompatible calls can 
then be implemented in that unit. This will keep the code more 
readable and easier to maintain.

If a language construct difference is found, then the \fpc team 
should be contacted and a bug should be reported.

%%%%%%%%%%%%%%%%%%%%%%%%%%%%%%%%%%%%%%%%%%%%%%%%%%%%%%%%%%%%%%%%%%%%%
% writing portable code
%%%%%%%%%%%%%%%%%%%%%%%%%%%%%%%%%%%%%%%%%%%%%%%%%%%%%%%%%%%%%%%%%%%%%

\section{Writing portable code}

\fpc is designed to be cross-platform. This means that the basic RTL 
units are usable on all platforms, and the compiler behaves the same 
on all platforms (as far as possible).  The Object Pascal 
language is the same on all platforms. Nevertheless, FPC comes with a
lot of units that are not portable, but provide access to all 
possibilities that a platform provides.

The following are some guidelines to consider when writing portable code:
\begin{itemize}
\item Avoid system-specific units. The system unit, the objects and 
classes units and the SysUtils unit are guaranteed to work on all 
systems. So is the DOS unit, but that is deprecated. 
\item Avoid direct hardware access. Limited, console-like hardware 
access is available for most platforms in the Video, Mouse and  
Keyboard units.
\item Do not use hard-coded filename conventions. 
See below for more information on this.
\item Make no assumptions on the internal representation of types. Various
processors store information in different ways ('endianness').
\item If system-specific functionality is needed, it is best to 
separate this out in a single unit. Porting efforts will then be 
limited to re-implementing this unit for the new platform.
\item Don't use assembler, unless you have to. Assembler is processor
specific. Some instructions will not work even on the same processor family.
\item Do not assume that pointers and integers have the same size. They do
on an Intel 32-bit processor, but not necessarily on other processors.
The \var{PtrInt} type is an alias for the integer type that has the same
size as a pointer. \var{SizeInt} is used for all size-related issues.
\end{itemize}

The system unit contains some constants which describe file access on a system:
\begin{description}
\item[AllFilesMask] a file mask that will return all files in a directory.
This is \var{*} on Unix-like platforms, and \var{*.*} on dos and windows
like platforms.
\item[LineEnding] A character or string which describes the end-of-line marker 
used on the current platform. Commonly, this is one of \#10, \#13\#10 or \#13.
\item[LFNSupport] A boolean that indicates whether the system supports long filenames
(i.e. is not limited to MS-DOS 8.3 filenames).
\item[DirectorySeparator] The character which acts as a separator between directory parts of a path.
\item[DriveSeparator] For systems that support drive letters, this is the character that
is used to separate the drive indication from the path.
\item[PathSeparator] The character used to separate items in a list (notably, a PATH).
\item[maxExitCode] The maximum value for a process exitcode.
\item[MaxPathLen] The maximum length of a filename, including a path.
\item[FileNameCaseSensitive] A boolean that indicates whether filenames are handled case sensitively.
\item[UnusedHandle] A value used to indicate an unused/invalid file handle.
\item[StdInputHandle] The value of the standard input file handle. 
This is not always 0 (zero), as is commonly the case on Unices.
\item[StdOutputHandle] The value of the standard output file handle. 
This is not always 1, as is commonly the case on Unices.
\item[StdErrorHandle] The value of the standard diagnostics output file handle. 
This is not always 2, as is commonly the case on Unices.
\item[CtrlZMarksEOF] A boolean that indicates whether the \#26 character marks the end of a file
(an old MS-DOS convention).
\end{description}

To ease writing portable filesystem code, the Free Pascal file routines in
the system unit and \file{sysutils} unit treat the common directory separator 
characters (/ and $\backslash$) as equivalent. That means that if you use / on a \windows 
system, it will be transformed to a backslash, and vice versa. 

This feature is controlled by 2 (pre-initialized) variables in the system unit:
\begin{description}
\item[AllowDirectorySeparators] A set of characters which, when used in
filenames, are treated as directory separators. They are transformed to
the \var{DirectorySeparator} character.
\item[AllowDriveSeparators] A set of characters which, when used in
filenames, are treated as drive separator characters. They are transformed
to the \var{DriveSeparator} character.
\end{description}

%%%%%%%%%%%%%%%%%%%%%%%%%%%%%%%%%%%%%%%%%%%%%%%%%%%%%%%%%%%%%%%%%%%%%
% Utilities.
%%%%%%%%%%%%%%%%%%%%%%%%%%%%%%%%%%%%%%%%%%%%%%%%%%%%%%%%%%%%%%%%%%%%%

\chapter{Utilities that come with Free Pascal}
\label{ch:Utilities}
Besides the compiler and the runtime Library, \fpc comes with some utility
programs and units. Here we list these programs and units.

%%%%%%%%%%%%%%%%%%%%%%%%%%%%%%%%%%%%%%%%%%%%%%%%%%%%%%%%%%%%%%%%%%%%%%%
% Demo programs and examples.
\section{Demo programs and examples}
A suite of demonstration programs comes included with the Free
Pascal distribution.
These programs have no other purpose than to demonstrate the capabilities of
\fpc. They are located in the \file{demo} directory of the sources.

All example programs mentioned in the documentation are available. Check out the
directories that are beneath the same directory as the demo directory.
The names of these directories end on \file{ex}.
There you will find all example sources.

\section{fpcmake}

\file{fpcmake} is the \fpc makefile constructor program.

It reads a \file{Makefile.fpc} configuration file and converts it to a
\file{Makefile} suitable for reading by GNU \file{make} to compile
your projects. It is similar in functionality to GNU \file{autoconf}
or \file{Imake} for making X projects.

\file{fpcmake} accepts filenames of makefile description files as its
command line arguments. For each of these files it will create a
\file{Makefile} in the same directory where the file is located,
overwriting any other existing file.

If no options are given, it just attempts to read the file \file{Makefile.fpc}
in the current directory and tries to construct a makefile from it.
Any previously existing \file{Makefile} will be erased.

The format of the \file{fpcmake} configuration file is described in great
detail in the appendices of the \progref.

\section{fpdoc - Pascal Unit documenter}

\file{fpdoc} is a program which generates fully cross-referenced
documentation for a unit. It generates documentation for each 
identifier found in the unit's interface section. The generated 
documentation can be in many formats, for instance HTML, RTF, Text, man
page and LaTeX. 
Unlike other documentation tools, the documentation can be in a separate
file (in XML format), so the sources aren't cluttered with documentation.
Its companion program \file{makeskel} creates an empty XML file with
entries for all identifiers, or it can update an existing XML file, 
adding entries for new identifiers.

\file{fpdoc} and \file{makeskel} are described in the \fpdocref.

\section{h2pas - C header to Pascal Unit converter}
\file{h2pas} attempts to convert a C header file to a Pascal unit. 
it can handle most C constructs that one finds in a C header file,
and attempts to translate them to their Pascal counterparts. 

See below (constructs) for a full description of what the translator can handle.
The unit with Pascal declarations can then be used to access code written in C.

The output of the h2pas program is written to a file with the same name as
the C header file that was used as input, but with the extension \file{.pp}
The output file that h2pas creates can be customized in a number of ways by
means of many options.

\subsection{Options}
The output of  \file{h2pas} can be controlled with the following options:

\begin{description}
\item[-d] Use \var{external;} for all procedure and function declarations.
\item[-D] Use \var{external libname name 'func\_name'} for function and 
procedure declarations.
\item[-e] Emit a series of constants instead of an enumeration type for the 
C \var{enum} construct. 
\item[-i] Create an include file instead of a unit (omits the unit header).
\item[-l] \textbf{libname} specify the library name for external function 
declarations.
\item[-o] \textbf{outfile} Specify the output file name. Default is the input file name with 
the extension replaced by \file{.pp}
\item[-p] Use the letter \var{P} in front of pointer type parameters instead of \^.
\item[-s] Strip comments from the input file. By default comments are converted
to comments, but they may be displaced, since a comment is handled by the
scanner.
\item[-t] Prepend typedef type names with the letter \var{T} (used to follow 
Borland's convention that all types should be defined with T).
\item[-v] Replace pointer parameters with call by reference parameters.
Use with care because some calls can expect a \var{Nil} pointer.
\item[-w] Header file is a win32 header file (adds support for some special macros).
\item[-x] Handle SYS\_TRAP of the PalmOS header files.
\end{description}

\subsection{Constructs}
The following C declarations and statements are recognized:

\begin{description}
\item[defines] 
Defines are changed into Pascal constants if they are simple defines.
Macros are changed - wherever possible - to functions; however the arguments
are all integers, so these must be changed manually. Simple expressions 
in define staments are recognized, as are most arithmetic operators: 
addition, substraction, multiplication, division, logical operators, 
comparison operators, shift operators. The C construct ( A ? B : C)
is also recognized and translated to a Pascal construct with an IF
statement. (This is buggy, however).

\item[preprocessor statements]
The conditional preprocessing commands are recognized and translated into
equivalent Pascal compiler directives. The special 
\begin{verbatim}
#ifdef __cplusplus
\end{verbatim}
is also recognized and removed.
\item[typedef] A typedef statement is changed into a Pascal type statement. 
The following basic types are recognized:
\begin{itemize}
\item \var{char} changed to \var{char}.
\item \var{float} changed to \var{real} (=double in \fpc).
\item \var{int} changed to \var{longint}.
\item \var{long} changed to \var{longint}.
\item \var{long int} changed to \var{longint}.
\item \var{short} changed to \var{integer}. 
\item \var{unsigned} changed to \var{cardinal}.
\item \var{unsigned char} changed to \var{byte}.
\item \var{unsigned int} changed to \var{cardinal}.
\item \var{unsigned long int} changed to \var{cardinal}.
\item \var{unsigned short} changed to \var{word}.
\item \var{void} ignored.
\end{itemize}
These types are also changed if they appear in the arguments of a function
or procedure.
\item[functions and procedures]
Functions and procedures are translated as well. Pointer types may be
changed to call by reference arguments (using the \var{var} argument) 
by using the \var{-p} command line argument. Functions that have a 
variable number of arguments are changed to a function with a \var{cvar}
modifier. (This used to be the \var{array of const} argument.)
\item[specifiers]
The \var{extern} specifier is recognized; however it is ignored. 
The \var{packed} specifier is also recognised and changed with the
\var{PACKRECORDS} directive. The \var{const} specifier is also 
recognized, but is ignored.
\item[modifiers] 
If the \var{-w} option is specified, then the following modifiers are recognized:
\begin{verbatim}
STDCALL
CDECL
CALLBACK
PASCAL
WINAPI
APIENTRY
WINGDIAPI
\end{verbatim}
as defined in the win32 headers. If additionally the \var{-x}
option is specified then the  
\begin{verbatim}
SYS_TRAP
\end{verbatim}
specifier is also recognized.
\item[enums]
Enum constructs are changed into enumeration types. Bear in mind that, in C,
enumeration types can have values assigned to them. Free Pascal also allows
this to a certain degree. If you know that values are assigned to enums, it
is best to use the \var{-e} option to change the enumerations to a series of 
integer  constants.

\item[unions] Unions are changed to variant records. 
\item[structs] Structs are changed to Pascal records, with C packing.
\end{description}

\section{h2paspp - preprocessor for h2pas}
\var{h2paspp} can be used as a simple preprocessor for \file{h2pas}. It
removes some of the constructs that h2pas has difficulties with. 
\file{h2paspp} reads one or more C header files and preprocesses them, writing the result 
to files with the same name as the originals as it goes along. 
It does not accept all preprocesser tokens of C, but takes care of the following 
preprocessor directives:

\begin{description}
\item [\#define symbol] Defines the new symbol \var{symbol}. Note that macros are not supported.
\item [\#if symbol] The text following this directive is included if \var{symbol} is defined.
\item [\#ifdef symbol] The text following this directive is included if \var{symbol} is defined. 
\item [\#ifndef symbol] The text following this directive is included if \var{symbol} is not defined.
\item [\#include filename] Include directives are removed, unless the \var{-I} option was given, 
in which case the include file is included and written to the output file.
\item[\#undef symbol] The symbol \var{symbol} is undefined.
\end{description}

\subsection{Usage}
\file{h2paspp} accepts one or more filenames and preprocesses them. 
It will read the input, and write the output to a file with the same name 
unless the \var{-o} option is given, in which case the file is written 
to the specified file. Note that only one output filename can be given.


\subsection{Options}
\file{h2paspp} has a small number of options to control its behaviour:
\begin{description}
\item[-dsymbol] Define the symbol \var{symbol} before processing is started.
\item[-h] Emit a small helptext.
\item[-I] Include include files instead of dropping the include statement.
\item[-ooutfile] If this option is given, the output will be written to a 
file named \file{outfile}. Note that only one output file can be given.
\end{description}

\section{ppudump program}

\file{ppudump} is a program which shows the contents of a \fpc unit. It
is distributed with the compiler. You can just issue the following command
\begin{verbatim}
  ppudump [options] foo.ppu
\end{verbatim}
to display the contents of the \file{foo.ppu} unit. You can specify multiple
files on the command line.

The options can be used to change the verbosity of the display. By default,
all available information is displayed.
You can set the verbosity level using the \var{-Vxxx} option.
Here, \var{xxx} is a combination of the following
letters:
\begin{description}
\item [h:\ ] Show header info.
\item [i:\ ] Show interface information.
\item [m:\ ] Show implementation information.
\item [d:\ ] Show only (interface) definitions.
\item [s:\ ] Show only (interface) symbols.
\item [b:\ ] Show browser info.
\item [a:\ ] Show everything (default if no -V option is present).
\end{description}


\section{ppumove program}
\label{se:ppumove}

\file{ppumove} is a program to make shared or static libraries from
multiple units. It can be compared with the \file{tpumove} program that
comes with Turbo Pascal.

It is distributed in binary form along with the compiler.

Its usage is very simple:
\begin{verbatim}
ppumove [options] unit1.ppu unit2.ppu ... unitn.ppu
\end{verbatim}
where \var{options} is a combination of:
\begin{description}
\item[-b:\ ] Generate a batch file that will
contain the external linking and archiving commands that must be
executed. The name of this batch file is \file{pmove.sh} on \linux (and Unix
like OSes), and \file{pmove.bat} on \windows and \dos.
\item[-d xxx:\ ] Set the directory in which to place the output files to
\file{xxx}.
\item[-e xxx:\ ] Set the extension of the moved unit files to \file{xxx}.
By default, this is \file{.ppl}. You don't have to specify the dot.
\item[-o xxx:\ ] Set the name of the output file, i.e. the name of the file
containing all the units. This parameter is mandatory when you use multiple
files. On \linux, \file{ppumove} will prepend this name with \file{lib} if it isn't
already there, and will add an extension appropriate to the type of library.
\item [-q:\ ] Operate silently.
\item [-s:\ ] Make a static library instead of a
dynamic one; By default a dynamic library is made on \linux.
\item [-w:\ ] Tell \file{ppumove} that it is working under \windowsnt. This will
change the names of the linker and archiving program to \file{ldw} and
\file{arw}, respectively.
\item[-h or -?:\ ] Display a short help.
\end{description}

The action of the \file{ppumove} program is as follows:
It takes each of the unit files, and modifies it so that the compiler will
know that it should look for the unit code in the library. The new unit
files will have an extension \file{.ppu}; this can be changed with the
\var{-e} option. It will then put together all the object files of the units
into one library, static or dynamic, depending on the presence of the
\var{-s} option.

The name of this library must be set with the \var{-o} option.
If needed, the prefix \file{lib} will be prepended under \linux.
The extension will be set to \file{.a} for static libraries,
for shared libraries, the extensions are \var{.so} on linux, and \var{.dll}
under \windowsnt and \ostwo.

As an example, the following command
\begin{verbatim}
./ppumove -o both -e ppl ppu.ppu timer.ppu
\end{verbatim}
will generate the following output under \linux{}:
\begin{verbatim}
PPU-Mover Version 0.99.7
Copyright (c) 1998 by the Free Pascal Development Team

Processing ppu.ppu... Done.
Processing timer.ppu... Done.
Linking timer.o ppu.o
Done.
\end{verbatim}
And it will produce the following files:
\begin{enumerate}
\item \file{libboth.so} : The shared library containing the code from
\file{ppu.o} and \file{timer.o}. Under \windowsnt, this file would be called
\file{both.dll}.
\item \file{timer.ppl} : The unit file that tells the \fpc compiler to look
for the timer code in the library.
\item \file{ppu.ppl} : The unit file that tells the \fpc compiler to look
for the ppu code in the library.
\end{enumerate}
You could then use or distribute the files \file{libboth.so}, \file{timer.ppl}
and \file{ppu.ppl}.

\section{ptop - Pascal source beautifier}

\subsection{ptop program}
% This section was supplied by Marco Van de voort, for which my thanks.
% I did some cleaning, and added the subsubsection with help on on the
% object. MVC.

\file{ptop} is a source beautifier written by Peter Grogono based on the ancient pretty-printer
by Ledgard, Hueras, and Singer, modernized by the \fpc team (objects, streams, configurability
etc).

This configurability, and the thorough bottom-up design are the advantages of this program over
the diverse Turbo Pascal source beautifiers on e.g. SIMTEL.

The program is quite simple to operate:

ptop "[-v] [-i indent] [-b bufsize ][-c \file{optsfile}] \file{infile} \file{outfile}"

The \file{infile} parameter is the Pascal file to be processed, and will be written
to \file{outfile}, overwriting an existing \file{outfile} if it exists.

Some options modify the behaviour of ptop:

\begin{description}
\item[-h] Write an overview of the possible parameters and command line syntax.
\item[-c \file{ptop.cfg}] Read some configuration data from configuration file instead of using
  the internal defaults then. A config file is not required, the program can
  operate without one. See also -g.
\item[-i ident] Set the number of indent spaces used for BEGIN END; and other blocks.
\item[-b bufsize] Set the streaming buffersize to bufsize. The default is 255; 
0 is considered non-valid and ignored.
\item[-v] Be verbose. Currently only outputs the number of lines read/written and some error messages.
\item[-g \file{ptop.cfg}] Write \file{ptop} configuration defaults to the file
"ptop.cfg". The contents of this file can be changed to your liking, and it
can be used with the -c option.
\end{description}

\subsection{The ptop configuration file}

Creating and distributing a configuration file for ptop is not necessary,
unless you want to modify the standard behaviour of \file{ptop}. The configuration
file is never preloaded, so if you want to use it you should always specify
it with a \var{-c ptop.cfg} parameter.

The structure of a ptop configuration file is a simple building block repeated
several (20-30) times, for each Pascal keyword known to the \file{ptop} program.
(See the default configuration file or \file{ptopu.pp} source to
find out which keywords are known).

The basic building block of the configuration file consists of one or two
lines, describing how \file{ptop} should react on a certain keyword.
First comes a line without square brackets with the following format:

keyword=option1,option2,option3,...

If one of the options is "dindonkey" (see further below), a second line
- with square brackets - is needed:

[keyword]=otherkeyword1,otherkeyword2,otherkeyword3,...

As you can see the block contains two types of identifiers: keywords 
(keyword and otherkeyword1..3 in above example) and options, (option1..3 above).

\var{Keywords} are the built-in valid Pascal structure-identifiers like BEGIN, END, CASE, IF,
THEN, ELSE, IMPLEMENTATION. The default configuration file lists most of these.

Besides the real Pascal keywords, some other codewords are used for operators
and comment expressions as in \seet{keywords}.

\begin{FPCltable}{lll}{Keywords for operators}{keywords}
Name of codeword       &     Operator \\  \hline
casevar                &     : in a case label ( unequal 'colon') \\
becomes                &     := \\
delphicomment          &     // \\
opencomment            &       \{ or (* \\
closecomment           &     \} or *) \\
semicolon              &     ; \\
colon                  &     : \\
equals                 &     = \\
openparen              &     [ \\
closeparen             &     ] \\
period                 &     . \\
\end{FPCltable}

The \textbf{options} codewords define actions to be taken when the keyword before
the equal sign is found, as listed in \seet{ptopoptions}.

\begin{FPCltable}{lll}{Possible options}{ptopoptions}
Option         &     does what \\ \hline
crsupp         &     Suppress CR before the keyword.\\
crbefore       &     Force CR before keyword.\\
               &     (do not use with crsupp.)\\
blinbefore     &     Blank line before keyword.\\
dindonkey      &     De-indent on associated keywords.\\
               &     (see below)\\
dindent        &     Deindent (always)\\
spbef          &     Space before\\
spaft          &     Space after\\
gobsym         &     Print symbols which follow a\\
               &     keyword but which do not\\
               &     affect layout. prints until\\
               &     terminators occur.\\
               &     (terminators are hard-coded in pptop,\\
               &     still needs changing)\\
inbytab        &     Indent by tab.\\
crafter        &     Force CR after keyword.\\
upper          &     Prints keyword all uppercase\\
lower          &     Prints keyword all lowercase\\
capital        &     Capitalizes keyword: 1st letter\\
               &     uppercase, rest lowercase.\\
\end{FPCltable}

The option "dindonkey" given in table \seet{ptopoptions} requires some
further explanation. "dindonkey" is a contraction of "DeINDent ON
associated KEYword". When it is present as an option in the first
line, then a second, square-bracketed, line is required. 
A de-indent will be performed when any of the other keywords listed
in the second line are encountered in the source.

Example: The lines
\begin{verbatim}
else=crbefore,dindonkey,inbytab,upper
[else]=if,then,else
\end{verbatim}

mean the following:

\begin{itemize}
\item The keyword this block is about is \textbf{else} because it's on the LEFT side
of both equal signs.
\item The option \var{crbefore} signals not to allow other code (so just spaces)
before the ELSE keyword on the same line.
\item The option \var{dindonkey} de-indents if the parser finds any of the keywords
 in the square brackets line (if,then,else).
\item The option \var{inbytab} means indent by a tab.
\item The option \var{upper} uppercase the keyword (else or Else becomes ELSE)
\end{itemize}

Try to play with the configfile step by step until you find the effect you desire.
The configurability and possibilities of ptop are quite large. E.g. I like all
keywords uppercased instead of capitalized, so I replaced all capital keywords in
the default file by upper.

\file{ptop} is still development software. So it is wise to visually check the generated
source and try to compile it, to see if \file{ptop} hasn't introduced any errors.

\subsection{ptopu unit}

The source of the \file{PtoP} program is conveniently split in two files:
one is a unit containing an object that does the actual beautifying of the
source, the other is a shell built around this object so it can be used
from the command line. This design makes it possible to include the object
in a program (e.g. an IDE) and use its features to format code.

The object resides in the \file{PtoPU} unit, and is declared as follows
\begin{verbatim}
  TPrettyPrinter=Object(TObject)
      Indent : Integer;    { How many characters to indent ? }
      InS    : PStream;
      OutS   : PStream;
      DiagS  : PStream;
      CfgS : PStream;
      Constructor Create;
      Function PrettyPrint : Boolean;
    end;
\end{verbatim}

Using this object is very simple. The procedure is as follows:
\begin{enumerate}
\item Create the object, using its constructor.
\item Set the \var{InS} stream. This is an open stream, from which Pascal source will be
read. This is a mandatory step.
\item Set the \var{OutS} stream. This is an open stream, to which the
beautified Pascal source will be written. This is a mandatory step.
\item Set the \var{DiagS} stream. Any diagnostics will be written to this
stream. This step is optional. If you don't set this, no diagnostics are
written.
\item Set the \var{CfgS} stream. A configuration is read from this stream.
(see the previous section for more information about configuration). This
step is optional. If you don't set this, a default configuration is used.
\item Set the \var{Indent} variable. This is the number of spaces to use
when indenting. Tab characters are not used in the program. This step is
optional. The indent variable is initialized to 2.
\item Call \var{PrettyPrint}. This will pretty-print the source in \var{InS}
and write the result to \var{OutS}. The function returns \var{True} if no
errors occurred, \var{False} otherwise.
\end{enumerate}

So, a minimal procedure would be:
\begin{verbatim}
Procedure CleanUpCode;

var
  Ins,OutS : PBufStream;
  PPRinter : TPrettyPrinter;

begin
  Ins:=New(PBufStream,Init('ugly.pp',StopenRead,TheBufSize));
  OutS:=New(PBufStream,Init('beauty.pp',StCreate,TheBufSize));
  PPrinter.Create;
  PPrinter.Ins:=Ins;
  PPrinter.outS:=OutS;
  PPrinter.PrettyPrint;
end;
\end{verbatim}

Using memory streams allows very fast formatting of code, and is
particularly suitable for editors.

\section{rstconv program}

The \file{rstconv} program converts the resource string files generated by
the compiler (when you use resource string sections) to \file{.po} files
that can be understood by the GNU \file{msgfmt} program.

Its usage is very easy; it accepts the following options:
\begin{description}
\item[-i file] Use the specified file instead of stdin as input file. This
option is optional.
\item[-o file] Write output to the specified file. This option is required.
\item[-f format] Specify the output format. At the moment, only one output
format is supported: {\em po} for GNU gettext \file{.po} format.
It is the default format.
\end{description}
As an example:
\begin{verbatim}
rstconv -i resdemo.rst -o resdemo.po
\end{verbatim}
will convert the \file{resdemo.rst} file to \file{resdemo.po}.

More information on the \file{rstconv} utility can be found in the \progref,
under the chapter about resource strings.

\section{unitdiff program}

\subsection{Synopsis}
\file{unitdiff} shows the differences between two unit interface sections. 
\begin{verbatim}
unitdiff [--disable-arguments] [--disable-private] [--disable-protected] 
[--help] [--lang=language] [--list] [--output=filename] [--sparse] 
file1 file2
\end{verbatim}

\subsection{Description and usage}

\file{Unitdiff} scans one or two Free Pascal unit source files and either lists all
available identifiers, or describes the differences in identifiers
between the two units.

You can invoke \file{unitdiff} with an input filename as the only required
argument. It will then simply list all available identifiers.

The regular usage is to invoke \file{unitdiff} with two arguments:
\begin{verbatim}
unitdiff input1 input2
\end{verbatim}
Invoked like this, it will show the difference in interface between the two
units, or list the available identifiers in both units. The output of 
\file{unitdiff} will go to standard output by default.

\subsection{Options}
Most of the \file{unitdiff} options are not required. Defaults will be used in most
cases.

\begin{description}
\item[--disable-arguments] Do not check the arguments of functions 
and procedures. The default action is to check them.
\item[--disable-private] Do not check private fields or methods of
classes. The default action is to check them.
\item[--disable-protected] Do not check protected fields or methods of
classes. The default action is to check them.
\item[--help] Emit a short help text and exit.
\item[--lang=language] Set the language for the output file. This will mainly 
set the strings used for the headers in various parts of the documentation files 
(by default they're in English). Currently, valid options are:
\begin{itemize}
\item \var{de}: German.
\item \var{fr}: French.
\item \var{nl}: Dutch.
\end{itemize}
\item[--list] Display just the list of available identifiers for the unit
or units. If only one unit is specified on the command line, this option 
is automatically assumed.
\item[--output=filename] Specify where the output should go. The default
action is to send the output is sent to standard output (the screen).
\item[--sparse] Turn on sparse mode. Output only the identifier names.
Do not output types or type descriptions. By default, type descriptions 
are also written.
\end{description}

%%%%%%%%%%%%%%%%%%%%%%%%%%%%%%%%%%%%%%%%%%%%%%%%%%%%%%%%%%%%%%%%%%%%%%%
% Supplied units
%%%%%%%%%%%%%%%%%%%%%%%%%%%%%%%%%%%%%%%%%%%%%%%%%%%%%%%%%%%%%%%%%%%%%%
\chapter{Units that come with Free Pascal}
\label{ch:Units}

Here we list the units that come with the \fpc distribution. Since there is
a difference in the supplied units per operating system, we first describe
the generic ones, then describe those which are operating system specific. 

%
% Common units
%
\section{Standard units}

The following units are standard and are meant to be ported to
all platforms supported by \fpc. A brief description of each unit
is also given.

\begin{description}
\item[charset] A unit to provide mapping of character sets.
\item[cmem] Using this unit replaces the Free Pascal memory manager with the 
memory manager of the C library.
\item[crt] This unit is similar to the unit of the same name of
Turbo Pascal. It implements writing to the console in color,  moving the 
text cursor around and reading from the keyboard.
\item[dos] This unit provides basic routines for accessing the operating
system. This includes file searching, environment variables access,
getting the operating system version, getting and setting the
system time. It is to note that some of these routines are duplicated
in functionality in the \var{sysutils} unit.
\item[dynlibs] Provides cross-platform access to loading dynamical libraries.
%\item[errors] returns 
\item[getopts] This unit gives you the \gnu \var{getopts} command line
arguments handling mechanism. It also supports long options.
\item[graph] \emph{This unit is deprecated}. This unit provides basic graphics handling, with routines to
draw lines on the screen, display text etc. It provides the same functions
as the Turbo Pascal unit. 
\item[heaptrc] a unit which debugs the heap usage. When the program exits, it outputs a summary of the used memory, and dumps a summary of unreleased memory blocks (if any).
\item[keyboard] provides basic keyboard handling routines in a platform independent way,
and supports writing custom drivers.
\item[macpas] This unit implements several functions available only in MACPAS mode. 
This unit should not be included; it's automatically included when the MACPAS mode is used.
\item[math] This unit contains common mathematical routines (trigonometric
functions, logarithms, etc.) as well as more complex ones (summations of arrays,
normalization functions, etc.).
\item[matrix] A unit providing matrix manipulation routines.
\item[mmx] This unit provides support for \var{mmx} extensions in your
code. 
\item[mouse] Provides basic mouse handling routines in a platform independent way,
and supports writing custom drivers. 
\item [objects]  This unit provides the base object for standard Turbo Pascal
objects. It also implements File and Memory stream objects, as well as sorted
and non-sorted collections, and string streams.
\item[objpas] Is used for Delphi compatibility. You should never load this
unit explicitly; it is automatically loaded if you request Delphi mode.
\item[printer]  This unit provides all you need for rudimentary access
to the printer using standard I/O routines.
\item[sockets] This gives the programmer access to sockets and TCP/IP 
programming.
\item[strings] This unit provides basic string handling routines for the
\var{pchar} type, comparable to similar routines in standard \var{C}
libraries.
\item[system] This unit is available for all supported platforms. It includes
among others, basic file I/O routines, memory management routines, all compiler 
helper routines, and directory services routines. 
\item[strutils] Offers a lot of extended string handling routines. 
\item[dateutils] Offers a lot of extended date/time handling routines for 
almost any date and time math. 
\item[sysutils] Is an alternative implementation of the sysutils unit of
Delphi. It includes file I/O access routines which takes care of file
locking, date and string handling routines, file search, date and string 
conversion routines.
\item[typinfo] Provides functions to access runtime Type Information, just
like Delphi.
\item[variants] Provides basic variant handling.
\item[video] Provides basic screen handling in a platform independent way,
and supports writing custom drivers.
\end{description}

%
% Under DOS
%
\section{Under DOS}
\begin{description}
\item [emu387] This unit provides support for the coprocessor emulator.
\item [go32] This unit provides access to capabilities of the \var{GO32}
\dos extender.
\end{description}

%
% Under Windows
%
\section{Under Windows}
\begin{description}
\item[wincrt] This implements a console in a standard GUI window, contrary
to the \var{crt} unit which is for the Windows console only.
\item[Windows] This unit provides access to all Win32 API calls. Effort has
been taken to make sure that it is compatible to the Delphi version of this
unit, so code for Delphi is easily ported to \fpc.
\item[opengl] Provides access to the low-level opengl functions in \windows.
\item[winmouse] Provides access to the mouse in \windows.
\item[ole2] Provides access to the OLE capabilities of \windows.
\item[winsock] Provides access to the \windows sockets API Winsock.
\item[Jedi windows header translations] The units containing the Jedi
translations of the Windows API headers is also distributed with Free
Pascal. The names of these units start with \file{jw}, followed by the 
name of the particular API.
\end{description}

%
% Under Linux
%
\section{Under Linux and BSD-like platforms}
\begin{description}
\item[baseunix] Basic Unix operations, basically a subset of the POSIX specification. 
Using this unit should ensure portability across most unix systems.
\item[clocale] This unit initializes the internationalization settings in
the \file{sysutils} unit with settings obtained through the C library.
\item[cthreads] This unit should be specified as the first or second
unit in the uses clause of your program: it will use the Posix threads
implementation to enable threads in your FPC program.
\item[cwstring] If widestring routines are used, then this unit should 
be inserted as one of the first units in the uses clause of your program:
it will initialize the widestring manager in the system unit with routines
that use C library functions to handle Widestring conversions and other
widestring operations.
\item[errors] Returns a string describing an operating system error code.
\item[Libc] This is the interface to GLibc on a linux I386 system. It will
{\em not} work for other platforms, and is in general provided for Kylix
compatibility. 
\item[oldlinux] \emph{This unit is deprecated}. This unit provides access to the
\linux operating system. It provides most file and I/O handling routines
that you may need. It implements most of the standard \var{C} library constructs
that you will find on a Unix system. It is recommended, however, that you
use the \file{baseunix}, \file{unixtype} and \file{unix} units. They are
more portable.
\item[ports] This implements the various \var{port[]} constructs. These are
provided for compatibility only, and it is not recommended to use them
extensively. Programs using this construct must be run as ruit or setuid
root, and are a serious security risk on your system.
\item[termio] Terminal control routines, which are compatible to the C
library routines.
\item[unix] Extended Unix operations.
\item[unixtype] All types used commonly on Unix platforms.
\end{description}

\section{Under OS/2}
\begin{description}
\item[doscalls] Interface to \file{doscalls.dll}.
\item[dive] Interface to \file{dive.dll}
\item[emx] Provides access to the EMX extender.
\item[pm*] Interface units for the program manager functions.
\item[viocalls] Interface to \file{viocalls.dll} screen handling library.
\item[moucalls] Interface to \file{moucalls.dll} mouse handling library.
\item[kbdcalls] Interface to \file{kbdcalls.dll} keyboard handling library.
\item[moncalls] Interface to \file{moncalls.dll} monitoring handling library.
\end{description}

\section{Unit availability}

Standard unit availability for each of the supported platforms 
is given in the FAQ / Knowledge base. 

%%%%%%%%%%%%%%%%%%%%%%%%%%%%%%%%%%%%%%%%%%%%%%%%%%%%%%%%%%%%%%%%%%%%%
% Debugging
%%%%%%%%%%%%%%%%%%%%%%%%%%%%%%%%%%%%%%%%%%%%%%%%%%%%%%%%%%%%%%%%%%%%%

\chapter{Debugging your programs}

\fpc supports debug information for the \gnu debugger \var{gdb}, or
its derivatives \file{Insight} on win32 or \file{ddd} on \linux.
It can write 2 kinds of debug information:
\begin{description}
\item[stabs] The old debug information format.
\item[dwarf] The new debug information format.
\end{description}
Both are understood by GDB.

This chapter briefly describes how to use this feature. It doesn't attempt
to describe completely the \gnu debugger, however.
For more information on the workings of the \gnu debugger, see the \var{GDB}
User Manual.

\fpc also suports \var{gprof}, the \gnu profiler. See section \ref{se:gprof}
for more information on profiling.

%%%%%%%%%%%%%%%%%%%%%%%%%%%%%%%%%%%%%%%%%%%%%%%%%%%%%%%%%%%%%%%%%%%%%%%
% Compiling your program with debugger support
\section{Compiling your program with debugger support}
First of all, you must be sure that the compiler is compiled with debugging
support. Unfortunately, there is no way to check this at run time, except by
trying to compile a program with debugging support.

To compile a program with debugging support, just specify the \var{-g}
option on the command line, as follows:
\begin{verbatim}
fpc -g hello.pp
\end{verbatim}
This will incorporate debugging information in the executable generated 
from your program source. You will notice that the size of the executable 
increases substantially because of this\footnote{A good reason not to include debug
information in an executable you plan to distribute.}.

Note that the above will only incorporate debug information {\em for the code
that has been generated} when compiling \file{hello.pp}. This means that if
you used some units (the system unit, for instance) which were not compiled
with debugging support, no debugging support will be available for the code
in these units.

There are 2 solutions for this problem.
\begin{enumerate}
\item Recompile all units manually with the \var{-g} option.
\item Specify the 'build' option (\var{-B}) when compiling with debugging
support. This will recompile all units, and insert debugging information in
each of the units.
\end{enumerate}
The second option may have undesirable side effects. It may be that some
units aren't found, or compile incorrectly due to missing conditionals,
etc.

If all went well, the executable now contains the necessary information with
which you can debug it using \gnu \var{gdb}.


%%%%%%%%%%%%%%%%%%%%%%%%%%%%%%%%%%%%%%%%%%%%%%%%%%%%%%%%%%%%%%%%%%%%%%%
% Using gdb
\section{Using \var{gdb} to debug your program}
\label{se:usinggdb}

To use gdb to debug your program, you can start the debugger, and give it as
an option the {\em full} name of your program:
\begin{verbatim}
gdb hello
\end{verbatim}
Or, under \dos :
\begin{verbatim}
gdb hello.exe
\end{verbatim}

This starts the debugger, and the debugger immediately loads your program
into memory, but it does not run the program yet. Instead, you are presented
with the following (more or less) message, followed by the \var{gdb} prompt
\var{'(gdb)'}:
\begin{verbatim}
GNU gdb 6.6.50.20070726-cvs
Copyright (C) 2007 Free Software Foundation, Inc.
GDB is free software, covered by the GNU General Public License, and you are
welcome to change it and/or distribute copies of it under certain conditions.
Type "show copying" to see the conditions.
There is absolutely no warranty for GDB.  Type "show warranty" for details.
This GDB was configured as "x86_64-suse-linux".
(gdb)
\end{verbatim}
The actual prompt will vary depending on  your operating system and
installed version of gdb, of course.

To start the program you can use the \var{run} command. You can optionally
specify command line parameters, which will then be fed to your program, for
example:
\begin{verbatim}
(gdb) run -option -anotheroption needed_argument
\end{verbatim}
If your program runs without problems, \var{gdb} will inform you of this,
and return the exit code of your program. If the exit code was zero, then
the message \var{'Program exited normally'} is displayed.

If something went wrong (a segmentation fault or such), \var{gdb} will stop
the execution of your program, and inform you of this with an appropriate
message. You can then use the other \var{gdb} commands to see what happened.
Alternatively, you can instruct \var{gdb} to stop at a certain point in your
program, with the \var{break} command.

Here is a short list of \var{gdb} commands, which you are likely to need when
debugging your program:
\begin{description}
\item [quit\ ] Exit the debugger.
\item [kill\ ] Stop a running program.
\item [help\ ] Give help on all \var{gdb} commands.
\item [file\ ] Load a new program into the debugger.
\item [directory\ ] Add a new directory to the search path for source
files.\\
\begin{remark} 
My copy of gdb needs '.' to be added explicitly to the search
path, otherwise it doesn't find the sources.
\end{remark}
\item [list\ ] List the program sources in chunks of 10 lines. As an option you can
specify a line number or function name.
\item [break\ ] Set a breakpoint at a specified line or function.
\item [awatch\ ] Set a watch-point for an expression. A watch-point stops
execution of your program whenever the value of an expression is either
read or written.
\end{description}

In appendix {\ref{ch:GdbIniFile}} a sample init file for
\var{gdb} is presented. It produces good results when debugging \fpc programs.

For more information, refer to the \var{gdb} User Manual, or use the 
\var{'help'} function in \var{gdb}.

The text mode IDE and Lazarus both use GDB as a debugging backend. It may
be preferable to use that, as they hide much of the details of the debugger
in an easy-to-use user interface.

\section{Caveats when debugging with \var{gdb}}

There are some peculiarities of \fpc which you should be aware of when using
\var{gdb}. We list the main ones here:
\begin{enumerate}
\item \fpc generates information for GDB in uppercase letters. This is a
consequence of the fact that Pascal is a case insensitive language. So, when
referring to a variable or function, you need to make its name all
uppercase.

As an example, if you want to watch the value of a loop variable
\var{count}, you should type
\begin{verbatim}
watch COUNT
\end{verbatim}
Or if you want to stop when a certain function (e.g \var{MyFunction}) is called,
type
\begin{verbatim}
break MYFUNCTION
\end{verbatim}

\item \var{gdb} does not know sets.

\item \var{gdb} doesn't know strings. Strings are represented in \var{gdb}
as records with a length field and an array of char containing the string.

You can also use the following user function to print strings:
\begin{verbatim}
define pst
set $pos=&$arg0
set $strlen = {byte}$pos
print {char}&$arg0.st@($strlen+1)
end

document pst
  Print out a Pascal string
end
\end{verbatim}
If you insert it in your \file{gdb.ini} file, you can look at a string with this
function. There is a sample \file{gdb.ini} in appendix \ref{ch:GdbIniFile}.

\item Objects are difficult to handle, mainly because \var{gdb} is oriented
towards C and C++. The workaround implemented in \fpc is that object methods
are represented as functions, with an extra parameter \var{this} (all
lowercase!). The name of this function is a concatenation of the object type
and the function name, separated by two underscore characters.

For example, the method \var{TPoint.Draw} would be converted to
\var{TPOINT\_\_DRAW}, and you could stop at it by using:
\begin{verbatim}
break TPOINT__DRAW
\end{verbatim}

\item Global overloaded functions confuse \var{gdb} because they have the same
name. Thus you cannot set a breakpoint at an overloaded function, unless you
know its line number, in which case you can set a breakpoint at the
starting line number of the function.
\end{enumerate}

%%%%%%%%%%%%%%%%%%%%%%%%%%%%%%%%%%%%%%%%%%%%%%%%%%%%%%%%%%%%%%%%%%%%%%%
% Using gprof
\section{Support for \var{gprof}, the \gnu profiler}
\label{se:gprof}

You can compile your programs with profiling support. For this, you just
have to use the compiler switch \var{-pg}. The compiler will insert the
necessary stuff for profiling.

When you have done this, you can run your program as you would normally 
run it:
\begin{verbatim}
yourexe
\end{verbatim}
Where \file{yourexe} is the name of your executable.

When your program finishes, a file called gmon.out is generated. Then you can start
the profiler to see the output. You can benefit from redirecting the output to a file,
because it could be quite a lot:
\begin{verbatim}
gprof yourexe > profile.log
\end{verbatim}

Hint: you can use the \var{--flat} option to reduce the amount of output of gprof. It will
then only output the information about the timings.

For more information on the \gnu profiler \var{gprof}, see its manual.

%%%%%%%%%%%%%%%%%%%%%%%%%%%%%%%%%%%%%%%%%%%%%%%%%%%%%%%%%%%%%%%%%%%%%%%
% Checking the heap
\section{Detecting heap memory leaks}
\label{se:heaptrc}
\fpc has a built in mechanism to detect memory leaks. There is a plug-in
unit for the memory manager that analyses the memory allocation/deallocation
and prints a memory usage report after the program exits.

The unit that does this is called \file{heaptrc}. If you want to use it,
you should include it as the first unit in your uses clause. Alternatively,
you can supply the \var{-gh} switch to the compiler, and it will include
the unit automatically for you.

After the program exits, you will get a report looking like this:
\begin{verbatim}
Marked memory at 0040FA50 invalid
Wrong size : 128 allocated 64 freed
  0x00408708
  0x0040CB49
  0x0040C481
Call trace for block 0x0040FA50 size 128
  0x0040CB3D
  0x0040C481
\end{verbatim}
The output of the heaptrc unit is customizable by setting some variables.
Output can also be customized using environment variables.

You can find more information about the usage of the \file{heaptrc} unit
in the \unitsref.

%%%%%%%%%%%%%%%%%%%%%%%%%%%%%%%%%%%%%%%%%%%%%%%%%%%%%%%%%%%%%%%%%%%%%%%
% Verbose Run-time errors.
\section{Line numbers in run-time error backtraces}
\label{se:lineinfo}

Normally, when a run-time error occurs, you are presented with a list
of addresses that represent the call stack backtrace, i.e. the addresses
of all procedures that were invoked when the run-time error occurred.

This list is not very informative, so there exists a unit that generates
the file names and line numbers of the called procedures using the
addresses of the stack backtrace. This unit is called lineinfo.

You can use this unit by giving the \var{-gl} option to the compiler. The
unit will be automatically included. It is also possible to use the unit
explicitly in your \var{uses} clause, but you must make sure that you
compile your program with debug info.

Here is an example program:
\begin{verbatim}
program testline;

procedure generateerror255;

begin
  runerror(255);
end;

procedure generateanerror;

begin
  generateerror255;
end;

begin
  generateanerror;
end.
\end{verbatim}
When compiled with \var{-gl}, the following output is generated:
\begin{verbatim}
Runtime error 255 at 0x0040BDE5
  0x0040BDE5  GENERATEERROR255,  line 6 of testline.pp
  0x0040BDF0  GENERATEANERROR,  line 13 of testline.pp
  0x0040BE0C  main,  line 17 of testline.pp
  0x0040B7B1
\end{verbatim}
This is more understandable than the normal message. Make sure that all
units you use are compiled with debug info, because if they are not, no
line number and filename can be found.

%%%%%%%%%%%%%%%%%%%%%%%%%%%%%%%%%%%%%%%%%%%%%%%%%%%%%%%%%%%%%%%%%%%%%%%
% Combining heaptrc and lineinfo
\section{Combining \file{heaptrc} and \file{lineinfo}}

If you combine the lineinfo and the heaptrc information, then the output
of the \file{heaptrc} unit will contain the names of the files and line
numbers of the procedures that occur in the stack backtrace.

In such a case, the output will look something like this:
\begin{verbatim}
Marked memory at 00410DA0 invalid
Wrong size : 128 allocated 64 freed
  0x004094B8
  0x0040D8F9  main,  line 25 of heapex.pp
  0x0040D231
Call trace for block 0x00410DA0 size 128
  0x0040D8ED  main,  line 23 of heapex.pp
  0x0040D231
\end{verbatim}
If lines without filename / line number occur, this means there is a unit which
has no debug info included (in the above case, the getmem call itself).

%%%%%%%%%%%%%%%%%%%%%%%%%%%%%%%%%%%%%%%%%%%%%%%%%%%%%%%%%%%%%%%%%%%%%
%%%%%%%%%%%%%%%%%%%%%%%%%%%%%%%%%%%%%%%%%%%%%%%%%%%%%%%%%%%%%%%%%%%%%
%% APPENDICES.
%%%%%%%%%%%%%%%%%%%%%%%%%%%%%%%%%%%%%%%%%%%%%%%%%%%%%%%%%%%%%%%%%%%%%
%%%%%%%%%%%%%%%%%%%%%%%%%%%%%%%%%%%%%%%%%%%%%%%%%%%%%%%%%%%%%%%%%%%%%

\appendix
%%%%%%%%%%%%%%%%%%%%%%%%%%%%%%%%%%%%%%%%%%%%%%%%%%%%%%%%%%%%%%%%%%%%%
% APPENDIX A.
%%%%%%%%%%%%%%%%%%%%%%%%%%%%%%%%%%%%%%%%%%%%%%%%%%%%%%%%%%%%%%%%%%%%%

\chapter{Alphabetical listing of command line options}
\label{ch:commandlineoptions}
The following is an alphabetical listing of all command line options, as
generated by the compiler:
\input{comphelp.inc}

%%%%%%%%%%%%%%%%%%%%%%%%%%%%%%%%%%%%%%%%%%%%%%%%%%%%%%%%%%%%%%%%%%%%%
% APPENDIX B.
%%%%%%%%%%%%%%%%%%%%%%%%%%%%%%%%%%%%%%%%%%%%%%%%%%%%%%%%%%%%%%%%%%%%%

\chapter{Alphabetical list of reserved words}
\label{ch:reserved}
\begin{multicols}{3}
\begin{verbatim}
absolute
abstract
and
array
as
asm
assembler
begin
break
case
cdecl
class
const
constructor
continue
destructor
dispose
div
do
downto
else
end
except 
exit
export
exports
external
fail
false
far
file
finally 
for
forward
function
goto
if
implementation
in
index
inherited
initialization 
inline
interface
interrupt
is
label
library
mod
name
near
new
nil
not
object
of
on
operator
or
otherwise
packed
popstack
private
procedure
program
property 
protected
public
raise
record
repeat
self
set
shl
shr
stdcall
string
then
to
true
try
type
unit
until
uses
var
virtual
while
with
xor
\end{verbatim}

\end{multicols}

%%%%%%%%%%%%%%%%%%%%%%%%%%%%%%%%%%%%%%%%%%%%%%%%%%%%%%%%%%%%%%%%%%%%%
% APPENDIX C.
%%%%%%%%%%%%%%%%%%%%%%%%%%%%%%%%%%%%%%%%%%%%%%%%%%%%%%%%%%%%%%%%%%%%%

\chapter{Compiler messages}
\label{ch:ErrorMessages}
This appendix is meant to list all the compiler messages. The list of
messages is generated from he compiler source itself, and should be fairly
complete. At this point, only assembler errors are not in the list.

% Message file is generated with msg2inc.
\input {messages.inc}

%%%%%%%%%%%%%%%%%%%%%%%%%%%%%%%%%%%%%%%%%%%%%%%%%%%%%%%%%%%%%%%%%%%%%%%
% Assembler reader errors
\section{Assembler reader errors.}

This section lists the errors that are generated by the inline assembler reader.
They are {\em not} the messages of the assembler itself.

% General assembler errors.
\subsection{General assembler errors}
\begin{description}
\item [Divide by zero in asm evaluator]
This fatal error is reported when a constant assembler expression
performs a division by zero.

\item [Evaluator stack overflow, Evaluator stack underflow]
These fatal error is reported when a constant assembler expression
is too big to be evaluated by the constant parser. Try reducing the
number of terms.

\item [Invalid numeric format in asm evaluator]
This fatal error is reported when a non-numeric value is detected
by the constant parser. Normally this error should never occur.

\item [Invalid Operator in asm evaluator]
This fatal error is reported when a mathematical operator is detected
by the constant parser. Normally this error should never occur.

\item [Unknown error in asm evaluator]
This fatal error is reported when an internal error is detected
by the constant parser. Normally this error should never occur.

\item [Invalid numeric value]
This warning is emitted when a conversion from octal, binary 
or hexadecimal to decimal is outside of the supported range.

\item [Escape sequence ignored]
This error is emitted when a non ANSI C escape sequence is detected in
a C string.

\item [Asm syntax error - Prefix not found]
This occurs when trying to use a non-valid prefix instruction.

\item [Asm syntax error - Trying to add more than one prefix]
This occurs when you try to add more than one prefix instruction.

\item [Asm syntax error - Opcode not found]
You have tried to use an unsupported or unknown opcode.

\item [Constant value out of bounds]
This error is reported when the constant parser determines that the
value you are using is out of bounds, either with the opcode or with
the constant declaration used.

\item [Non-label pattern contains @]
This only applied to the m68k and Intel styled assembler. 
This is reported when you try to use a non-label identifier with an '@' prefix.
\item [Internal error in Findtype()]
\item [Internal Error in ConcatOpcode()]
\item [Internal Errror converting binary]
\item [Internal Errror converting hexadecimal]
\item [Internal Errror converting octal]
\item [Internal Error in BuildScaling()]
\item [Internal Error in BuildConstant()]
\item [internal error in BuildReference()]
\item [internal error in HandleExtend()]
\item [Internal error in ConcatLabeledInstr()]
\label{InternalError}
These errors should never occur. If they do then you have found
a new bug in the assembler parsers. Please contact one of the
developers.
\item [Opcode not in table, operands not checked]
This warning only occurs when compiling the system unit, or related
files. No checking is performed on the operands of the opcodes.

\item [@CODE and @DATA not supported]
This Turbo Pascal construct is not supported.
\item [SEG and OFFSET not supported]
This Turbo Pascal construct is not supported.
\item [Modulo not supported]
Modulo constant operation is not supported.
\item [Floating point binary representation ignored]
\item [Floating point hexadecimal representation ignored]
\item [Floating point octal representation ignored]
These warnings occur when a floating point constant is declared in
a base other than decimal. No conversion can be done on these formats.
You should use a decimal representation instead.
\item [Identifier supposed external]
This warning occurs when a symbol is not found in the symbol table. 
It is therefore considered external.
\item [Functions with void return value can't return any value in asm code]
Only routines with a return value can have a return value set.

\item [Error in binary constant]
\item [Error in octal constant]
\item [Error in hexadecimal constant]
\item [Error in integer constant]
\label{ErrorConst}
These errors are reported when you tried using a constant
expression that is invalid or whose value is out of range.

\item [Invalid labeled opcode]
\item [Asm syntax error - error in reference]
\item [Invalid Opcode]
\item [Invalid combination of opcode and operands]
\item [Invalid size in reference]
\item [Invalid middle sized operand]
\item [Invalid three operand opcode]
\item [Assembler syntax error]
\item [Invalid operand type]
You tried using an invalid combination of opcode and operands. Check the syntax
and if you are sure it is correct, please contact one of the developers.

\item [Unknown identifier]
The identifier you are trying to access does not exist, or is not within the
current scope.

\item [Trying to define an index register more than once]
\item [Trying to define a segment register twice]
\item [Trying to define a base register twice]
You are trying to define an index/segment register more than once.

\item [Invalid field specifier]
The record or object field you are trying to access does not exist, or
is incorrect.

\item [Invalid scaling factor]
\item [Invalid scaling value]
\item [Scaling value only allowed with index]
Allowed scaling values are 1,2,4 or 8.


\item [Cannot use SELF outside a method]
You are trying to access the SELF identifier for objects outside a method.


\item [Invalid combination of prefix and opcode]
This opcode cannot be prefixed by this instruction.

\item [Invalid combination of override and opcode]
This opcode cannot be overriden by this combination.

\item [Too many operands on line]
At most three operand instructions exist on the m68k, and i386, you
are probably trying to use an invalid syntax for this opcode.

\item [Duplicate local symbol]
You are trying to redefine a local symbol, such as a local label.

\item [Unknown label identifer]
\item [Undefined local symbol]
\item [local symbol not found inside asm statement]
This label does not seem to have been defined in the current scope.


\item [Assemble node syntax error]
\item [Not a directive or local symbol]
The assembler statement is invalid, or you are not using a recognized
directive.

\end{description}

% I386 specific errors
\subsection{I386 specific errors}

\begin{description}
\item [repeat prefix and a segment override on \var{<=} i386 ...]
A problem with interrupts and a prefix instruction may occur and may cause
false results on 386 and earlier computers.

\item [Fwait can cause emulation problems with emu387]
This warning is reported when using the FWAIT instruction. It can
cause emulation problems on systems which use the em387.dxe emulator.

\item [You need GNU as version >= 2.81 to compile this MMX code]
MMX assembler code can only be compiled using GAS v2.8.1 or later.

\item [NEAR ignored]
\item [FAR ignored]
\label{FarIgnored}
\var{NEAR} and \var{FAR} are ignored in the Intel assemblers, but 
are still accepted for compatiblity with the 16-bit code model.

\item [Invalid size for MOVSX/MOVZX]

\item [16-bit base in 32-bit segment]
\item [16-bit index in 32-bit segment]
16-bit addressing is not supported. You must use 32-bit addressing.


\item [Constant reference not allowed]
It is not allowed to try to address a constant memory address in protected
mode.

\item [Segment overrides not supported]
Intel style (eg: rep ds stosb) segment overrides are not supported by
the assembler parser.

\item [{Expressions of the form [sreg:reg...] are currently not supported}]
To access a memory operand in a different segment, you should use the
sreg:[reg...] snytax instead of [sreg:reg...]

\item [Size suffix and destination register do not match]
In intel AT\&T syntax, you are using a register size which does
not concord with the operand size specified.

\item [Invalid assembler syntax. No ref with brackets]
\item [ Trying to use a negative index register ]
\item [ Local symbols not allowed as references ]
\item [ Invalid operand in bracket expression ]
\item [ Invalid symbol name:  ]
\item [ Invalid Reference syntax ]
\item [ Invalid string as opcode operand: ]
\item [ Null label references are not allowed ]
\item [ Using a defined name as a local label ]
\item [ Invalid constant symbol  ]
\item [ Invalid constant expression ]
\item [ / at beginning of line not allowed ]
\item [ NOR not supported ]
\item [ Invalid floating point register name ]
\item [ Invalid floating point constant:  ]
\item [ Asm syntax error - Should start with bracket ]
\item [ Asm syntax error - register:  ]
\item [ Asm syntax error - in opcode operand ]
\item [ Invalid String expression ]
\item [ Constant expression out of bounds ]
\item [ Invalid or missing opcode ]
\item [ Invalid real constant expression ]
\item [ Parenthesis are not allowed ]
\item [ Invalid Reference ]
\item [ Cannot use \_\_SELF outside a method ]
\item [ Cannot use \_\_OLDEBP outside a nested procedure ]
\item [ Invalid segment override expression ]
\item [ Strings not allowed as constants ]
\item [ Switching sections is not allowed in an assembler block ]
\item [ Invalid global definition ]
\item [ Line separator expected ]
\item [ Invalid local common definition ]
\item [ Invalid global common definition ]
\item [ assembler code not returned to text ]
\item [ invalid opcode size ]
\item [ Invalid character: < ]
\item [ Invalid character: > ]
\item [ Unsupported opcode ]
\item [ Invalid suffix for intel assembler ]
\item [ Extended not supported in this mode ]
\item [ Comp not supported in this mode ]
\item [ Invalid Operand: ]
\item [ Override operator not supported ]
\end{description}

% m68k specific errors
\subsection{m68k specific errors.}
\begin{description}
\item [Increment and Decrement mode not allowed together]
You are trying to use dec/inc mode together.

\item [Invalid Register list in movem/fmovem]
The register list is invalid. Normally a range of registers should
be separated by - and individual registers should be separated by
a slash.
\item [Invalid Register list for opcode]
\item [68020+ mode required to assemble]
\end{description}

%%%%%%%%%%%%%%%%%%%%%%%%%%%%%%%%%%%%%%%%%%%%%%%%%%%%%%%%%%%%%%%%%%%%%%%
% Runtime errors listing
%%%%%%%%%%%%%%%%%%%%%%%%%%%%%%%%%%%%%%%%%%%%%%%%%%%%%%%%%%%%%%%%%%%%%%%
\chapter{Run-time errors}

Applications generated  by \fpc might generate run-time errors when certain 
abnormal conditions are detected in the application. This appendix lists the 
possible run-time errors and gives information on why they might be produced.

\begin{description}
\item [1  Invalid function number]
An invalid operating system call was attempted.

\item [2  File not found]
Reported when trying to erase, rename or open a non-existent
file.

\item [3  Path not found]
Reported by the directory handling routines when a path does not 
exist or is invalid. Also reported when trying to access a 
non-existent file.

\item [4  Too many open files]
The maximum number of files currently opened by your process
has been reached. Certain operating systems limit the number
of files which can be opened concurrently, and this error
can occur when this limit has been reached.

\item [5  File access denied]
Permission to access the file is denied. This error might
be caused by one of several reasons:
\begin{itemize}
\item Trying to open for writing a file which is
read-only, or which is actually a directory.
\item File is currently locked or used by another process.
\item Trying to create a new file, or directory while a 
file or directory of the same name already exists.
\item Trying to read from a file which was opened in write-only mode.
\item Trying to write from a file which was opened in read-only mode.
\item Trying to remove a directory or file while it is not possible.
\item No permission to access the file or directory.        
\end{itemize}

\item [6  Invalid file handle]
If this happens, the file variable you are using is trashed; it
indicates that your memory is corrupted.

\item [12  Invalid file access code]
Reported when a reset or rewrite is called with an invalid \var{FileMode}
value.

\item [15  Invalid drive number]
The number given to the \var{Getdir} or \var{ChDir} function specifies a 
non-existent disk.

\item [16  Cannot remove current directory]
Reported when trying to remove the currently active directory.

\item [17  Cannot rename across drives]
You cannot rename a file such that it would end up on another disk or
partition.

\item [100  Disk read error]
An error occurred when reading from disk. Typically happens when you try
to read past the end of a file.

\item [101  Disk write error]
Reported when the disk is full, and you're trying to write to it.

\item [102  File not assigned]
This is reported by \var{Reset}, \var{Rewrite}, \var{Append}, 
\var{Rename} and \var{Erase}, if you call
them with an unassigned file as a parameter.

\item [103  File not open]
Reported by the following functions : \var{Close, Read, Write, Seek,
EOf, FilePos, FileSize, Flush, BlockRead,} and \var{BlockWrite} if the 
file is not open.

\item [104  File not open for input]
Reported by \var{Read, BlockRead, Eof, Eoln, SeekEof} or \var{SeekEoln} if 
the file is not opened with \var{Reset}.

\item [105  File not open for output]
Reported by write if a text file isn't opened with \var{Rewrite}.

\item [106  Invalid numeric format]
Reported when a non-numeric value is read from a text file, and a numeric
value was expected.

\item [150  Disk is write-protected]
(Critical error)
\item [151  Bad drive request struct length]
(Critical error)
\item [152  Drive not ready]
(Critical error)
\item [154  CRC error in data]
(Critical error)
\item [156  Disk seek error]
(Critical error)
\item [157  Unknown media type]
(Critical error)
\item [158  Sector Not Found]
(Critical error)
\item [159  Printer out of paper]
(Critical error)
\item [160  Device write fault]
(Critical error)
\item [161  Device read fault]
(Critical error)
\item [162  Hardware failure]
(Critical error)
\item [200  Division by zero]
The application attempted to divide a number by zero.
\item [201  Range check error]
If you compiled your program with range checking on, then you can get this
error in the following cases:
\begin{enumerate}
\item An array was accessed with an index outside its declared range.
\item Trying to assign a value to a variable outside its range (for
instance an enumerated type).
\end{enumerate}
\item [202  Stack overflow error]
The stack has grown beyond its maximum size (in which case the size of 
local variables should be reduced to avoid this error), or the stack has 
become corrupt. This error is only reported when stack checking is enabled.
\item [203  Heap overflow error]
The heap has grown beyond its boundaries. This is caused when trying to allocate
memory explicitly with \var{New}, \var{GetMem} or \var{ReallocMem}, or when
a class or object instance is created and no memory is left. Please note 
that, by default, \fpc provides a growing heap, i.e. the heap will
try to allocate more memory if needed. However, if the heap has reached the
maximum size allowed by the operating system or hardware, then you will get
this error.
\item [204  Invalid pointer operation]
You will get this if you call \var{Dispose} or \var{Freemem} with an invalid 
pointer (notably, \var{Nil}).
\item [205  Floating point overflow]
You are trying to use or produce real numbers that are too large.
\item [206  Floating point underflow]
You are trying to use or produce real numbers that are too small.
\item [207  Invalid floating point operation]
Can occur if you try to calculate the square root or logarithm of a negative
number.
\item [210  Object not initialized]
When compiled with range checking on, a program will report this error if
you call a virtual method without having called its object's constructor.
\item [211  Call to abstract method]
Your program tried to execute an abstract virtual method. Abstract methods
should be overridden, and the overriding method should be called.
\item [212  Stream registration error]
This occurs when an invalid type is registered in the objects unit.
\item [213  Collection index out of range]
You are trying to access a collection item with an invalid index
(\var{objects} unit).
\item [214  Collection overflow error]
The collection has reached its maximal size, and you are trying to add
another element (\var{objects} unit).
\item[215 Arithmetic overflow error]
This error is reported when the result of an arithmetic operation
is outside of its supported range. Contrary to Turbo Pascal, this error
is only reported for 32-bit or 64-bit arithmetic overflows. This is due
to the fact that everything is converted to 32-bit or 64-bit before
doing the actual arithmetic operation.
\item [216  General Protection fault]
The application tried to access invalid memory space. This can
be caused by several problems:
\begin{enumerate}
 \item Dereferencing a \var{nil} pointer.
 \item Trying to access memory which is out of bounds
       (for example, calling \var{move} with an invalid length).
\end{enumerate}

\item [217 Unhandled exception occurred]
An exception occurred, and there was no exception handler present.
The \var{sysutils} unit installs a default exception handler which catches
all exceptions and exits gracefully.

\item [219 Invalid typecast]

Thrown when an invalid typecast is attempted on a class using the \var{as}
operator. This error is also thrown when an object or class is
typecast to an invalid class or object and a virtual method of
that class or object is called. This last error is only detected
if the \var{-CR} compiler option is used.


\item[222 Variant dispatch error]
No dispatch method to call from variant.

\item[223 Variant array create]
The variant array creation failed. Usually when there is not enough memory.

\item[224 Variant is not an array]
This error occurs when a variant array operation is attempted on a variant
which is not an array.

\item[225 Var Array Bounds check error]
This error occurs when a variant array index is out of bounds.

\item [227 Assertion failed error]
An assertion failed, and no \var{AssertErrorProc} procedural variable was 
installed.

\item [229 Safecall error check]
This error occurs is a safecall check fails, and no handler routine is
available.

\item [231 Exception stack corrupted]
This error occurs when the exception object is retrieved and none is
available.

\item [232 Threads not supported]
Thread management relies on a separate driver on some operating systems
(notably, Unixes). The unit with this driver needs to be specified on
the uses clause of the program, preferably as the first unit
(\file{cthreads} on unix).

\end{description}

%%%%%%%%%%%%%%%%%%%%%%%%%%%%%%%%%%%%%%%%%%%%%%%%%%%%%%%%%%%%%%%%%%%%%%%%%%%%%%%%


%%%%%%%%%%%%%%%%%%%%%%%%%%%%%%%%%%%%%%%%%%%%%%%%%%%%%%%%%%%%%%
%  GDB Configuration file
\chapter{A sample \file{gdb.ini} file}
\label{ch:GdbIniFile}

Here you have a sample \file{gdb.ini} file listing, which gives better
results when using \var{gdb}. Under \linux you should put this in a
\file{.gdbinit} file in your home directory or the current directory.

\begin{verbatim}
set print demangle off
set gnutarget auto
set verbose on
set complaints 1000
dir ./rtl/dosv2
set language c++
set print vtbl on
set print object on
set print sym on
set print pretty on
disp /i $eip

define pst
set $pos=&$arg0
set $strlen = {byte}$pos
print {char}&$arg0.st@($strlen+1)
end

document pst
  Print out a Pascal string
end
\end{verbatim}

%%%%%%%%%%%%%%%%%%%%%%%%%%%%%%%%%%%%%%%%%%%%%%%%%%%%%%%%%%%%%%%%%%%%%%%
% Options summary tables
%%%%%%%%%%%%%%%%%%%%%%%%%%%%%%%%%%%%%%%%%%%%%%%%%%%%%%%%%%%%%%%%%%%%%%%
\chapter{Options and settings}
In \seet{booloptions} a summary of available boolean compiler directives 
and the corresponding command line options are listed. Other directives and
the corresponding options are shown in \seet{options}. For more information
about the command-line options, see \seec{CompilerConfiguration}. For more
information about the directives, see the \progref.

% Directive
\newcommand{\dir}[1]{\var{\$#1}}
% Boolean directive (short)
\newcommand{\bdir}[1]{\var{\$#1[+/-]}}
% Boolean directive (long)
\newcommand{\lbdir}[1]{\var{\$#1[ON/OFF]}}
% Command line option.
\newcommand{\copt}[1]{\var{-#1}}
\begin{FPCltable}{llll}{Boolean Options and directves}{booloptions}
Short & long & Opt & Explanation \\ \hline
\bdir{A} & \lbdir{ALIGN} &  & Data alignment \\
\bdir{B} & \lbdir{BOOLEVAL} & & Boolean evaluation mode \\
\bdir{C} & \lbdir{ASSERTIONS} & \copt{Sa} & Include assertions \\
\bdir{D} & \lbdir{DEBUGINFO} & \copt{g} & Include debug info \\
\bdir{E} & & & Coprocessor emulation \\
\bdir{F} & & & Far or near function (ignored) \\
\bdir{G} & & & Generate 80286 code (ignored) \\
         & \lbdir{GOTO} & \copt{Sg} & Support \var{GOTO} and \var{Label}\\
         & \lbdir{HINTS} & \copt{vh} & Show hints \\
\bdir{H} & \lbdir{LONGSTRINGS} & \copt{Sh} & Use ansistrings\\
\bdir{I} & \lbdir{IOCHECKS} & \copt{Ci} & Check I/O operation result  \\
         & \lbdir{INLINE} & \copt{Si} & Allow inline code \\
\bdir{L} & \lbdir{LOCALSYMBOLS} & & Local symbol information \\
\bdir{M} & \lbdir{TYPEINFO} & & Generate RTTI for classes \\
         & \lbdir{MMX} & & Intel MMX support \\
\bdir{N} & & & Floating point support \\
         & \lbdir{NOTES} & \copt{vn} & Emit notes \\ 
\bdir{O} & & & Support overlays (ignored) \\
\bdir{P} & \lbdir{OPENSTRINGS} & & Support open strings \\
\bdir{Q} & \lbdir{OVERFLOWCHECKS} & \copt{Co} & Overflow checking \\
\bdir{R} & \lbdir{RANGECHEKS} & \copt{Cr} & Range checks \\
\bdir{S} & & \copt{Ct} & Stack checks \\ 
         & \lbdir{SMARTLINK} & \copt{CX} & Use smartlinking \\
         & \lbdir{STATIC} & \copt{St} & Allow use of \var{static} \\
\bdir{T} & \lbdir{TYPEDADDRESS} & & Typed addresses \\ \hline
%
%
\end{FPCltable}

\begin{FPCltable}{llll}{Options and directives}{options}
Short & long & Opt & Explanation \\ \hline
         & \dir{APPTYPE} & \copt{W} & Application type (Win32/OS2) \\
         & \dir{ASMMODE} & \copt{R} & Assembler reader mode \\
%%         & \dir{COPYRIGHT} & & Netware module copyright string \\
         & \dir{DEFINE} & \copt{d} & Define symbol \\
         & \dir{DESCRIPTION} & & Set program description \\ 
         & \dir{ELSE} & & Conditional compilation switch \\
         & \dir{ENDIF} & & Conditional compilation end \\
         & \dir{FATAL} & & Report fatal error \\
         & \dir{HINT} & & Emit hint message\\
\dir{I file} & \dir{INCLUDE} & & Include file or literal text \\
         & \dir{IF} & & Conditional compilation start \\
         & \dir{IFDEF NAME} & & Conditional compilation start \\
         & \dir{IFNDEF} & & Conditional compilation start \\
         & \dir{IFOPT} & & Conditional compilation start \\
         & \dir{INCLUDEPATH} & \copt{Fi} & Set include path \\
         & \dir{INFO} & & Emit information message \\
\dir{L file} & \dir{LINK} & & Link object file \\
         & \dir{LIBRARYPATH} & \copt{Fl} & Set library path\\
         & \dir{LINKLIB name} & & Link library \\
\dir{M MIN,MAX}  & \dir{MEMORY} & & Set memory sizes \\
         & \dir{MACRO} & \copt{Sm} & Allow use of macros \\
         & \dir{MESSAGE} & & Emit message \\
         & \dir{MODE} & & Set compatibility mode \\
         & \dir{NOTE} & & Emit note message \\
         & \dir{OBJECTPATH} & \copt{Fo} & Set object path \\
         & \dir{OUTPUT} & \copt{A} & Set output format \\
         & \dir{PACKENUM} & & Enumeration type size \\
         & \dir{PACKRECORDS} & & Record element alignment \\
         & \dir{SATURATION} & & Saturation (ignored) \\
%%         & \dir{SCREENNAME} & & Netware Module CLib-screenname \\
         & \dir{STOP} & &  Stop compilation \\
         & \dir{UNDEF} & \copt{u} & Undefine symbol \\ \hline
%%         & \dir{VERSION} & & Set Netware Module Version number \\
%
%        
\end{FPCltable}

\end{document}
